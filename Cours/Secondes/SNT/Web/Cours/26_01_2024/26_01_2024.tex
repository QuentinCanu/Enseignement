\documentclass{article}
\usepackage{main}

\title{Cous : Requête HTTP}
\author{Quentin Canu}
\date{26 Janvier 2024}

\begin{document}
\maketitle
\section{Rappels}
\begin{itemize}
\item IP/TCP : les protocoles d'internet.
\item HTTP et HTTPS : les protocoles du web.
\end{itemize}
\section{Illustration des requêtes HTTP}
\section{Cours}
Faire un schéma pour illustrer l'architecture clien-serveur. Différence entre TCP/IP et HTTP : enveloppe-lettre.

Les differents types de requêtes : \verb|GET|, \verb|HEAD| et \verb|POST|.

Une requête \verb|GET| typique ressemble à

\begin{verbatim}
GET /chemin/vers/le/contenu HTTP/1.1
Host: site.com
User-Agent: Chrome/5.0 (Windows 10)
Accept-Language: en-US
Accept-encoding: gzip
\end{verbatim}

Une réponse possible est 

\begin{verbatim}
HTTP/1.1 200 OK
Date: Mon, 27 Jul 2022 12:28:53 GMT
Server: Apache/2.2.14 (Win32)
Last-Modified: Wed, 22 Jul 2022 19:15:56 GMT
Content-Length: 88
Content-Type: text/html
Connection: Closed</code.
\end{verbatim}

Importance de la première ligne pour la requête puis pour la réponse. Présentation des entêtes. 

Faire des recherches sur les requêtes \verb|HEAD| et \verb|POST| ainsi que sur leur utilité.

\section{Devoir maison}
Utiliser les outils de développement d'un navigateur pour analyser les requêtes HTTP permettant d'afficher la page web sur le site de votre choix.

\end{document}