\documentclass{article}
\usepackage{main}

\title{Activité requêtes HTTP}
\date{26 Janvier 2024}
\author{}

\begin{document}
\maketitle
Une requête \verb|GET| typique ressemble à

\begin{verbatim}
GET /mondossier/monFichier.html HTTP/1.1
User-Agent : Mozilla/5.0
Accept : text/html
\end{verbatim}

Une réponse possible est 

\begin{verbatim}
HTTP/1.1 200 OK
Date: Thu, 15 feb 2019 12:02:32 GMT
Server: Apache/2.0.54 (Debian GNU/Linux) DAV/2 SVN/1.1.4
Connection: close
Transfer-Encoding: chunked
Content-Type: text/html; charset=ISO-8859-1
<!doctype html>
<html lang="fr">
<head>
<meta charset="utf-8">
<title>Voici mon site</title>
</head>
<body>
<h1>Hello World! Ceci est un titre</h1>
<p>Ceci est un <strong>paragraphe</strong>. Avez-vous bien compris ?</p>
</body>
</html>
\end{verbatim}

D'après vous,
\begin{enumerate}[label = \emph{\alph*)}]
\item Quel type de fichier demande l'utilisateur ?
\item Quel navigateur utilise-t-il ?
\item Quel jour est-il ?
\item Que va afficher le navigateur ? 
\end{enumerate}
\end{document}