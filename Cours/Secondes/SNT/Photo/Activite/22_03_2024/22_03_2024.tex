\documentclass{article}
\usepackage{main}

\title{Algorithmes de traitement d'image}
\author{Seconde 9}
\date{22 Mars 2024}

\begin{document}
\maketitle

\section{De la prise de vue au rendu final}

Document : \url{https://interstices.info/tout-ce-que-les-algorithmes-de-traitement-dimages-font-pour-nous/}

Répondre aux questions suivantes :
\begin{enumerate}
\item Qu'est-ce qu'un photosite ?
\item Comment sont réparties les couleurs par photosite ? Quel agencements de filtre colorés est le plus populaire ?
\item Rappeler le nom d'extension de fichier comportant les données \og brutes \fg renvoyées par les photosites ?
\item Nommer et décrire en une phrase les différentes opérations effectués sur l'image entre la prise de vue et le rendu final.
\end{enumerate}
\section{IPOL}
La plateforme IPOL permet de tester différents algorithmes de traitement d'image. Choisir un des liens suivants et essayer l'algorithme proposé sur une des entrées (input) possibles. Cliquer sur \og run \fg en bas de la page pour exécuter l'algorithme.

Décrire le ou les sorties de l'algorithme en quelques phrases. D'après vous, comment agir sur les pixels d'une image pour obtenir l'effet en sortie.
\begin{itemize}
\item Contour : \url{https://ipolcore.ipol.im/demo/clientApp/demo.html?id=38}
\item Couleurs : \url{https://ipolcore.ipol.im/demo/clientApp/demo.html?id=51}
\item Bruit : \url{https://ipolcore.ipol.im/demo/clientApp/demo.html?id=55}
\item Eclairage : \url{https://ipolcore.ipol.im/demo/clientApp/demo.html?id=429}
\end{itemize}

\end{document}