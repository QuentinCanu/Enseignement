\documentclass{exam}
\usepackage[exos]{main}

\title{Exercices : Équations Produit-Nul}
\date{29 Avril 2024}
\author{Seconde 9}

\begin{document}
\maketitle

\begin{tcolorbox}
\begin{proposition}
Soient $A(x)$ et $B(x)$ deux expressions dépendant d'une inconnue $x$. Alors, l'équation
\begin{equation*}
A(x) \times B(x) = 0    
\end{equation*}
admet pour solutions les valeurs de $x$ telles que $A(x) = 0$ et les valeurs de $x$ telles que $B(x) = 0$.
\end{proposition}
\end{tcolorbox}
\begin{example}
L'équation
\begin{equation*}
(x+1)(x+2)=0    
\end{equation*}
admet pour solutions $-1$ (car $-1$ est solution de $x+1=0$) et $-2$ (car $-2$ est solution de $x+2=0$). L'ensemble des solutions de l'équation produit-nul est donc $\mathcal{S}=\{-1;-2\}$.
\end{example}
\begin{questions}
\question Résoudre dans $\R$ les équations produit-nul suivantes :
\begin{parts}
\part $(-6x+4)(-2x+5)=0$
\part $(-2x-2)(-6x+9)=0$
\part $(-x+7)(4x+1)=0$
\part $(6x-7)(5x+5)=0$
\part $(-6x+5)(-x+3)=0$
\part $(-9x-4)(-2x+4)=0$    
\end{parts}
\question 
Résoudre dans $\R$ les équations suivantes :
\begin{parts}
\part $(6x+3)(x+8)+(6x+3)(7x+5)=0$
\part $(-6x-6)(2x-5)-(-6x-6)(-9x+6)=0$
\part $(5x-8)^2+(5x-8)(-9x-2)=0$
\part $(8x-9)(-9x-8)-(8x-9)^2=0$
\end{parts}
\makeemptybox{10cm}
\end{questions}
\newpage
\begin{tcolorbox}
\begin{proposition}
Soient $A(x)$ et $B(x)$ deux expressions dépendant d'une inconnue $x$. Pour résoudre l'équation $\dfrac{A(x)}{B(x)}=0$, il faut procéder à deux étapes :
\begin{enumerate}
\item Identifier les valeurs pour lesquelles $B(x)= 0$. L'équation n'a pas de sens si $B(x) = 0$.
\item Résoudre l'équation $A(x)=0$, en excluant les solutions identifiées précedemment.
\end{enumerate}
\end{proposition}    
\end{tcolorbox}
\begin{example}
L'équation
\begin{equation*}
\dfrac{x+2}{x+1} = 0
\end{equation*}
est définie sur $\R$ privé de $-1$ (car si $x = -1$, alors le dénominateur est nul). On résout sur cet ensemble l'équation $x+2=0$, ce qui donne $-2$ comme unique solution. L'ensemble des solutions de cette équation quotient-nul est $\mathcal{S}=\{-2\}$.
\end{example}
\begin{questions}
\question Préciser au préalable quelles sont les valeurs interdite pour les équations ci-après, puis les résoudre :
\begin{parts}
\part $\dfrac{9x-4}{-2x-6}=0$
\part $\dfrac{100-x^2}{9x-27}=0$
\part $\dfrac{-3x-4}{5x+3}=-9$
\part $\dfrac{1}{5x-2}=\dfrac{6}{4x+9}$
\end{parts}
\makeemptybox{10cm}
\end{questions}
\end{document}