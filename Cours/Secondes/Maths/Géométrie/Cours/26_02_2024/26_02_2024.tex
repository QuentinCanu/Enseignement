\documentclass{article}
\usepackage{main}

\title{Cours : Distance entre deux points}
\author{Quentin Canu}
\date{26 Février 2024}

\begin{document}
\maketitle
\section{Interrogation de cours}
Placer suivant le plan de classe.
\section{Activité}
\begin{enumerate}
\item Première partie en $5$ minutes, conversation sur l'importance du théorème de Pythagore, rappel sur les repères orthonormés.
\item Deuxième partie en autonomie en $25$ minutes. 
\end{enumerate}
\section{Cours}
\subsection*{Distance entre deux points}
\begin{proposition}
Soit deux points $A(x_A,y_A)$ et $B(x_B,y_B)$. Alors la longueur du segment $[AB]$ est donnée par
\begin{equation*}
AB = \sqrt{(x_B - x_A)^2 + (y_B - y_A)^2}
\end{equation*}
\end{proposition}
\begin{example}
Pour chacune des questions suivantes, calculer la longueur du segment $[AB]$ :
\begin{itemize}
\item $A(1;2)$ et$B(4;6)$. Dans ce cas,
\begin{equation*}
AB = \sqrt{(4 - 1)^2 + (6 - 2)^2} = \sqrt{3^2 + 4^2} = \sqrt{25} = 5
\end{equation*}
\item $A(0;2)$ et $B(5;6)$ ($AB = \sqrt{41}$)
\item $A(-2;8)$ et $B(3;-3)$ ($AB = \sqrt{146}$)
\end{itemize}
\end{example}
\section{Exercice}
Exercices 39 et 41 page 173
\end{document}