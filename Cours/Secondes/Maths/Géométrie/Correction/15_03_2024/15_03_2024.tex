\documentclass{article}
\usepackage{main}

\title{Correction du contrôle}
\author{Seconde 9}
\date{15 Mars 2024}

\begin{document}
\maketitle
\thispagestyle{empty}
\begin{enumerate}[label=\textbf{Exercice \arabic* :}]
\item Simplifier les fractions de sorte à avoir un entier au dénominateur. 
\begin{enumerate}
    \item $\dfrac{1}{\sqrt{3}}$
    \item $\dfrac{5 + \sqrt{2}}{5 - \sqrt{2}}$
\end{enumerate}
\item Dire si oui ou non le quadrilatère $ABCD$ est un parallélogramme, avec $A(-2;2)$; $B(2;-4)$; $C(8;-4)$ et $D(4;2)$.
\item Proportions de proportions : On a $42\%$ d'oiseau dans un zoo, et parmi ces oiseaux, $5\%$ ont des plumes rouges. Combien y a-t-il d'oiseau à plumes rouges ?
\item Taux d'évolution : Donner la formule du taux d'évolution entre une valeur de départ $V_d$ et une valeur d'arrivée $V_a$. Exemple, taux d'évolution du salaire moyen des employés passant de $1872$ à $1880$ ?
\item Taux d'évolution successifs : On augmente un prix de $2\%$, puis on augmente le résultat de $3\%$. De combien a-t-on augmenté ce prix au final ?
\end{enumerate}

\end{document}