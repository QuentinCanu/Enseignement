\documentclass{exam}
\usepackage{main}

\title{Exercices : Inéquations}
\date{18 Mars 2024}
\author{Seconde 9}

\qformat{\textbf{Exerice \thequestion :}\hfill}

\begin{document}
\maketitle
\thispagestyle{empty}
\begin{questions}
\question
\begin{parts}
\part $3$ est-il solution de l'inéquation $-3x^2+2x-8 < -30$ ?
\part  $2$ est-il solution de l'inéquation $-2x^2+4x-9 \leqslant -3x+1$ ?
\part  $-4$ est-il solution de l'inéquation $3x^2+2x+9 \geqslant 51$ ?
\end{parts}
\vspace*{1cm}
\question Résoudre les inéquations suivantes.
\begin{parts}
\part $9x+2>0$
\part $x+1\geqslant-4$
\part $-9x-13>3$
\part $-6x+12\leqslant -2x-4$
\part $-11x<-10$
\end{parts}
\vspace*{1cm}
\question
\begin{parts}
\part Une société de location de véhicules particulièrs propose deux tarifs :\\
$\bullet$ Tarif A : un forfait de $27$ \euro et $0{,}28$ \euro par km parcouru ;\\
$\bullet$  Tarif B : un forfait de $41$ \euro et $0{,}16$ \euro par km parcouru ;\\
À partir de combien de km (arrondi à l'unité), le tarif B est-il plus intéressant que le tarif A ?\\
\part On donne les deux programmes de calcul suivants :\\
\textbf{Programme 1 :}
\begin{itemize}
\item Choisir un nombre
\item Ajouter $-3$
\item Multiplier le résultat par le nombre choisi au départ
\end{itemize}
\textbf{Programme 2 :}
                \begin{itemize}
\item Choisir un nombre
\item Ajouter $1$
\item Prendre le carré du résultat
\end{itemize}
Déterminer les nombres que l'on  doit entrer dans ces deux programmes pour qu'au final le résultat obtenu
avec le programme 1 soit supérieur ou égal  à celui obtenu avec le programme 2.
\end{parts}
\end{questions}
\end{document}