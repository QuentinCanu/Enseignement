\documentclass{article}
\usepackage{main}
\title{Cours : Coefficient multiplicateur}
\author{Quentin Canu}
\date{19 Janvier 2024}
\begin{document}
\maketitle
\section{Question Flash}
Calculer les variations absolues et variations relatives entre les valeurs de départ $V_d$ et d'arrivées $V_a$ suivantes :
\begin{enumerate}[label=\alph*)]
\item $V_d = 120$ et $V_a = 132$;
\item $V_d = 16$ et $V_a = 32$;
\item $V_d = 46$ et $V_a = 23$;
\item $V_d = 66$ et $V_a = 88$.
\end{enumerate}
\section{Correction de l'exercice 56 page 321}
\begin{itemize}
\item Vérifier les cahiers.
\item Projeter l'exercice.
\end{itemize}
\section{Cours}
\begin{remark}
Un taux d'évolution peut être négatif, dans ce cas, cela signifie qu'il y a eu diminution de $V_d$ vers $V_a$.
\end{remark}
\subsection*{Coefficient multiplicateur}
\begin{proposition}
Pour augmenter une valeur de départ $V_d$ par un certain taux $t$, on fait la multiplication
    \begin{equation*}
        V_d (1 + t)\,.    
\end{equation*}
Pour diminuer une valeur de départ $V_d$ par un certain taux $t$, on fait la multiplication
\begin{equation*}
    V_d (1 - t)\,.
\end{equation*}
Les valeurs $(1 + t)$ ou $(1 - t)$ sont appelées coefficients multiplicateurs.
\end{proposition}
\begin{remark}
Si le taux d'évolution donné est en pourcentages ($t \%$), alors l'augmentation (resp. la diminution) est donnée par $V_d (1 + \dfrac{t}{100})$ (resp. $V_d (1 - \dfrac{t}{100})$).
\end{remark}
\begin{example}
\begin{enumerate}
\item Un pull d'une valeur de $50$ \euro{} est soldé de $30 \%$. Donner le nouveau prix du pull.
\item Il y a $13$ personnages jouables dans un jeu vidéo. On augmente ce nombre de $46\%$ pour la suite. Combien y aura-t-il de personnages jouables dans cette suite ?
\end{enumerate}
\end{example}
\section{Activité sur les CA}
\end{document}