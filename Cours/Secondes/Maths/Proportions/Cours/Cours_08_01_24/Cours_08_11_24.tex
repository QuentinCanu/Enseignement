\documentclass{article}
\usepackage{main}
\title{Cours : Racines carrées}
\date{08 Janvier 2024}
\author{Quentin Canu}

\begin{document}
\maketitle

\section{Questions Flash}
$\sqrt{4}; \sqrt{9}; \sqrt{16}$ ?

Quels sont les carrés parfaits suivants ? $25; 36; 49; 64; 89; 100; 121; 144$.
\section{Cours : Definitions}
\begin{definition}
Soit $x$ un nombre réel positif. On définit la racine carrée de $x$, notée $\sqrt{x}$, le seul nombre positif tel qu'on obtient $x$ en le mettant au carré.    
\end{definition}
\begin{example}
$\sqrt{4} = 2$ car $2^2 = 4$
\end{example}
\begin{remark}
Il faut donc toujours s'assurer que ce qui est à l'intérieur de la racine est un nombre positif.
\end{remark}
\begin{example}
Les expressions suivantes sont-elles bien définies ?
\begin{enumerate}
\item $\sqrt{10}$
\item $\sqrt{-2}$
\item $\sqrt{-3^2}$
\item $\sqrt{(-3)^2}$
\item $\sqrt{8 - \pi}$
\item $\sqrt{\sqrt{3}}$ 
\end{enumerate}
\end{example}
\begin{proposition}
Soit $a$ et $b$ deux nombres réels positifs. Alors
\begin{align*}
\sqrt{ab} &= \sqrt{a}\sqrt{b}\\
\sqrt{\dfrac{a}{b}} &= \dfrac{\sqrt{a}}{\sqrt{b}}\\
\sqrt{a}^2 &= \sqrt{a}\sqrt{a} = a
\end{align*}
\end{proposition}
\begin{example}
Donner une forme simplifiée aux expressions suivantes, si possible :
\begin{enumerate}
\item $\sqrt{\dfrac{500}{5}}$
\item $\sqrt{5} \times \sqrt{55}$
\item $\sqrt{11} + \sqrt{13}$
\item $-2\sqrt{11} \times 5\sqrt{11}$
\item $(3\sqrt{13})^2$
\end{enumerate}
\end{example}
\section{Cours : Savoir-Faire}
\begin{example}
Face à une racine, il faut chercher à la mettre sous la forme $a\sqrt{b}$ pour la simplifier.
\begin{enumerate}
\item $\sqrt{18} = \sqrt{9 \times 2} = \sqrt{9}\sqrt{2} = 3\sqrt{2}$
\item $\sqrt{216} = \sqrt{4 \times 54} = \sqrt{4}\sqrt{54} = 2\sqrt{54} = 2\sqrt{9 \times 6} = 6\sqrt{6}$
\item $\sqrt{363} = \sqrt{121 \times 3} = 11\sqrt{3}$
\end{enumerate}    
\end{example}
\begin{example}
Il faut être capable d'encadrer rapidement une racines par des entiers consécutifs, à l'aide des carrés parfaits.
\begin{enumerate}
\item $\sqrt{17}$ est à encadrer entre $4 = \sqrt{16}$ et $5$.
\item $\sqrt{82}$
\item $\sqrt{103}$ 
\end{enumerate}
\end{example}
\end{document}