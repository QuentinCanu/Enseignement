\documentclass{exam}
\usepackage{main}

\title{Interrogation de cours}
\date{29 Janvier 2023}
\author{Quentin Canu}

\qformatExos{}
\begin{document}
\section*{Version 1}
\begin{questions}
\question Donnez une fraction équivalente sans racine carrée au dénominateur :
\begin{equation*}
\dfrac{17}{2 - 3\sqrt{5}}
\end{equation*}
\question On augmente une valeur $V_d = 144$ de $25\%$ puis on diminue le résultat par $60\%$. Quelle valeur obtient-on ?
\question À l'aide d'une pompe à eau, la pression dans un tuyau passe de $1$ bar (la pression atmosphérique) à $2,4$ bars. Quel est le pourcentage d'augmentation de la pression ?
\end{questions}
\newpage
\section*{Version 2}
\begin{questions}
\question Donnez une fraction équivalente sans racine carrée au dénominateur :
\begin{equation*}
\dfrac{15}{3 - 2\sqrt{7}}
\end{equation*}
\question On diminue une valeur $V_d = 144$ de $25\%$ puis on augmente le résultat par $60\%$. Quelle valeur obtient-on ?
\question À l'aide d'une pompe à eau, la pression dans un tuyau passe de $2,4$ bars à $1$ bar (la pression atmosphérique). Quel est le pourcentage de diminution de la pression ?
\end{questions}
\end{document}