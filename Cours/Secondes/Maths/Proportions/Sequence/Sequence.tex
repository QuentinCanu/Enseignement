\documentclass{article}
\usepackage{main}

\title{Proportions et Evolution}
\date{}
\author{Quentin Canu}

\begin{document}
\maketitle
\section*{Contenu}
\begin{itemize}
\item Population, Sous-Population 
\item Proportions, proportions de proportions
\item \'Evolution : Variation globale, variation relative
\item Evolutions successives : relation sur les coefficients multiplicateurs. 
\end{itemize}
\section*{Capacités attendues}
\begin{itemize}
\item Exploiter la relation entre effectifs, proportions et pourcentages.
\item Traiter des situations simples mettant en jeu des pourcentages de pourcentages.
\item Exploiter la relation entre deux valeurs successives et leur taux d'évolution.
\item Calculer un taux d'évolution global.
\item Calculer un taux d'évolution réciproque. 
\end{itemize}
\section*{Structure}
\begin{enumerate}
\item Proportions
\begin{enumerate}
\item Population et sous-population\label{sect-populations}
\item Effectifs, proportions\label{sect-effectifs}
\item Proportions de proportions\label{sect-proportions-de-proportions}
\end{enumerate}
\item Taux d'evolution
\begin{enumerate}
\item Variation absolue, variation relative \label{sect-variations}
\item Coefficient multiplicateur \label{sect-CM}
\end{enumerate}
\item Evolutions successives et réciproques
\begin{enumerate}
\item Evolutions successives \label{sect-successives}
\item Evolutions réciproques
\end{enumerate}
\end{enumerate}
\section*{Séances}
\begin{enumerate}
\item 20 Décembre 2023 : Parties \ref{sect-populations} et \ref{sect-effectifs}.
\item 8 Janvier 2024 : Racines carrées.
\item 12 Janvier 2024 : Parties \ref{sect-effectifs}, \ref{sect-proportions-de-proportions}.
\item 15 Janvier 2024 : 
    \begin{itemize}
    \item Matin : Racines carrées, développements et factorisations.
    \item Après-Midi : Parties \ref{sect-variations}  
    \end{itemize}
\item 19 Janvier 2024 : Partie \ref{sect-CM}
\item 22 Janvier 2024 :
    \begin{itemize}
    \item Matin : racines carrées, simplification de fractions.
    \item Après-midi : Partie \ref{sect-successives}  
    \end{itemize}
\end{enumerate}
\end{document}
