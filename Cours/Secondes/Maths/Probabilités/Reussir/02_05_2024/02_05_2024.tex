\documentclass{exam}
\usepackage[exos]{main}

\author{Seconde 9}
\title{Exercices de dénombrement}
\date{2 Mai 2024}

\begin{document}
\maketitle

\begin{questions}
\question Combien de menus différent peut-on composer si on dispose de $3$ entrées, $2$ plats et $4$ desserts ?
\vspace*{0.5cm}
\question Deux équipes de hockey de $12$ et $15$ joueurs se serrent la main avant le début du match. Combien de poignées de main sont échangées en tout ?
\vspace*{0.5cm}
\question Un QCM, autorisant une seule bonne réponse par question, comprend $15$ questions avec $4$ choix possibles. De combien de façon peut-on répondre au QCM ?
\vspace*{0.5cm}
\question Un autre QCM, autorisant \emph{une ou plusieurs} bonnes réponses par question, comprend lui aussi $15$ questions et $4$ choix possibles. De combien de façons peut-on répondre à ce QCM-ci ?
\vspace*{0.5cm}
\question Combien de digicode à $5$ chiffres est-il possible de créer pour sécuriser l'entrée d'un bâtiment ? Les symboles utilisés sont les chiffres de $0$ à $9$ ainsi que les lettres $A$ et $B$.
\vspace*{0.5cm}
\question À l'occasion d'une rencontre sportive entre $18$ compétiteurs, on attribue une médaille d'or, d'argent et de bronze. Combien y a-t-il de distributions possibles ?
\vspace*{0.5cm}
\question Combien y a-t-il d'anagrammes au mot \textsc{math} ?
\vspace*{0.5cm}
\question Combien y a-t-il d'anagrammes au mot \textsc{tableau} ?
\vspace*{0.5cm}
\question Un groupe de trois élèves de la seconde 9 doit aller chercher des livres au CDI. Combien de groupes est-il possible de composer ?
\end{questions}
\end{document}