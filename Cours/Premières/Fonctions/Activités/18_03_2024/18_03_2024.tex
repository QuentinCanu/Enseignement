\documentclass{article}
\usepackage{main}

\author{Maths Spécifiques}
\title{Fonction Exponentielle de base $a>0$}
\date{18 Mars 2024}

\begin{document}
\maketitle

\paragraph{Rappels :}
\begin{enumerate}
\item Puissance de $a$ par un entier positif.
\vspace*{0.5cm}
\item Puissance de $a$ par un entier négatif. 
\vspace*{0.5cm}
\end{enumerate}
\section{Le cas d'une puissance fractionnaire}
Soit $a>0$, on souhaite définir une notion de puissance fractionnaire, c'est-à-dire
\begin{equation*}
a^{p/q}\,,    
\end{equation*}
où $p$ est un entier relatif, et $q$ un entier naturel non nul. On souhaite que cette notion de puissance respecte les propriétés déjà vues pour les puissances, comme $a^k \times a^{k'} = a^{k + k'}$.
\begin{enumerate}
\item Justifier que notre définition de $a^{p/q}$ devrait être égal à $(a^p)^{1/q}$.
\end{enumerate}
On peut donc concentrer nos recherches à donner une définition à $a^{1/q}$. Testons pour $q = 2$.
\begin{enumerate}[resume*]
\item Que signifie la notion de racine carrée, \emph{en français} ?
\item Justifier que $a^{1/2}$ correspond à la racine carrée de $a$, c'est-à-dire $\sqrt{a}$.
\item Que pourrait bien signifier la notion de racine $q$\ieme{} de $a$, notée $\sqrt[q]{a}$, \emph{en français} ?
\item Vérifier que cette notion correspond bien à la définition que l'on souhaite donner à $a^{1/q}$.
\end{enumerate}
\vspace*{7cm}
On donne donc la définition suivante pour la puissance $\dfrac{1}{q}$ d'un nombre : 
\begin{definition}
Le nombre $a^{1/q}$ est le nombre tel que si on le multipliait $q$ fois par lui-même, on obtiendrait $a$. Il correspond donc à la racine $q$\ieme{} de $a$, notée aussi $\sqrt[q]{a}$.
\begin{equation*}
a^{1/q} = \sqrt[q]{a}\,.
\end{equation*}
\end{definition}
\section{Le cas d'une puissance réelle}
On souhaite aller encore plus loin et définir la puissance de $a>0$ par tout nombre \emph{réel}. Par exemple, quelle définition donner à $2^{\sqrt{2}}$ ?
\begin{enumerate}[resume*]
\item Montrer que $(\sqrt{2} + 1)(\sqrt{2} - 1) = 1$. En déduire que $\sqrt{2} = 1 + \dfrac{1}{\sqrt{2} + 1}$.
\item Une première approximation grossière de $\sqrt{2}$ est $\sqrt{2}=\dfrac{3}{2}$. En utilisant cette égalité dans $1 + \dfrac{1}{\sqrt{2}+1}$, en déduire une autre fraction permettant d'approcher $\sqrt{2}$.
\end{enumerate}
En continuant, on obtient ainsi une suite de fractions approchant de plus en plus $\sqrt{2}$ : $\dfrac{3}{2}; \dfrac{7}{5}; \dfrac{17}{12}\dots$
\begin{enumerate}[resume*]
\item Utiliser ces fractions pour calculer une valeur approchée de $2^{\sqrt{2}}$ sur votre calculatrice.
\end{enumerate}
On pourrait continuer longtemps. Plus les fractions sont précises, moins le résultat bougera. Ce comportement nous permet de déduire une définition pour $2^{\sqrt{2}}$ : c'est la valeur que l'on obtient en continuant ce processus à l'infini.
\section{Conclusion}
On a travaillé sur un moyen de parler de la puissance \emph{réelle} d'un nombre $a>0$. Maintenant que nous sommes convaincu que c'est possible, il nous est possible de définir une fonction qui à tout nombre réel $x$ renvoie $a^x$. C'est ce qu'on appelle la \emph{fonction exponentielle de base $a$}.
\begin{definition}
Soit $a>0$. La fonction exponentielle de base $a$ est la fonction
\begin{equation*}
\function{f}{\R}{\R}{x}{a^x}\,.    
\end{equation*}    
\end{definition} 
\end{document}
