\documentclass{exam}
\usepackage{main}

\title{Le paradoxe des trois pièces de monnaie}
\date{11 Décembre 2023}

\qformat{\textbf{Exercice}\hfill}
\begin{document}
\begin{questions}
\question Le paradoxe des trois pièces de monnaie est donné par l'énoncé suivant :
\begin{quote}
\og On se donne trois pièces de monnaie avec lesquelles on joue à pile ou face. Pour gagner, il faut que les trois pièces donnent toutes Pile, ou toutes Face. Quelle est la probabilité de gagner ?

Puisqu'au moins deux pièces sur trois vont forcément donner le même côté, cela signifie qu'il y a une chance sur deux pour que la troisième pièce donne le même côté. La réponse est donc $\dfrac{1}{2}$\fg
\end{quote} 
Nous allons voir que ce raisonnement est faux.
\begin{parts}
\part Voici une configuration possible : $PFP$. Cela signifie que la première pièce a donné Pile, la deuxième Face et la troisième Pile. Donner la liste de toutes les configurations possibles.
\part Quelle sont les configurations qui correspondent à une victoire ?
\part En déduire la probabilité de victoire.
\part Ce résultat est différent du raisonnement donné plus haut. En vous inspirant du paradoxe des deux enfants, expliquer ce qui ne convient pas dans le raisonnement.
\end{parts}
\end{questions}
\vspace{0.75cm}
\begin{questions}
    \question Le paradoxe des trois pièces de monnaie est donné par l'énoncé suivant :
    \begin{quote}
    \og On se donne trois pièces de monnaie avec lesquelles on joue à pile ou face. Pour gagner, il faut que les trois pièces donnent toutes Pile, ou toutes Face. Quelle est la probabilité de gagner ?
    
    Puisqu'au moins deux pièces sur trois vont forcément donner le même côté, cela signifie qu'il y a une chance sur deux pour que la troisième pièce donne le même côté. La réponse est donc $\dfrac{1}{2}$\fg
    \end{quote} 
    Nous allons voir que ce raisonnement est faux.
    \begin{parts}
    \part Voici une configuration possible : $PFP$. Cela signifie que la première pièce a donné Pile, la deuxième Face et la troisième Pile. Donner la liste de toutes les configurations possibles.
    \part Quelle sont les configurations qui correspondent à une victoire ?
    \part En déduire la probabilité de victoire.
    \part Ce résultat est différent du raisonnement donné plus haut. En vous inspirant du paradoxe des deux enfants, expliquer ce qui ne convient pas dans le raisonnement.
    \end{parts}
    \end{questions}
\end{document}