\documentclass{exam}
\usepackage{main}
\title{Evaluation de cours}
\date{22 Janvier 2021}
\author{Quentin Canu}

\qformatExos{}

\begin{document}
\begin{questions}
\question Soit $A$ et $B$ deux événements d'une expérience aléatoire, avec $P(B) \neq 0$. Donner l'expression de la probabilité de $A$ sachant $B$, notée $P_B(A)$.
\vspace*{0.5cm}
\question On tire une pièce équilibrée à pile ou face. Si l'on fait face, on tire une boule dans une urne avec $3$ boules rouges et $2$ boules noires. Si l'on fait pile, on tire une boule dans une urne avec $2$ boules rouges et $3$ boules noires.
\begin{parts}
\part Faire un arbre pondéré résumant l'expérience.
\part Quelle est la probabilité d'obtenir une boule rouge au terme de cette expérience ?
\end{parts}
\end{questions}
\vspace*{1cm}
\begin{questions}
\question Soit $A$ et $B$ deux événements d'une expérience aléatoire, avec $P(B) \neq 0$. Donner l'expression de la probabilité de $A$ sachant $B$, notée $P_B(A)$.
\vspace*{0.5cm}
\question On tire une pièce équilibrée à pile ou face. Si l'on fait face, on tire une boule dans une urne avec $3$ boules rouges et $2$ boules noires. Si l'on fait pile, on tire une boule dans une urne avec $2$ boules rouges et $3$ boules noires.
\begin{parts}
\part Faire un arbre pondéré résumant l'expérience.
\part Quelle est la probabilité d'obtenir une boule rouge au terme de cette expérience ?
\end{parts}
\end{questions}
\vspace*{1cm}
\begin{questions}
\question Soit $A$ et $B$ deux événements d'une expérience aléatoire, avec $P(B) \neq 0$. Donner l'expression de la probabilité de $A$ sachant $B$, notée $P_B(A)$.
\vspace*{0.5cm}
\question On tire une pièce équilibrée à pile ou face. Si l'on fait face, on tire une boule dans une urne avec $3$ boules rouges et $2$ boules noires. Si l'on fait pile, on tire une boule dans une urne avec $2$ boules rouges et $3$ boules noires.
\begin{parts}
\part Faire un arbre pondéré résumant l'expérience.
\part Quelle est la probabilité d'obtenir une boule rouge au terme de cette expérience ?
\end{parts}
\end{questions}
\vspace*{1cm}
\begin{questions}
\question Soit $A$ et $B$ deux événements d'une expérience aléatoire, avec $P(B) \neq 0$. Donner l'expression de la probabilité de $A$ sachant $B$, notée $P_B(A)$.
\vspace*{0.5cm}
\question On tire une pièce équilibrée à pile ou face. Si l'on fait face, on tire une boule dans une urne avec $3$ boules rouges et $2$ boules noires. Si l'on fait pile, on tire une boule dans une urne avec $2$ boules rouges et $3$ boules noires.
\begin{parts}
\part Faire un arbre pondéré résumant l'expérience.
\part Quelle est la probabilité d'obtenir une boule rouge au terme de cette expérience ?
\end{parts}
\end{questions}
\end{document}