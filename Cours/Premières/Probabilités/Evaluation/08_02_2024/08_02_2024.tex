\documentclass{exam}
\usepackage{main}
\qformatExos{}

\title{\'Evaluation}
\date{8 Février 2024}
\author{}

\begin{document}
\maketitle
\subsection*{Version 1}
\begin{questions}
\question Soit $A$ et $B$ deux événements, tels que $P(B) \neq 0$. Donner l'expression de la probabilité de $A$ sachant $B$.

\question Un joueur de tennis a une probabilité de $0,7$ de réussir son premier service. Sinon, il a une probabilité de $0,6$ de réussir le deuxième. On nomme $R_1$ la réussite du premier service et $R_2$ la réussite du second.
\begin{parts}
\part Représenter la situation par un arbre pondéré.
\part Quelle est la probabilité que le joueur réussisse son deuxième service uniquement ? 
\part Quelle est la probabilité que le joueur ne réussisse aucun des deux services ?
\end{parts}
\end{questions}
\newpage
\maketitle
\subsection*{Version 2}
\begin{questions}
\question Soient $A$ et $B$ deux événéments, tels que $P(B) \neq 0$. Que doivent vérifier $A$ et $B$ pour qu'ils soient considérés indépendants ?
\question Un modèle de voiture a une probabilité de $0,2$ d'avoir une avarie de moteur, et dans ce cas, le risque de tomber en panne est d'une probabilité de $0,6$.
\begin{parts}
\part Représenter la situation par un arbre pondéré.
\part Quelle est la probabilité que la voiture tombe en panne ?
\part Quelle est la probabilité que la voiture ne tombe pas panne mais qu'elle aie une avarie de moteur ?
\end{parts}
\end{questions}
\end{document}