\documentclass{article}
\usepackage{main}

\title{Cours : Début des probabilités}
\date{11 Décembre 2023}
\begin{document}
\maketitle
\section{Fin sur les suites (10 minutes)}
\begin{proposition}
Soit $(u_n)$ une suite qui ne s'annule jamais. Alors, $(u_n)$ est croissante si et seulement si
\begin{equation*}
\dfrac{u_{n+1}}{u_n} \geq 1    
\end{equation*}
pour tout $n$ entier naturel.
De même, $(u_n)$ est décroissante si et seulement si
\begin{equation*}
\dfrac{u_{n+1}}{u_n} \leq 1    
\end{equation*}
\end{proposition}
\begin{proposition}
Soit $(u_n)$ une suite géométrique de raison $q > 0$. Alors,
\begin{itemize}
\item $(u_n)$ est croissante si et seulement si $q \geq 1$;
\item et $(u_n)$ est décroissante si et seulement si $q \leq 1$. 
\end{itemize}
\end{proposition}
Donner l'idée de la preuve.
\section{Probabilités : Activité}
\begin{itemize}
\item Paradoxe des anniversaires
\item Paradoxe des deux enfants
\item Paradoxe des trois pièces de monnaie (à finir à la maison)
\end{itemize}

\end{document}