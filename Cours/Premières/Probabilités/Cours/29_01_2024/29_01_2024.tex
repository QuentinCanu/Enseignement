\documentclass{article}
\usepackage{main}

\title{Cours : succession d'événements}
\date{29 Janvier 2024}
\author{Quentin Canu}

\begin{document}
\maketitle

\section{Début}

Lancer \url{https://astucesweb.fr/projets/loto/#!} avec le préréglage \og Vitesse Moyenne \fg.

\section{Question Flash}

Formules ?

\section{Correction de l'exercice 3}

\section{Loto}

\begin{definition}
Le loto est le tirage de $6$ nombres entre $1$ et $49$ sans remise.    
\end{definition}

\section{Premières pensées}
Trier
\begin{enumerate}
\item Probabilité que $14;19;21;28;31$ gagne au loto;
\item Probabilité que $1;2;3;4;5;6$ gagne au loto;
\item Probabilité que le $14$ sorte au tirage du loto;
\item Probabilité que votre date d'anniversaire (numéro de jour + numéro du mois) sorte au tirage du loto;
\item Probabilité que $14;19;21;28;31$ gagne au loto sachant que cette séquence n'est pas sortie depuis $3$ ans.
\end{enumerate}
\section{Un tirage}
\begin{enumerate}
\item Comment calculer la probabilité qu'un tirage donné tombe au loto ?
\item Sachant que notre premier numéro est tombé au loto, quelle est la probabilité que le reste tombe aussi ?
\end{enumerate}
\section{Succession de tirages}
Sachant que l'on a gagné la veille, quelle est la probabilité de gagner au loto ?



\end{document}