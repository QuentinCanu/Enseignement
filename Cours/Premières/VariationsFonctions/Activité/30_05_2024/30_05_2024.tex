\documentclass{exam}
\usepackage{mainExam}

\title{Énigmes}
\date{30 Mai 2024}
\author{Maths Spécifiques}

\begin{document}
\maketitle
\begin{questions}
\titledquestion{Le Code d'Honneur des Pirates}
Cinq pirates, Alexandre, Bilel, Charlie, Daniela et Etienne, se partagent leur butin de $100$ pièces d'or. Il appliquent pour cela le Code d'Honneur des Pirates sur le partage :
\begin{enumerate}
\item Le capitaine, Alexandre, propose un partage des gains. Celui-ci peut être équitable (\og Chaque pirate reçoit $20$ pièces d'or\fg) ou arbitraire (\og Alexandre reçoit $87$ pièces, Bilel en reçoit $12$, Charlie $1$ et les deux autres $0$. \fg).
\item Tous les pirates votent \og Pour \fg ou \og Contre \fg. Le vote est obligatoire : Le vote blanc et l'abstention ne sont pas possibles.
\item  Si le \og Pour \fg l'emporte, \textsc{ou s'il y a égalité}, le partage proposé est adopté.
\item Si le \og Contre \fg l'emporte, le partage n'est pas adopté, et le capitaine, Alexandre, subit le supplice de la planche. Dans ce cas, le nouveau capitaine est Bilel, et il repropose un nouveau partage du butin, en suivant le Code d'Honneur des Pirates.
\end{enumerate}
L'ordre de succession du rôle de Capitaine est donnée par Alexandre, Bilel, Charlie, Daniela et Etienne.

Quel partage peut proposer Alexandre afin qu'il puisse maximiser son gain ?

\titledquestion{D'or et de platine}

Devant vous se tiennent cent pièces particulières : une face est en or, l'autre en platine. Votre but est de séparer les cent pièces en deux tas. Il doit y avoir autant de pièces tournées du côté platine dans les deux tas. 

De plus, au moment du jeu, les lumières vont s'éteindre : il faudra constituer les deux tas dans le noir ! Il n'y a aucun moyen de distinguer les faces au toucher. Par chance, vous vous souvenez juste avant l'extinction des feux qu'il y avait exactement $20$ pièces tournées du côté platine.

Comment procédez-vous ?

\end{questions}
\end{document}