\documentclass{article}
\usepackage{main}

\title{Cours : Rappels sur les fonctions}
\author{Quentin Canu}
\date{8 Février 2023}

\begin{document}
\maketitle
\section{Interrogation de Cours}
\section{Activité Fonctions VS Suites}
Pour chacune des situations suivantes, dire s'il faut privilégier une suite ou une fonction pour la modéliser.
\begin{enumerate}
\item L'aire d'un carré en fonction de sa longueur.
\item Le salaire touché tous les mois par un employé de bureau.
\item La vitesse d'une voiture durant toute la période de freinage.
\item Le nombre de poignée de mains au début d'une réunion en fonction du nombre de personnes présentes.
\item La proximité de la comète de Halley à la Terre à chaque instant.
\item Le nombre de cellules dans un corps humain durant sa vie.
\end{enumerate}
\section{Rappels sur les fonctions affines}
\begin{definition}
Une fonction affine est une fonction particulière qui à tout réel $x$ associe un nombre de la forme
\begin{equation*}
ax + b\,.
\end{equation*}
\end{definition}
\begin{proposition}
La courbe représentative d'une fonction affine est une droite.  
\end{proposition}
Tracer plusieurs exemples.
\begin{definition}
Soit une fonction affine 
\begin{equation*}
\function{f}{\R}{\R}{x}{ax + b}\,.
\end{equation*}
Le nombre $a$ est appelé \emph{coefficient directeur}, et le nombre $b$ est appelé \emph{ordonnée à l'origine}.
\end{definition}
\begin{example}
Soit $f : x \mapsto 2x + 1$. Identifier son coefficient directeur et son ordonnée à l'origine.
Dessiner la courbe représentative de $g : x \mapsto - x + \dfrac{1}{2}$.
\end{example}
\end{document}