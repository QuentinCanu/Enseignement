\documentclass{exam}
\usepackage[exos]{main}

\title{Exercices : fonction exponentielle}
\date{29 Avril 2024}
\author{Maths Spécifiques}

\begin{document}
\maketitle
\begin{definition}
Soit $X$ une quantité initiale, et $n$ un entier naturel. Au bout de $n$ étapes, la quantité $X$ a augmenté de $t\%$ ($t$ peut-être négatif dans le cas d'une diminution). On définit le \emph{taux d'évolution moyen} de cette augmentation par
\begin{equation*}
\left(1+\dfrac{t}{100}\right)^{\dfrac{1}{n}} - 1    
\end{equation*} 
\end{definition}
Les exercices suivants consistent à explorer cette notion.
\begin{questions}
\question
\begin{parts}
\part Par combien doit-on multiplier une quantité pour augmenter cette quantité de $12\%$ ?
\part On augmente une quantité de $47\%$, puis de $23\%$. Par quel coefficient multiplie-t-on la quantité initiale pour obtenir le résultat ?
\part Quel est le taux d'évolution associé ?
\end{parts}
\makeemptybox{5cm}
\question On a augmenté une valeur $X_0$ de $t_1 = 30\%$, puis de $t_2 = 6\%$. On note $X_1$ le résultat.
\begin{parts}
\part De quel pourcentage doit-on augmenter $X_0$ pour directement obtenir $X_1$ ? On note ce taux d'évolution $T$.
\part On note $m$ la moyenne de $t_1$ et de $t_2$. Calculer $m$.
\part On augmente deux fois la même valeur initiale $X_0$ de $m\%$. Obtient-on $X_1$?
\part On note $m'$ la valeur obtenue en appliquant la définition à $T$. Calculer $m'$.
\part On augmente deux fois la même valeur initiale $X$ de $m'\%$. Obtient-t-on $X_1$?
\end{parts}
\makeemptybox{5cm}
\newpage
\question La population d'une ville a doublé en $20$ ans. Donner son taux d'évolution annuel moyen en arrondissant le résultat à $0,1\%$ près.
\makeemptybox{3cm}
\question Les bénéfices d'une entreprise ont chuté de $40\%$ en un an. Déterminer son taux d'évolution trimestriel moyen en arrondissant le résultat à $0,1\%$ près.
\makeemptybox{3cm}
\question Entre 2015 et 2018, le record de hauteur du perchiste Armand Duplentis s'est amélioré chaque année passant de $\qty{5,30}{\meter}$ à $\qty{6,05}{\meter}$. Déterminer le taux d'évolution annuel moyen de son record.
\makeemptybox{3cm}
\end{questions}
\end{document}