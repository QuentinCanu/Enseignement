\documentclass{article}
\usepackage{main}

\author{Quentin Canu}
\title{Cours : Coefficient Directeur}
\date{26 Février 2024}

\begin{document}
\maketitle
\section{Questions Flashs}
\begin{enumerate}
\item Soit $f : x \mapsto 4x + 7$. Donner l'image de $2$ et de $-1$ par $f$. Quel est son coefficient directeur ? Quelle est son ordonnée à l'origine ?
\end{enumerate}
\section{Cours}
\subsection*{Coefficient Directeur}
\begin{proposition}
Soit $f$ une fonction affine, et $x_1,x_2$ deux nombres. On note $y_1 = f(x_1)$ et $y_2 = f(x_2)$ l'image de ces nombres par $f$. Alors, le coefficient directeur $a$ de $f$ se calcule en effectuant
\begin{equation*}
a = \dfrac{y_2 - y_1}{x_2 - x_1} = \dfrac{f(x_2) - f(x_1)}{x_2 - x_1}
\end{equation*}
\end{proposition}
\begin{example}
Soit $f$ une fonction affine, telle que
\begin{equation*}
\begin{aligned}
f(3) &= 2\\
f(1) &= 1\\
\end{aligned}    
\end{equation*}
Alors, le coefficient directeur de $f$ est donné par
\begin{equation*}
a = \dfrac{2 - 1}{3 - 1} = \dfrac{1}{2}\,.
\end{equation*}
\end{example}
\begin{center}
\begin{tikzpicture}[scale=0.75]
\millirepere[step = 1](-3.25,-3.25) -- (3.25,3.25);
\draw (-3.25, -9/8) -- (3.25,17/8);
\end{tikzpicture}
\end{center}
Le coefficient directeur correspond à la pente de la droite représentant la fonction : si l'on avance de $1$ en abscisses, on progresse de $a$ en ordonnée.
\begin{definition}
Une fonction $f$ est croissante (resp. décroissante) si et seulement si pour tout $x \leq y$, on a $f(x) \leq f(y)$ (resp. $f(x) \geq f(y)$). Une fonction à la fois croissante et décroissante est dite constante : dans ce cas, pour tout $x \leq y$, on a $f(x) = f(y)$.
\end{definition}
\begin{proposition}
Soit $f$ une fonction affine. Alors $f$ est croissante (resp. décroissante) si et seulement si son coefficient directeur est positif (resp. négatif). Si son coefficient directeur est nul, alors la fonction est constante.
\end{proposition}
\begin{center}
\begin{tikzpicture}[scale=0.75]
\millirepere[step = 1](-3.25,-3.25) -- (3.25,3.25);
\draw (-3.25, -9/8) -- (3.25,17/8) node[right] {$f$};
\draw (-3.25, 2.5) -- (3.25, -1) node[right] {$g$};
\end{tikzpicture}
\end{center}
Ici, $f$ est croissante et $g$ est décroissante.
\section{Exercices}
\begin{enumerate}[label=\textbf{ Exercice \arabic*}]
\item Soit $f$ une fonction affine. Pour chacune des situations suivantes, déterminer le coefficient directeur de $f$.
\begin{enumerate}[label=\emph{\alph*)}]
\multido{}{4}{
\item $f(\random{-5}{5}) = \random{-5}{5}$ et $f(\random{-5}{5}) = \random{-5}{5}$}
\end{enumerate}
\item Soit $f$ une fonction affine telle que $f(\random{1}{7}) = \random{-2}{2}$ et $f(\random{1}{7}) = \random{-2}{2}$.
\begin{enumerate}[label=\emph{\alph*)}]
\item Calculer le coefficient directeur de $f$.
\item Calculer son ordonnée à l'origine.
\end{enumerate}
\item Un restaurant propose un menu à $20$ \euro{}. Le responsable constate que baisser le prix du menu de $20$ centimes augmente le nombre moyen de clients à midi de $5$.
\begin{enumerate}[label=\emph{\alph*)}]
\item Combien y a-t-il de clients en moyenne quand le prix du menu est à $19,80$ \euro{}. Et $19,60$ \euro{} ?
\item On note $x$ le nombre moyen de clients et $f(x)$ le prix du menu en \euro{}. Déterminer $m$ et $n$ tel que $f(x) = mx + n$.
\item Quel sera le prix si le nombre de clients est de $120$ ?
\item Quel sera le nombre de clients moyen si le prix est de $16$ \euro{} ? 
\end{enumerate}
\end{enumerate}
\end{document}