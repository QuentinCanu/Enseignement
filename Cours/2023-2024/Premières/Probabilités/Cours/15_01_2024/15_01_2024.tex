\documentclass{article}
\usepackage{main}

\title{Cours : Indépendance d'événements}
\author{Quentin Canu}
\date{15 Janvier 2024}

\begin{document}
\maketitle
\section{Questions Flash}
Formule de $P_B(A)$ ?
\section{Correction}
\section{Exercice}
Dans une urne, il y a $3$ boules blanches, $5$ boules noires et $4$ boules rouges. On tire deux boules l'une après l'autre de l'urne sans les remettre. Quelle est la probabilité que la deuxième boule tirée soit une boule rouge ?
Que se passe-t-il si on remet les boules tirées ?
\section{Cours}
\subsection*{\'Evénements indépendants}
\subsubsection*{Définition et exemples}
\begin{definition}
Soit $A$ et $B$ deux événements, tel que $P(B) \neq 0$. Les événements $A$ et $B$ sont indépendants $P_B(A) = P(A)$   
\end{definition}
\begin{remark}
En clair, cela signifie que la réalisation de $B$ n'a aucune influence sur la probable réalisation de $A$.
\end{remark}
\begin{proposition}
Si on a aussi $P(A) \neq 0$, alors $A$ et $B$ indépendants implique $B$ et $A$ indépendants.
\end{proposition}
\begin{proof}
On veut donc montrer que $P_A(B) = P(B)$. Or
\begin{align*}
P_A(B) &= \dfrac{P(A \cap B)}{P(A)}\\
&= \dfrac{P(A \cap B)}{P_B(A)} \text{ (Par hypothèse)}\\
&= \dfrac{P(A \cap B)}{\dfrac{P(A \cap B)}{P(B)}} \\
&= P(B)
\end{align*}
ce qui conclut.
\end{proof}
\begin{example}
On lance $3$ pièces équilibrées. Quelle est la probabilité que la troisième pièce donne face ? On voit bien que cela ne dépend pas du résultat obtenu par les deux autres pièces. 
\end{example}
\section{Exercice ?}
\end{document}