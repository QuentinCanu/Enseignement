\documentclass{exam}
\usepackage{main}

\qformatExam{}

\date{4 Mars 2023}
\author{Seconde 9}
\title{Contrôle : Proportions et taux d'évolution}

\begin{document}
\maketitle

\begin{itemize}
\item Une présentation soignée est de rigueur, et sera notée sur $2$ points.
\item Tout effort de recherche, même s'il n'aboutit pas, sera valorisé.
\item Tout copie est interdite et sera notée sur $0$.
\item La calculatrice est autorisée.
\end{itemize}
\vspace*{1cm}
\begin{questions}
\titledquestion{Racines carrées}[4]
En précisant les étapes de calcul, pour chacune des fractions suivantes, donner une fraction équivalente sans racine carrée au dénominateur.
\begin{parts}
\vspace*{0.3cm}
\part $\dfrac{1}{\sqrt{37}}$
\vspace*{0.3cm}
\part $\dfrac{\sqrt{36}}{\sqrt{81}}$
\vspace*{0.3cm}
\part $\dfrac{1}{6 - 2\sqrt{7}}$
\vspace*{0.3cm}
\part $\dfrac{5 - \sqrt{11}}{5 + \sqrt{11}}$
\end{parts}
\vspace*{0.5cm}
\titledquestion{Géométrie}[4]
On considère le quadrilatère $ABCD$ dans un repère orthonormé $(O;I;J)$. Dans chacun des cas suivants, dire si oui ou non le quadrilatère $ABCD$ est un parallélogramme.
\begin{parts}
\vspace*{0.3cm}
\part $A(-2;2)$; $B(2;-4)$; $C(4;2)$ et $D(10;0)$
\vspace*{0.3cm}
\part $A(-5;3)$; $B(-10;6)$; $C(0;4)$ et $D(-5;7)$
\vspace*{0.3cm}
\part $A(2;3)$; $B(-1;6)$; $C(-1;-1)$ et $D(-4;2)$
\vspace*{0.3cm}
\part $A(-5;3)$; $B(-2;1)$; $C(-2;-2)$ et $D(1;-5)$
\end{parts}
\vspace*{0.5cm}
\titledquestion{Nature}[2]
Dans une réserve naturelle, $32\%$ des animaux sont des mammifères. Parmi eux, $15\%$ ont moins d'un an.
\begin{parts}
\part Donner la proportion de mammifères de moins d'un an dans cette réserve.
\part Il y a $104$ mammifères dans cette réserve. En déduire le nombre d'animaux dans cette réserve naturelle.
\end{parts}
\vspace*{0.5cm}
\titledquestion{\'Economie}[5]
L'évolution du salaire net mensuel moyen en France est donnée par le tableau suivant
\begin{center}
\begin{tabular}{|l|cccc|}
\hline
Année&Cadres&Professions Intermédiaires&Employés&Ouvriers\\
\hline
2019&4542&2588&1872&1963\\
2020&4629&2623&1909&1977\\
2021&4546&2593&1895&1958\\
2022&4490&2570&1880&1940\\
\hline
\end{tabular}
\end{center}
\vspace*{0.5cm}
\begin{parts}
\part Calculer le taux d'évolution du salaire moyen des professions intermédiaires et des employés entre $2019$ et $2022$. Quelle catégorie professionnelle a le plus vu son salaire moyen diminuer ?
\part 
\begin{subparts}
\subpart Vérifier que le taux d'évolution du salaire moyen d'un ouvrier entre $2020$ et $2021$ est de $-0,9\%$.
\subpart De quel pourcentage le salaire moyen d'un ouvrier en $2021$ devrait augmenter pour atteindre le salaire moyen d'un ouvrier en $2020$ ? 
\end{subparts}
\end{parts}
\vspace*{0.5cm}
\titledquestion{Musique}[3]
Un artiste enregistre son album dans un studio. Chaque titre de musique augmente le prix de la location du studio de $2\%$.
\begin{parts}
\part De combien va augmenter ce prix sachant que l'album présente $3$ titres ?
\part Initialement, le studio coûte $255$ \euro{}. Sachant que la location du studio a coûté environ $281,54$ \euro{}, combien y a-t-il de titres dans l'album ?
\end{parts}
\end{questions}
\end{document}
