\documentclass{article}
\usepackage{main}

\title{Cours : Taux de variation}
\author{Quentin Canu}
\date{15 Janvier 2024}

\begin{document}
\maketitle
\section{Interrogation/Rattrapages (7 min)}
\section{Correction Exercice 57}
Vérifier le travail fait dans les rangs.
\section{Activité (10 minutes)}
On augmente la longueur d'un champ rectangulaire de $20\%$ tout en diminuant de $10\%$ sa largeur. Sa surface a-t-elle augmentée ou diminuée ? De quel pourcentage ?
\section{Cours (40 minutes)}
\subsection*{Variation absolue, variation relative}
\begin{definition}
Soit $V_d$ une valeur de départ qui évolue vers une valeur d'arrivée $V_a$.
\begin{itemize}
\item La variation absolue de $V_d$ vers $V_a$ est donnée par $V_a - V_d$.
\item La variation relative, ou taux d'évolution, est donnée par $\dfrac{V_a - V_d}{V_d}$.
\end{itemize}    
\end{definition}
\begin{remark}
La variation absolue correspondant à ce qu'on rajoute dans l'\emph{absolu} pour aller de $V_d$ à $V_a$. 

La variation relative est la proportion de $V_d$ que l'on rajoute à $V_d$ pour arriver à $V_a$.

Pour avoir le taux d'évolution en pourcentages, il faut le multiplier par $100$.
\end{remark}
\begin{example}
Le prix de l'electricité est passé de $110$ \euro{} par an à $134$ \euro{} par an.
Donner la variation absolue du prix de l'electricité, puis le taux d'évolution en pourcentages.
\end{example}
\end{document}