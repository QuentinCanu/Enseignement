\documentclass{article}
\usepackage{main}
\title{Proportions : Cours}
\date{18 Décembre 2023}
\author{Quentin Canu}

\begin{document}
\maketitle
\section{Evaluation de Cours}
\begin{itemize}
\item Eva
\item Aziz
\item Anais
\item Georgiana
\item Jenna
\item Gift
\item Alia
\end{itemize}
\section{Discussion sur la charte}
\section{Résumé des attendus sur les fonctions}
\begin{itemize}
\item Vocabulaire : fonction, image, antécédent, ensemble de définition \dots
\item Définition de Courbes représentatives.
\item Montrer qu'un point est sur la courbe.
\item Tableau de valeur ou comment tracer la courbe.
\item Lire l'image d'un nombre par une fonction sur sa courbe.
\item Lire les antécédents d'un nombre par une fonction sur sa courbe.
\item Résoudre $f(x) = k$.
\item Résoudre $f(x) \geq k$.
\item Tableau de signes.
\item Résoudre $f(x) = g(x)$.
\item Résoudre $f(x) \geq g(x)$.
\end{itemize}
Contrôle le lundi 8 Janvier.
Devoir Maison à rendre le Vendredi 12 Janvier.
\section{Exercices}
Calculer des proportions sur des exemples.
\end{document}