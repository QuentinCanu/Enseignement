\documentclass{article}[onecolumn]
\usepackage{main}
\usepackage[margin=1.5cm]{geometry}
\pagestyle{empty}

\title{Révisions proportion et taux d'évolution}
\date{5 Février 2024}
\author{}
\renewcommand{\arraystretch}{1.5}

\begin{document}
\begin{table}[h]
\begin{center}
\begin{tabular}{|m{5cm}|m{5cm}|m{5cm}|}
        \hline
        Exemple d'énoncé    & 
        Calcul      &
        Notion du cours \\
        \hline
Il y a $6$ sportifs parmi $12$ lycéens. Quel est le pourcentage de sportifs parmi les lycéens ? 
                    &
\[
\dfrac{6}{12} = \dfrac{1}{2} = 0,5
\]
Réponse : $50\%$.
                        &
Proportions, pourcentages       
                                \\
\hline
Il y a $50\%$ de sportifs parmi les lycéens, et il y a $50\%$ de footballeurs parmi les sportifs. Quel est le pourcentage de footballeurs parmi les lycéens ? 
                    &
\[
0,5 \times 0,5 = 0,25
\]
Réponse : $0,25\%$.
                        &
Pourcentages de pourcentages
                                \\
\hline
On passe de $6$ à $9$ sportifs. De quel pourcentage ce nombre a-t-il augmenté ?
                    &
\[
\dfrac{9 - 6}{6} = \dfrac{3}{6} = 0,5
\] 
Réponse : $50\%$
                        &
Taux d'évolution
                                \\
\hline
Le nombre de sportifs initialement de $6$ a augmenté de $50\%$. Combien y a-t-il de sportifs maintenant ?
                    &
\[
6 \times \left(1 + \dfrac{50}{100}\right) = 6 \times 1,5 = 9
\] 
Réponse: $9$.
                        &
Taux d'évolution, coefficient multiplicateur.
                                \\
\hline
Le nombre de sportifs initialement de $6$ a diminué de $50\%$. Combien y a-t-il de sportifs maintenant ?
                    &
\[
6 \times \left(1 - \dfrac{50}{100}\right) = 6 \times 0,5 = 3
\] 
Réponse: $9$.
                        &
Taux d'évolution, coefficient multiplicateur.
                                \\
\hline
Le nombre de sportifs a augmenté de $30\%$, puis a diminué de $20\%$. Donner le pourcentage d'évolution du nombre de sportifs.
                    &
\[
\begin{aligned}
CM_1 &= 1 + \dfrac{30}{100} = 1,3\\
CM_2 &= 1 - \dfrac{20}{100} = 0,8\\
CM_g &= CM_1 \times CM_2 = 1,04\\
t_g &= CM_g - 1 = 0,04
\end{aligned}
\]
Réponse : $4\%$ d'augmentation.
                        &
Taux d'évolution successifs.
                            \\
\hline
Le nombre de sportifs a augmenté de $25\%$. De quel pourcentage diminuer ce nombre afin d'obtenir la valeur initiale ?
                    &
\[
\begin{aligned}
CM &= 1 + \dfrac{25}{100} = 1,25\\
CM_r &= \dfrac{1}{CM} = 0,8\\
t_r &= CM_r - 1 = -0,2
\end{aligned}
\]
Réponse : $20\%$
                        &
Taux d'évolution réciproque.
                            \\
\hline
\end{tabular}
\end{center}
\end{table}
\end{document}