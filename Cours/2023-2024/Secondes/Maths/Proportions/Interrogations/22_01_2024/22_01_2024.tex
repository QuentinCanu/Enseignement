\documentclass{exam}
\usepackage{main}

\title{Evaluation}
\date{22 Janvier 2023}
\author{Quentin Canu}

\qformatExos{}

\begin{document}
\begin{questions}
\question Développer l'expression suivante :
\begin{equation*}
(2 - \sqrt{3})(5\sqrt{3} + 1)
\end{equation*}
\vspace*{0.5cm}
\question Le prix d'un blouson est passé de $20$ \euro{} à $32$ \euro. De quel pourcentage ce prix a-t-il augmenté ?
\vspace*{0.5cm}
\question Donner le coefficient multiplicateur associé aux taux d'évolutions suivants :
\begin{parts}
\part $48\%$    
\part $-56\%$    
\part $120\%$    
\end{parts}
\vspace*{1cm}
\end{questions}
\begin{questions}
\question Développer l'expression suivante :
\begin{equation*}
(2 - \sqrt{3})(5\sqrt{3} + 1)
\end{equation*}
\vspace*{0.5cm}
\question Le prix d'un blouson est passé de $20$ \euro{} à $32$ \euro. De quel pourcentage ce prix a-t-il augmenté ?
\vspace*{0.5cm}
\question Donner le coefficient multiplicateur associé aux taux d'évolutions suivants :
\begin{parts}
\part $48\%$    
\part $-56\%$    
\part $120\%$    
\end{parts}
\vspace*{1cm}
\end{questions}
\begin{questions}
\question Développer l'expression suivante :
\begin{equation*}
(2 - \sqrt{3})(5\sqrt{3} + 1)
\end{equation*}
\vspace*{0.5cm}
\question Le prix d'un blouson est passé de $20$ \euro{} à $32$ \euro. De quel pourcentage ce prix a-t-il augmenté ?
\vspace*{0.5cm}
\question Donner le coefficient multiplicateur associé aux taux d'évolutions suivants :
\begin{parts}
\part $48\%$    
\part $-56\%$    
\part $120\%$    
\end{parts}
\vspace*{1cm}
\end{questions}

\end{document}