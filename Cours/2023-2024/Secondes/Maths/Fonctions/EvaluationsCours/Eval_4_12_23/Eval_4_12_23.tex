\documentclass{exam}
\usepackage{main}
\title{Evaluation de Cours}
\qformat{\textbf{Exercice \thequestion :} \emph{(\thequestiontitle)\hfill}}

\begin{document}
\maketitle
\begin{questions}
\titledquestion{Vrai ou Faux ?}
Répondre à chacune des questions suivantes par vrai ou faux.

Soit $\function{f}{\R}{\R}{t}{3t + 1}$.
\begin{parts}
\part L'ensemble de définition de $f$ est $\R$.
\part L'image de $2$ par $f$ est $9$.
\part Un antécédent de $10$ par $f$ est $3$.
\end{parts}
\vspace{0.5cm}
\titledquestion{Courbes représentatives}
Soit $f$ une fonction dont la courbe représentative est donnée sur le repère ci-dessous.
\begin{center}
\begin{tikzpicture}
\shorthandoff{:}
\millirepere[step=0.5](-3.25,-3.5) -- (3.25,3.25);
\draw[domain=-3:3] plot (\x,{(2/5)*\x^3 - (13/5)*\x});
\draw[thick] (1,-0.25) -- (1, 0.25) node[right] {$1$};
\draw[thick] (-0.25,1) -- (0.25,1) node[above] {$1$};
\end{tikzpicture}
\end{center}
\begin{parts}
\part Par lecture graphique, quelle est l'image de $3$ par $f$ ?
\part Faire apparaître sur l'axe des abscisses les antécédents de $2$.
\end{parts}
\end{questions}
\end{document}