\documentclass{article}
\usepackage{main}
\title{Cours : Equations et inéquations entre deux fonctions}
\date{11 Décembre 2023}

\begin{document}
\maketitle
\section{Evaluation de Cours (5 minutes)}
\begin{itemize}
\item Karl
\item Taline
\item Eva
\item Sasha
\item Chanelle
\item Bilel
\item Gift
\item Giulia
\end{itemize}
\section{Choses à dire (1 minute)}
\begin{itemize}
\item Fichier geogebra pour les aider à réviser sur pronote.
\item Réussir Maths.
\item Note de participation.
\end{itemize}
\section{Activité (30 minutes)}
Activité $1$ page $248$
\section{Cours (10 minutes)}
\begin{proposition}
\begin{enumerate}
\item Pour résoudre graphiquement une équation de la forme $f(x) = g(x)$, on regarde les abscisses des points d'intersection des courbes représentatives $C_f$ et $C_g$. Tracer exemple.
\item Pour résoudre graphiquement une inéquation de la forme $f(x) \geq g(x)$, on regarde les abscisses des points de la courbe $C_f$ "au-dessus" de ceux de la courbe $C_g$.
\end{enumerate}
Donner un exemple graphique pour accompagner.
\end{proposition}

\end{document}