\documentclass{article}
\usepackage{main}

\title{\'Evénements aléatoires et Probabilités}
\date{}
\author{Quentin Canu}

\begin{document}
\maketitle
\section{Contenu}
\begin{itemize}
\item Ensemble (Univers) des issues.
\item \'Evénements. Réunion, Intersection, Complémentaire.
\item Loi de probabilités. Probabilités d'un événement : somme des probabilités des issues de l'événement.
\item $P(A \cup B) + P(A \cap B) = P(A) + P(B)$.
\item Dénombrement à l'aide de tableau et d'arbre.
\end{itemize}
\section{Capacités}
\begin{itemize}
\item Utiliser des modèles théoriques de référence (dé équilibré, pièce équilibrée, tirage au sort avec équiprobabilité dans la population) en comprenant que les probabilités sont définies a priori.
\item Construire un modèle à partir de fréquences observées, en distinguant nettement modèle et réalité.
\item Calculer des probabilités dans des cas simples : expérience aléatoire à deux ou trois épreuves. 
\end{itemize}
\section{Structure}
\subsection{Vocabulaire des probabilités}
\subsubsection{Univers}
\subsubsection{Evénement}
\subsection{Combinaison d'événements}
\subsection{Probabilités}
\subsubsection{Lois de probabilités}
\subsubsection{Calcul de Probabilité}
\subsection{Modélisation}
\subsubsection{Equiprobabilité}
\subsubsection{Tableaux, arbres}
\end{document}