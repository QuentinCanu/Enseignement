\documentclass{article}
\usepackage{main}

\title{Cours : Fin Géométrie}
\date{15 Mars 2024}
\author{Quentin Canu}

\begin{document}
\maketitle
\section{Questions Flashs}
Soient $A(3;2)$ et $B(9;4)$.
\begin{enumerate}
\item Longueur du segment $[AB]$ ?
\item Coordonnées du milieu $I$ du segment $[AB]$ ?
\end{enumerate}
\section{Correction du DS}
\begin{enumerate}[label=\textbf{Exercice \arabic* :}]
\item Simplifier les fractions de sorte à avoir un entier au dénominateur. 
\begin{enumerate}
    \item $\dfrac{1}{\sqrt{3}}$
    \item $\dfrac{5 + \sqrt{2}}{5 - \sqrt{2}}$
\end{enumerate}
\item Dire si oui ou non le quadrilatère $ABCD$ est un parallélogramme, avec $A(-2;2)$; $B(2;-4)$; $C(8;-4)$ et $D(4;2)$.
\item Proportions de proportions : On a $42\%$ d'oiseau dans un zoo, et parmi ces oiseaux, $5\%$ ont des plumes rouges. Combien y a-t-il d'oiseau à plumes rouges ?
\item Taux d'évolution : Donner la formule du taux d'évolution entre une valeur de départ $V_d$ et une valeur d'arrivée $V_a$. Exemple, taux d'évolution du salaire moyen des employés passant de $1872$ à $1880$ ?
\item Taux d'évolution successifs : On augmente un prix de $2\%$, puis on augmente le résultat de $3\%$. De combien a-t-on augmenté ce prix au final ?
\end{enumerate}
\section{Annonces}
\begin{enumerate}
\item Réussir maths jeudi prochain
\item Al-Khwarizmi
\end{enumerate}
\section{Carte Mentale}
\section{Exercices}
\begin{enumerate}[label=\textbf{Exercice \arabic* : }]
\item Dans un repère orthonormé, on considère $A(3;2)$, $B(9;4)$, $C(1;8)$ et $D(x;y)$ où $x$ et $y$ sont deux réels.
\begin{enumerate}
    \item Montrer que $ABC$ est un triangle rectangle en $A$.
    \item Calculer les coordonnées de $D$ pour que $ABDC$ soit un carré. 
\end{enumerate}
\item Dans un repère orthonormé, on considère les points $A(-2;3)$, $B(-3;1)$ et $C(4;0)$. On note $H$ le pied de la hauteur du triangle $ABC$ passant par $A$.
\begin{enumerate}
\item Faire une figure représentant la situation.
\item Montrer que le triangle $ABC$ est rectangle.
\item En déduire l'aire du triangle, puis la longueur $AH$.
\end{enumerate}
\end{enumerate}
\section{Devoirs}
\begin{enumerate}
\item Identifier les éléments de la carte mentale pour toutes les questions de la page 175.
\item Faire les exercices de la feuille (exercices 60 et 63 page 175 dans le manuel).
\item Faculatatif, faire l'exercice 66 page 175.
\end{enumerate}
\end{document} 