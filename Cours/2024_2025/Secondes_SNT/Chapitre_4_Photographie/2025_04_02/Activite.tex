\documentclass{article}
\usepackage{main}

\title{Fichiers d'images : Compression}
\date{15 Mars 2024}
\author{Seconde 9}

\newcommand{\Rien}{Blanc $(255,255,255)$}
\newcommand{\Noir}{Noir $(0,0,0)$}
\newcommand{\Fonce}{Bleu $(35,39,83)$}
\newcommand{\Orange}{Orange $(255,162,5)$}
\newcommand{\Jaune}{Jaune $(254,255,36)$}
\newcommand{\Rouge}{Rouge $(251,1,70)$}
\newcommand{\Marron}{Marron $(170,82,48)$}


\begin{document}
\maketitle
\thispagestyle{empty}
\pagestyle{empty}
Nous allons aujourd'hui étudier la manière dont sont stockées les images dans un ordinateur.

\section{Image non compressée}
L'une des manières de décomposer une image est de considérer ses \emph{pixels} : une image est alors vue comme un tableau de cellules, chacune affichant une couleur. Chaque pixel contient des données définissant la couleur à afficher.

\begin{enumquestions}
\item Notre image a pour résolution $8 \times 8$. Combien de pixels cette image comporte-t-elle ?
\item Sachant qu'une couleur est stockée dans un octet, combien d'octets sont nécessaires pour stocker cette image ?
\item Même question pour une image de taille $1920 \times 1080$.
\item Trouver une image sur Internet dans le format $1920 \times 1080$, et donner sa taille en octets. Cela correspond-t-il à votre réponse précédente ?
\end{enumquestions}

\section{Image compressée}

Compresser une image revient à décrire les pixels de l'image de façon plus efficace.
\begin{enumquestions}
\item Aller sur le site \url{piskelapp.com}, puis cliquer sur \url{Create Sprite}. Suivre les instructions au dos du sujet ligne par ligne pour colorier une image de $8 \times 8$ pixels. L'onglet palette sert à gérer les couleurs possibles.
\item Proposer une autre série d'instructions permettant à un autre élève de la classe de produire la même image, mais plus efficacement. Faites tester un camarade.
\end{enumquestions}

\newpage 
\section*{Instructions}
\begin{multicols}{3}
\begin{enumerate}
\multido{}{1}{\item \Rien}
\multido{}{2}{\item \Fonce}
\multido{}{4}{\item \Rien}
\multido{}{1}{\item \Fonce}
\multido{}{2}{\item \Rien}
\multido{}{1}{\item \Jaune}
\multido{}{1}{\item \Orange}
\multido{}{3}{\item \Rien}
\multido{}{1}{\item \Orange}
\multido{}{3}{\item \Rien}
\multido{}{4}{\item \Jaune}
\multido{}{1}{\item \Orange}
\multido{}{2}{\item \Orange}
\multido{}{1}{\item \Rien}
\multido{}{1}{\item \Jaune}
\multido{}{1}{\item \Noir}
\multido{}{2}{\item \Jaune}
\multido{}{1}{\item \Noir}
\multido{}{2}{\item \Orange}
\multido{}{1}{\item \Rien}
\multido{}{1}{\item \Rouge}
\multido{}{3}{\item \Jaune}
\multido{}{1}{\item \Orange}
\multido{}{1}{\item \Rien}
\multido{}{1}{\item \Orange}
\multido{}{1}{\item \Rien}
\multido{}{1}{\item \Jaune}
\multido{}{3}{\item \Orange}
\multido{}{1}{\item \Rien}
\multido{}{1}{\item \Rien}
\multido{}{1}{\item \Orange}
\multido{}{1}{\item \Jaune}
\multido{}{1}{\item \Orange}
\multido{}{1}{\item \Jaune}
\multido{}{1}{\item \Orange}
\multido{}{1}{\item \Jaune}
\multido{}{1}{\item \Rien}
\multido{}{2}{\item \Rien}
\multido{}{1}{\item \Jaune}
\multido{}{1}{\item \Orange}
\multido{}{2}{\item \Marron}
\multido{}{1}{\item \Orange}
\multido{}{1}{\item \Rien}
\item[\vspace{\fill}]
\end{enumerate}
\end{multicols}
\end{document}
