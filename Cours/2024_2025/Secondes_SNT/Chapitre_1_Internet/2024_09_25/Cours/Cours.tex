\documentclass{article}
\usepackage{main}

\title{Internet : structure du réseau}
\date{25 Septembre 2024}
\author{Quentin Canu}

\begin{document}
\maketitle

\section{Réseau internet}

Internet est un réseau d'ordinateurs interconnectés dont les communications sont gérées par des protocoles nommés TCP et IP.

Il est constitué d'ordinateurs et autres machines personnelles, qui peuvent être interconnectées au niveau local à l'aide de \emph{switchs ou commutateurs}. La box internet fait office de switch: tous les objets connectés au réseau sont connecté au même réseau local (LAN).

Pour connecter entre eux plusieurs réseaux LAN, on utilise des ordinateurs dédiés appelés \emph{routeurs}.

Pour échanger des données entre deux ordinateurs de réseaux différent, chaque routeur doit utiliser une \emph{table de routage} qui décide quel nouveau routeur doit continuer la transmission. Cela se fait grâce à l'\emph{adresse IP}.

Format de l'adresse IPv4: 255.255.255.255
\end{document}