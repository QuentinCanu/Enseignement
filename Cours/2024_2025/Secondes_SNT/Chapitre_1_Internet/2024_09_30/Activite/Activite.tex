\documentclass{article}
\usepackage{main}

\title{Protocole TCP}
\date{30 Septembre 2024}
\author{Seconde 9}

\begin{document}
\maketitle

\begin{tcolorbox}
Nous avons vu la semaine dernière les principes derrière le fonctionnement du protocole IP. Tous les ordinateurs du réseau sont identifiés à l'aide d'un identifiant unique appelé adresse IP. Maintenant que l'on sait comment trouver notre destinataire, la question est la suivante : comment être sûr que le message sera bien reçu par le destinataire ?
\end{tcolorbox}

\section{Courrier postal}
Vous souhaitez envoyer à votre correspondant ou votre correspondante une photocopie de l'image de votre plat préféré par la Poste. Malheureusement, votre photo est de dimension $100 \unit{\centi\meter} \times 50 \unit{\centi\meter}$, et la Poste ne transmet que des photos de dimension $10 \unit{\centi\meter} \times 5 \unit{\centi\meter}$. Heureusement, vous avez chez vous autant de timbres que vous le souhaitez, un cutter et des post-it. Votre correspondant ou votre correspondante possède de la colle.

\textbf{Important ! Vous ne pouvez communiquer qu'au moyen de la poste. D'autres moyens (téléphonie, internet...) sont proscrits dans cette exercice.} 

\begin{enumquestions}
\item En combien de morceaux devez-vous découper votre image pour pouvoir envoyer tous les morceaux par la poste ?
\item Pourquoi cela ne suffit-il pas d'envoyer tous les morceaux par la Poste ?
\item Il vous est possible de coller sur chaque morceau un post-it, sur lequel vous pouvez y noter quelque chose. Proposer une façon d'annoter les post-it permettant à votre correspondant ou correspondante de reconstituer l'image.
\end{enumquestions}
\vspace*{0.5cm}
\emptybox{11cm}
\end{document}