\documentclass{article}
\usepackage{main}

\title{Activité : URL et HTTP}
\author{Seconde 9}
\date{5 Mars 2025}

\begin{document}
\maketitle

\section{URL}
\begin{tcolorbox}
Le Web est constitué de serveurs hébergeant des pages web. Une page web est identifiée à l'aide d'une adresse URL. Une adresse URL utilise le format suivant :
\url{https://ent.iledefrance.fr/welcome}

Il est composé de plusieurs éléments :
\begin{itemize}
\item \url{https://} : Le protocole utilisé (voir partie \ref{http}). La différence avec \url{http://} est la fait que la connexion est sécurisée.
\item \url{ent.iledefrance.fr} : L'adresse symbolique du serveur. Il correpond à l'adresse présent sur votre DNS : on peut donc remplacer cette adresse par l'adresse IP correspondante.
\item \url{/welcome} : L'emplacement de la page web correspondante sur le serveur.  
\end{itemize}
\end{tcolorbox}
\begin{enumquestions}
\item Certaines adresses URL contiennent aussi des informations supplementaires pour aider votre navigateur. Dans chacune des situations, identifier cette information supplémentaire.
\begin{itemize}
\item Ancre (\#) \url{https://fr.wikipedia.org/wiki/World_Wide_Web} et \url{https://fr.wikipedia.org/wiki/World_Wide_Web#Histoire}

\answersline

\answersline
\item Paramètre (?) \url{https://www.google.com/} et \url{https://www.google.com/search?q=triangle+de+sierpinski}

\answersline

\answersline
\end{itemize}
\item Identifier les composantes des addresses URL suivantes :
\begin{itemize}
\item \url{https://www.larrousse.fr/dictionnaires/anglais-francais/request}

\answersline
\item \url{https://www.youtube.com/watch?v=e6uLDvUUs8A}

\answersline
\end{itemize}
\item 
\begin{enumerate}
\item Où vous envoie l'adresse URL \url{https://172.217.20.174/search?q=flocon+de+koch} sur votre navigateur ?
\item Trouver l'adresse IP du serveur \url{random.org} et donner l'adresse URL équivalente de \url{https://www.random.org/playing-cards/}. 
\end{enumerate}
\end{enumquestions}
\newpage
\section{HTTP}
\label{http}
\begin{tcolorbox}
Le client (votre navigateur) communique avec les serveurs web à l'aide du protocole HTTP (HyperText Transfert Protocol). Le protocole HTPP est basé sur un système de requête : à chaque action de votre part, votre navigateur envoie une demande à un serveur qui vous renvoie de l'information, comme le code HTML d'une page web.

On retiendra les requêtes suivantes qui sont les plus courantes :
\begin{itemize}
\item \textbf{GET} : le client demande une information au serveur (exemple : obtenir le code html d'une page web afin de l'afficher).
\item \textbf{POST} : Le client envoie lui-même des informations au serveur (exemple : vous envoyez votre mot de passe au serveur afin de vous connecter à votre compte).
\item \textbf{HEAD} : le client demande uniquement les informations d'en-tête au serveur.
\end{itemize}
\end{tcolorbox}
\begin{enumquestions}
\item Quand vous tapez sur un navigateur \url{https://fr.wikipedia.org/wiki/Alan_Turing}, en réalité, il envoie une requête \textbf{GET} ayant le format suivant.
\begin{lstlisting}
GET /wiki/Alan_Turing HTTP/1.1
Host: fr.wikipedia.org
\end{lstlisting}
Quelle requête \textbf{GET} envoie votre navigateur quand vous tapez \url{https://www.youtube.com/watch?v=7ell8KEbhJo} ?
\vspace*{0.3cm}

\emptybox{4cm}
\item La réponse d'un serveur est aussi sous un certain format :
\begin{lstlisting}
HTTP/1.1 200 OK
Date: Mon, 4 Mar 2019 22:38:34 GMT
Content-Type: text/html; charset=UTF-8
Content-Length: 983
Last-Modified: Wed, 08 Jan 2003 23:11:55 GMT

<html>
<head><title>Une page de test</title></head>
<body><p>Un simple paragraphe de texte.</p></body>
</html>
\end{lstlisting}
\begin{enumerate}
\item De quand date la requête ? \answersline
\item Quand la page a-t-elle été modifiée la dernière fois ? \answersline
\item Quel est le titre de cette page ? \answersline 
\end{enumerate}
\item Pour tester la production de requêtes \textbf{GET}, on pourra utiliser l'invite de commandes windows. Sur la barre de recherche en bas de votre écran, taper \og cmd \fg puis entrée. Ensuite, tapez
\begin{lstlisting}
curl --verbose duckduckgo.com
\end{lstlisting}
Analysez la réponse obtenue : identifier la requête \textbf{GET} générée, puis la réponse du serveur à la requête.
\end{enumquestions}
\end{document}