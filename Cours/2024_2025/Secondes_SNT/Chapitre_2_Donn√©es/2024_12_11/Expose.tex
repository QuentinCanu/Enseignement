\documentclass{article}
\usepackage{main}

\title{Exposés sur les données}
\date{11 Décembre 2024}
\author{Seconde 9}

\begin{document}
\maketitle

\section*{Instructions}
Vous allez choisir un thème parmi quatre autour de la notion de donnée, et de son impact sur les pratiques humaines.

Une fois votre sujet choisi, vous allez devoir préparer un exposé présentant votre concept. Il faudra nécessairement présenter :
\begin{itemize}
\item Une définition du concept.
\item La ou les problématiques que votre concept permet de résoudre. 
\item Un court historique de votre concept.
\item Les défis ou les inconvénients soulevés par votre concept.
\end{itemize}

Votre présentation devra durer $5$ minutes.

Parallèlement, vous devez préparer $5$ questions en rapport avec votre sujet. Vos camarades, le public de votre exposé, auront comme tâche de répondre à ces questions, et auront un point bonus en SNT s'ils répondent correctement à ces questions.

Vous serez notés sur la qualité de votre présentation, sur son contenu, et sur la pertinence des questions que vous aurez préparées pour vos camarades.
\section*{Notez le sujet choisi : }
\section*{Questions choisies :}
\section{}
\section{}
\section{}
\section{}
\section{}
\end{document}