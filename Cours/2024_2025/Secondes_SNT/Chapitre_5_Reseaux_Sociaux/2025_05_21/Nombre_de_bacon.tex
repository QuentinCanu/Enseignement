\documentclass{article}
\usepackage{main}

\title{Petits Mondes}
\date{21 Mai 2025}
\author{Seconde 9}

\begin{document}
\maketitle
\section{Expérience de Milgram}
Consultez la page de l'Étude du petit monde sur wikipedia.

\begin{enumquestions}
\item En quoi consiste l'expérience que Stanley Milgram a conduite en 1967 ?
\item Que signifie l'expression \og six degrés de séparation \fg ?
\item Quelles critiques ont été faites vis à vis des résultats obtenus par Milgram ?
\end{enumquestions}

\vspace*{0.5cm}
\emptybox{6cm}
\section{Nombre de Bacon}
Kevin Bacon est un acteur très central à Hollywood. Il a collaboré avec de nombreux comédiens et comédiennes lors de sa carrière. Le \textbf{nombre de Bacon} est un indicateur de l'interconnection régnant dans le monde du cinéma.

Pour déterminer le nombre de Bacon d'un acteur ou d'une actrice, il faut passer par la procédure suivante:
\begin{itemize}
\item Si l'acteur en question est Kevin Bacon, alors son nombre de Bacon est 0.
\item Si l'acteur ou l'actrice en question partage son nom avec celui de Kevin Bacon au générique d'un film, alors son nombre de Bacon est 1.
\item Si l'acteur ou l'actrice en question partage son nom au générique d'un film avec un acteur ou une actrice dont le nombre de Bacon est 1, alors son nombre de Bacon est 2.
\item Et ainsi de suite\dots
\end{itemize}
\begin{enumquestions}
\item Trouver le nombre de Bacon de 5 acteurs de votre choix.
\item Trouver l'acteur dont le nombre de Bacon est le plus élevé.
\item Se renseigner sur le nombre de Erdős-Bacon-Sabbath, et trouver quelques personnalités ayant un tel nombre.
\end{enumquestions}
\end{document}