\documentclass{exam}
\usepackage{mainExam}

\title{Contrôle : Suites Géométriques}
\author{Terminale STMG2}
\date{21 Mars 2025}
\begin{document}
\maketitle
\instructions{autorisée}

\begin{questions}
\titledquestion{Sommes géométriques}[5]
\begin{parts}
\part[3] Calculer les sommes suivantes, à l'aide de la formule du cours.
\begin{subparts}
\subpart $S_1 = 1+3+3^2+3^3+\dots+3^{12}$
\subpart $S_2 = 1+1,7+1,7^2+1,7^3+\dots+1,7^8$
\subpart $S_3 = 1+0,8+0,8^2+0,8^3+\dots+0,8^{21}$
\end{subparts}
\part[2] Démontrer que la somme $S = 1 + 2 + 2^2 + 2^3 + \dots + 2^{63}$ vaut
\begin{equation*}
S = 2^{64} - 1
\end{equation*}
à l'aide de la démonstration du cours.
\end{parts}
\vspace*{0.5cm}
\titledquestion{Évolutions successives}[4]
Lors de l'année $2020$, le prix du loyer moyen d'une métropole est de \num{1500} €. On estime que chaque année, le prix diminue de $5\%$. On pose $(p_n)$ le prix de cette technologie lors de l'année $2020 + n$
\begin{parts}
\part[1] Calculer $p_1$, $p_2$, et $p_3$.
\part[1] Montrer que $(p_n)$ est une suite géométrique, en précisant son premier terme et sa raison.
\part[1] En déduire une expression explicite de $p_n$ en fonction de $n$.
\part[1] À partir de quelle année le loyer moyen sera inférieur à \num{800} € ?
\end{parts}
\vspace*{0.5cm}
\titledquestion{Calcul de termes}[6]
On suppose que $(u_n)_{n \in \N}$ est une suite géométrique. Dans chacun des contextes suivants, calculer $u_5$. Les questions sont indépendantes.
\begin{parts}
\part[1] La raison de $(u_n)$ est $q = 4$ et son premier terme est $u_0 = 8$.
\part[1] La raison de $(u_n)$ est $q = 3,5$ et $u_7 = 8$.
\part[1] $u_4 = 15$ et $u_6 = 135$
\part[1] $u_4 = 128$ et $u_6 = 32$
\part[2] $u_6 = 49$ et $u_8 = 2401$
\end{parts}

\vspace*{0.5cm}
\titledquestion{Suites arithmético-géométriques}[5]
Soit $(u_n)_{n \in \N}$ une suite définie par $u_0 = 5$ et $u_{n+1} = 3u_n + 1$.
\begin{parts}
\part[0,5] Calculer $u_1$, $u_2$ et $u_3$.
\part[1] La suite $(u_n)$ est-elle arithmétique ? La suite $(u_n)$ est-elle géométrique ?
\part[1,5] Pour tout $n$, on pose $v_n = u_n + 0,5$. Montrer que la suite $(v_n)$ est géométrique de raison $3$.
\part[0,5] En déduire une expression de $v_n$ en fonction de $n$.
\part[0,5] En déduire une expression de $u_n$ en fonction de $n$.
\part[1] Calculer alors $u_{15}$.
\end{parts}
\end{questions}

\end{document}