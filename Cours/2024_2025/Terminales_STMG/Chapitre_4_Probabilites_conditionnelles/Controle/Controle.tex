\documentclass{exam}
\usepackage{mainExam}

\title{Contrôle : Probabilités Conditionnelles}
\author{Terminale STMG2}
\date{14 Février 2025}

\begin{document}
\maketitle
\instructions{Autorisée}

\begin{questions}
\titledquestion{Arbre de probabilités}[5]
On procède à l'expérience aléatoire suivante : il y a dans une urne trois boules rouges et deux boules bleues. On tire successivement deux boules de cette urne, sans remettre la première à l'intérieur. On pose les événements suivants :
\begin{itemize}
\item $B_1$ \og La première boule tirée est bleue \fg
\item $B_2$ \og La deuxième boule tirée est bleue \fg
\end{itemize}
\begin{parts}
\part[1] Décrire en français l'événement $\overbar{B_1}$.
\part[1] Compléter l'arbre pondéré suivant.
\begin{center}
\begin{tikzpicture}
\node (B1) at (2,1.5) {$B_1$};
\node (R1) at (2,-1.5) {$\overbar{B_2}$};
\node (B1B2) at (5,2.5) {$B_2$};
\node (B1R2) at (5,0.5) {$\overbar{B_2}$};
\node (R1B2) at (5,-0.5) {$B_2$};
\node (R1R2) at (5,-2.5) {$\overbar{B_2}$};

\draw (0,0) -- (B1) node[midway,above left] {$\dots$};
\draw (0,0) -- (R1) node[midway,below left] {$\dots$};
\draw (B1) -- (B1B2) node[midway,above] {$\dots$};
\draw (B1) -- (B1R2) node[midway,below] {$\dots$};
\draw (R1) -- (R1B2) node[midway,above] {$\dots$};
\draw (R1) -- (R1R2) node[midway,below] {$\dots$};
\end{tikzpicture}
\end{center}
\part[1] Expliquer en une phrase à quelle probabilité correspond $P_{B_1}(B_2)$, et donner sa valeur par lecture sur l'arbre pondéré.
\part[1] Expliquer en une phrase à quoi correspond l'événement $B_1 \cap B_2$, puis calculer $P(B_1 \cap B_2)$.
\part[1] Calculer $P(B_2)$.
\end{parts}
\vspace*{0.5cm}
\titledquestion{Tableau}[5]
On interroge le public d'un festival de musique sur ce qu'il sont venus voir, ainsi que sur leur âge. Aucun n'est allé voir à la fois du black metal et du death metal. Le résultat de cette étude est consignée sur le tableau suivant :
\begin{center}
\begin{tabular}{|c|c|c|c|}
\hline
& Black Metal & Death Metal & Total\\
\hline
Moins de 18 ans & 78 & 72 & 150\\
\hline
Entre 18 et 30 ans & 237 & 63 & 300\\
\hline
Plus de 30 ans & 135 & 415 & 550\\
\hline
Total & 450 & 550 & 1000\\
\hline
\end{tabular}
\end{center}
On tire une personne au hasard dans cette foule. On considère les événements suivants :
\begin{itemize}
\item $B$ \og La personne intérrogée est venu voir du Black Metal \fg
\item $D$ \og La personne intérrogée est venu voir du Death Metal \fg
\item $M$ \og La personne interrogée a moins de 18 ans \fg
\item $V$ \og La personne interrogée a entre 18 ans et 30 ans \fg
\item $T$ \og La personne interrogée a plus de 30 ans \fg
\end{itemize}
\begin{parts}
\part[1] Calculer $P(B)$ et $P(T)$.
\part[2] Calculer $P(D \cap V)$ et $P(M \cap B)$.
\part[2] Calculer $P_B(T)$ et $P_V(B)$.
\end{parts}
\vspace*{0.5cm}

\titledquestion{Ressources humaines}[7]
Une entreprise fait une campagne de recrutement qui se déroule de la façon suivante : Les candidats et candidates passent un test d'entrée ayant $40\%$ de réussite. Ceux et celles qui réussissent le test passent alors un premier entretien de motivation qui accepte alors $70\%$ des candidats. Enfin, les derniers candidats et candidates retenus rencontrent le directeur des ressources humaines qui n'embauche que $25\%$ des candidats qu'il rencontre.

On choisit au hasard le dossier d'un candidat ou d'une candidate. On considère les événements suivants :
\begin{itemize}
\item $D$ \og Le candidat ou la candidate a réussi le test d'entrée \fg
\item $E$ \og Le candidat ou la candidate a réussi l'entretien de motivation \fg
\item $F$ \og Le candidat ou la candidate est accepté par le directeur des ressources humaines \fg
\end{itemize}
\begin{parts}
\part[1] Compléter l'arbre pondéré suivant.
\begin{center}
\begin{tikzpicture}
\node (D) at (2,2) {$D$};
\node (DN) at (2,0) {$\overbar{D}$};
\node (E) at (4, 3) {$E$};
\node (EN) at (4, 1) {$\overbar{E}$};
\node (F) at (6, 4) {$F$};
\node (FN) at (6, 2) {$\overbar{F}$};

\draw (0,1) -- (D);
\draw (0,1) -- (DN);
\draw (D) -- (E);
\draw (D) -- (EN);
\draw (E) -- (F);
\draw (E) -- (FN);
\end{tikzpicture}
\end{center}
\part[1] Calculer la probabilité qu'un candidat ou une candidate soit reçu.
\part[1] En déduire que la probabilité qu'un candidat ne soit pas engagé à la suite de cette campagne est de $0,93$.
\part[3] On suppose qu'un frère et sa sœur passent tous les deux la même campagne de recrutement. Leurs performances sont considérées indépendantes. On note $F_1$ \og Le frère est engagé \fg et $F_2$ \og La sœur est engagée \fg. \begin{subparts}
\subpart Que signifie en français $\overbar{F_1} \cap \overbar{F_2}$ ?
\subpart On admet que puisque $F_1$ et $F_2$ sont indépendantes, alors $\overbar{F_1}$ et $\overbar{F_2}$ sont aussi indépendantes. En déduire que $P(\overbar{F_1} \cap \overbar{F_2}) \simeq 0,86$.
\subpart En déduire que la probabilité qu'au moins un des deux soit engagé est d'environ $0,14$.
\end{subparts}
\part[1] S'inspirer des réponses précédentes pour vérifier que si l'on considère trois candidatures, alors la probabilité qu'au moins un des trois candidats soit embauché est d'environ $0,2$.
\end{parts}
\vspace*{0.5cm}

\end{questions}
\end{document}