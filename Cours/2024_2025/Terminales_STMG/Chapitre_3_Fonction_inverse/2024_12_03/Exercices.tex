\documentclass{article}
\usepackage{main}
\title{Exercice type}
\date{3 Décembre 2024}
\author{Quentin Canu}

\begin{document}
\maketitle

\section{Exercice}

Soit $f \colon x \mapsto 4x + \dfrac{16}{x}$.

\begin{enumquestions}
\item Donner l'ensemble de définition de $f$.
\item Vérifier que $f$ est dérivable sur son ensemble de définition calculer sa dérivée.
\item \label{verif} Vérifier que sa dérivée vaut, pour tout $x$,
\begin{equation*}
f'(x) = \dfrac{4(x-2)(x+2)}{x^2}
\end{equation*}
\item En déduire le tableau de variation de $f$. Il faudra donner les valeurs aux limites des bornes.
\end{enumquestions}

\section{D'autres exemples}

Faire le même exercice, mais en remplaçant $f$ par sa nouvelle valeur, ainsi que la valeur $f'$ pour la question \ref{verif}.

\begin{enumquestions}
\item $f(x) = -5x - \dfrac{5}{x}$ et $f'(x) = \dfrac{-5(x-1)(x+1)}{x^2}$
\item $f(x) = 8x + \dfrac{72}{x} $ et $f'(x) = \dfrac{8(x-3)(x+3)}{x^2}$
\item $f(x) = \dfrac{3}{2}x^2 - 7x - \dfrac{4}{x}$ et $f'(x) = \dfrac{(x-2)(x-1)(3x+2)}{x^2}$
\item $f(x) = \dfrac{7}{2}x^2 - 37x - \dfrac{144}{x}$ et $f'(x) = \dfrac{(x-3)(x-4)(7x+12)}{x^2}$
\item $f(x) = x - \dfrac{5}{x}$ et $f'(x) = \dfrac{x^2 + 5}{x^2}$
\end{enumquestions}

\section{Créer vos exemples}

Pour créer vos propres exemples, vous pouvez partir de la valeur à vérifier pour $f'(x)$. Par exemple, il est facile de fournir des exemples quand $f'(x)$ est de la forme $\dfrac{a(x-r)(x+r)}{x^2}$. Développez, puis demandez-vous quelle fonction $f$ donnerait cette dérivée.

\end{document}