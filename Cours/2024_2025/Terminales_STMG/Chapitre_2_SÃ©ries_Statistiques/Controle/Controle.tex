\documentclass{exam}
\usepackage{mainExam}

\title{Séries statistiques à deux variables}
\date{12 Novembre 2024}
\author{Terminale STMG2}

\begin{document}
\maketitle

\instructions{interdite}

\begin{questions}
\question À l'aide de votre liste de données (ou de la liste de donnée fournie), suivre les instructions suivantes. Vous aurez besoin d'une copie à carreaux et d'une autre copie visant à justifier votre démarche.
\begin{parts}
\part Tracer sur une copie à carreaux un repère orthogonal à deux axes gradués. Chaque axe correspond à une des deux variables de votre liste. Justifier sur votre deuxième copie le choix de votre graduation.
\part Tracer une droite de contrôle donnant une tendance sur l'évolution de vos données. Expliquer sur votre deuxième copie votre méthodologie pour tracer cette droite.
\part Rédiger un paragraphe explicitant quelles informations peut-on déduire de votre nuage de points :
\begin{itemize}
\item La droite de contrôle semble-t-elle justifiée ? Si oui, préciser quelles interpolations ou extrapolation peut-on en déduire. Si non, le justifier en disant si une autre courbe aurait été justifiée ou non.
\item Des groupes de points sont-ils identifiables ? Si oui, dire à quoi correspondent ces groupes. Si non, dire pourquoi. 
\end{itemize}
\end{parts}
\end{questions}
\end{document}