\documentclass{exam}
\usepackage{mainExam}

\title{Contrôle : Probabilités conditionnelles, indépendance}
\author{Première Spécialité Mathématiques}
\date{15 Janvier 2024}

\begin{document}
\maketitle
\begin{questions}
\titledquestion{Arbre pondéré}[5]
Un prof de maths prépare deux contrôles pour ses élèves : le contrôle de Décembre et le contrôle de Janvier.

Il y a une probabilité de $60\%$ qu'il inclut en décembre un exercice de géométrie :
\begin{itemize}
\item Si c'est le cas, alors il inclura un exercice de géométrie en Janvier avec une probabilité de $12\%$.
\item Si ce n'est pas le cas, alors il inclura un exercice de géométrie en Janvier avec une probabilité de $45\%$.
\end{itemize}

On note $D$ \og Le contrôle de décembre comporte un exercice de géométrie \fg et $J$ \og Le contrôle de janvier comporte un exercice de géométrie \fg.

\begin{parts}
\part[1] Dessiner un arbre pondéré présentant la situation.
\part[1] Dire en français à quoi correspond la probabilité $P_D(J)$, puis donner sa valeur.
\part[1] Dire en français à quoi correspond l'événement $D \cap J$, puis calculer $P(D \cap J)$.
\part[1] Calculer $P(J)$.
\part[1] Dire en français à quoi correspond l'événement $D \cup J$, puis calculer $P(D \cup J)$.
\end{parts}
\vspace*{1cm}

\titledquestion{Football}[5]
On considère les 23 joueurs vainqueurs de la coupe du monde de Football, et on regarde leur poste et s'ils jouaient en France ou à l'étranger durant la saison 2017-2018.
\begin{center}
\begin{tabular}{|c|c|c|c|}
\hline
&France&Étranger&Total\\
\hline
Gardien&$2$&$1$&$3$\\
\hline
Défenseur&$3$&$5$&$8$\\
\hline
Milieu&$1$&$5$&$6$\\
\hline
Attaquant&$3$&$3$&$6$\\
\hline
Total&$9$&$14$&$23$\\
\hline
\end{tabular}
\end{center}
On tire un joueur au hasard. On définit les événements suivants :
\begin{itemize}
\item $G$ \og Le joueur est gardien \fg
\item $D$ \og Le joueur est défenseur \fg
\item $M$ \og Le joueur est milieu \fg
\item $A$ \og Le joueur est attaquant \fg
\item $F$ \og Le joueur a joué en France durant la saison 2017-2018 \fg 
\end{itemize}
\begin{parts}
\part[1] Calculer $P(G)$ et $P(F)$.
\part[1] Calculer $P_M(F)$, $P_F(M)$ et $P_F(A)$.
\part[1] Calculer $P_G(\overbar{F})$ et $P_{\overbar{F}}(D)$.
\part[1] Trouver une probabilité conditionnelle valant $\dfrac{5}{8}$.
\part[1] Calculer $P_{G \cup D}(F)$ et $P_{\overbar{F}}(M \cup A)$.
\end{parts}
\vspace*{1cm}
\titledquestion{Formule de Bayes}[5]
\begin{parts}
\part[1] Soit une expérience aléatoire d'univers $\Omega$, et $A$ et $B$ deux événements d'$\Omega$. On suppose de plus que $P(A) \neq 0$ et $P(B) \neq 0$. Rappeler l'expression de $P_A(B)$ en fonction de $P(A \cap B)$ et de $P(A)$.
\part[2] En déduire que $P_B(A) = \dfrac{P(A)}{P(B)}P_A(B)$ (Théorème de Bayes).
\part[2] Soit une urne opaque contenant deux boules rouges et trois boules vertes. On tire deux boules successivement sans remise. On note $A$ \og la première boule est verte\fg et $B$ la deuxième boule est verte.
\begin{subparts}
\subpart[1] Calculer $P(B)$.
\subpart[1] En déduire $P_B(A)$ à l'aide du théorème de Bayes.
\end{subparts}
\end{parts}
\titledquestion{Urnes}[5]
Une urne contient 3 boules : 1 rouge, 1 verte et une bleue. On vide l'urne après tirages successifs des boules (sans remise). On considère les événements suivants :
\begin{itemize}
\item $A$ \og La boule rouge est tirée avant la boule bleue\fg;
\item $B$ \og La boule rouge est tirée lors du premier tirage\fg;
\item $C$ \og La boule rouge est tirée lors du deuxième tirage\fg;
\end{itemize}
\begin{parts}
\part[1] Déterminer les probabilités de ces trois événements.
\part[2] Les événements $A$ et $B$ sont-ils indépendants ?
\part[2] Les événements $A$ et $C$ sont-ils indépendants ?
\end{parts}
\end{questions}
\end{document}