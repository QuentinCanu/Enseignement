\documentclass{article}
\usepackage{main}

\title{Le paradoxe des bancs}
\author{Première Spécialité Mathématiques}
\date{26 Novembre 2024}

\begin{document}
\maketitle

\begin{tcolorbox}
Dans un square, il y a trois bancs à deux places. Mohammed et Natalie viennent s'asseoir successivement en choisissant une place au hasard. Mohammed s'assoit en premier, et Natalie ne peut pas choisir la même place que Mohammed. Quelles sont les chances que Mohammed et Natalie se retrouvent sur le même banc ? 
\end{tcolorbox}

\section{Représenter l'expérience aléatoire}
\begin{enumquestions}
\item Proposer un tableau à double entrée pour représenter l'ensemble des issues possibles.
\item De la même manière, proposer un arbre de dénombrement pour représenter l'expérience.

\vspace*{0.5cm}
\begin{minipage}{0.45\textwidth}
\textbf{Tableau}
\vspace*{0.5cm}

\emptybox{10cm}
\end{minipage}
\hfill\vline\hfill
\begin{minipage}{0.45\textwidth}
\textbf{Arbre}
\vspace*{0.5cm}

\emptybox{10cm}
\end{minipage}
\vspace*{0.5cm}
\item Comment utiliser ces représentations pour en déduire la probabilité recherchée ?
\vspace*{0.5cm}

\emptybox{2cm}
\end{enumquestions}

\newpage
\section{Modélisation d'expériences aléatoires}
\begin{enumquestions}
\item Comparer la réponse de la question précédente avec d'autres membres de la classe. Que remarquez-vous ?
\item Pourquoi l'énoncé rend-elle ce paradoxe possible ?
\item Proposer une nouvelle formulation de cette expérience aléatoire pour éviter toute ambiguïté.
\end{enumquestions}
\vspace*{0.5cm}

\emptybox{6cm}

\section{Autre paradoxe : le paradoxe des enfants}

\begin{tcolorbox}
Un foyer accueille deux enfants nés le même jour. L'un des deux est un garçon, quelle est la probabilité que ce foyer accueille une fille ?
\end{tcolorbox}

\begin{enumquestions}
\item Quelle est votre réponse a priori ?
\item Représenter la situation à l'aide d'un tableau, en veillant à indiquer quelles sont les issues autorisées par l'expérience.
\item En déduire la probabilité recherchée. 
\end{enumquestions}
\vspace*{0.5cm}

\emptybox{6cm}
\end{document}