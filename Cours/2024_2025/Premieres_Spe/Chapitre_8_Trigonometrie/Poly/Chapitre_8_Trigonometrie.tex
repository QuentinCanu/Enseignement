\documentclass{article}
\usepackage{mainPoly}

\title{Fonctions Trigonométriques}
\date{}
\author{Premières Spécialité Mathématiques}

\begin{document}
\maketitle

\section{Cercle trigonométrique : mesure en radians}
On se munit d'un repère orthonormé $(O;I;J)$. Le cercle \textbf{trigonométrique} est le cercle de centre $O$, et de rayon $1$.

\begin{center}
\begin{tikzpicture}
\draw[->] (-3,0) -- (3,0);
\draw[->] (0,-3) -- (0,3);
\draw (0,0) node[above left] {$O$};
\draw (1.5,0) node[below right] {$I$};
\draw (0,1.5) node[above left] {$J$};
\draw (0,0) circle (1.5);
\end{tikzpicture}
\end{center}
On va considérer un point $M$ qui circule le long de ce cercle. Son point de départ est $I$, et le sens de parcours sera le sens anti-horaire.

\begin{center}
\begin{tikzpicture}
\draw[->] (-3,0) -- (3,0);
\draw[->] (0,-3) -- (0,3);
\draw (0,0) node[above left] {$O$};
\draw (1.5,0) node[below right] {$I$};
\draw (0,1.5) node[above left] {$J$};
\draw (0,0) circle (1.5);
\draw (45:1.5) node {$\bullet$} node[above right] {$M$};
\path[->] (15:3) edge[bend right] node[right] {Sens de parcours} (75:3);
\end{tikzpicture}
\end{center}

\begin{tcolorbox}
\begin{definition}
Le sens anti-horaire est appelé \textbf{sens direct} ou \textbf{sens trigonométrique}.
\end{definition}
\end{tcolorbox}
\end{document}