\documentclass{exam}
\usepackage{mainExam}

\title{Produit Scalaire}
\author{Première Spécialité Mathématiques}
\date{4 Septembre 2024}

\begin{document}
\maketitle
\begin{questions}
\question Soit $\vect{u}$ et $\vect{v}$ tels que $\norm{\vect{u}} = 2$, $\norm{\vect{v}} = 3$ et $\widehat{\vect{u},\vect{v}} = 60°$. Calculer le produit scalaire $\vect{u} \cdot \vect{v}$.
\vspace*{0.5cm}

\question Soit $\vect{p}$ et $\vect{q}$ tels que $\norm{\vect{p}} = 5$, $\norm{\vect{q}} = \sqrt{3}$ et $\widehat{\vect{p},\vect{q}} = 135°$. Calculer le produit scalaire $\vect{p} \cdot \vect{q}$.
\vspace*{0.5cm}

\question Soit $ABC$ un triangle équilatéral de côté $5$. Calculer le produit scalaire $\vect{AB} \cdot \vect{AC}$.
\vspace*{0.5cm}

\question Soit $ABCD$ un carré de côté $5$. Calculer le produit scalaire $\vect{AB} \cdot \vect{AC}$.
\vspace*{0.5cm}

\question Soit $ABCD$ un carré de côté $4$ et de centre $O$. On place les points les milieux $I$, $J$, $K$ et $L$ respectifs des segments $[AB]$, $[BC]$, $[CD]$ et $[DA]$. Calculer les produits scalaires suivants:
\begin{parts}
\part $\vect{CO} \cdot \vect{CK}$
\part $\vect{CJ} \cdot \vect{LJ}$
\part $\vect{IJ} \cdot \vect{KL}$
\part $\vect{IJ} \cdot \vect{IL}$
\end{parts} 
\end{questions}
\end{document}