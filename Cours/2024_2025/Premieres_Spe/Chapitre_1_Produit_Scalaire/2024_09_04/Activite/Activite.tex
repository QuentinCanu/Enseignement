\documentclass{article}
\usepackage{main}

\title{Théorème de Pythagore \og général \fg}
\date{4 Septembre 2024}
\author{Première Spécialité Mathématiques}

\begin{document}
\maketitle

On pose un triangle $ABC$ quelconque. Si le triangle est rectangle en $A$, alors le théorème de Pythagore nous assure que $BC^2 = AB^2 + AC^2$. On se pose la question de la valeur $P$ définie par
\begin{equation*}
P = \dfrac{1}{2}\left(AB^2 + AC^2 - BC^2\right)
\end{equation*}
dans le cas où $ABC$ est quelconque. On suppose pour commencer que l'angle $\widehat{BAC}$ est aigu ($\alpha < 90°$), comme représenté ci-dessous.

\begin{center}
\begin{tikzpicture}
\coordinate (A) at (0,0);
\coordinate (B) at (5,0);
\coordinate (C) at (2,3);

\draw (A) node[below left] {$A$} -- (B) node[right] {$B$} -- (C) node[above] {$C$} -- cycle;
\draw (A) ++ (0.5,0) arc (0:55:0.5) node[midway, above right] {$\alpha$};
\end{tikzpicture}
\end{center}
\begin{enumerate}[label=\emph{\arabic*)}]
\item Tracer le cas où l'angle $\widehat{BAC}$ est obtu ($\alpha > 90°$).
\item Rappeler la définition de projeté orthogonal d'un point sur une droite. Placer le point $H$, projeté orthogonal de $C$ sur la droite $(AB)$.
\item Démontrer que 
\begin{equation*}
P = \dfrac{1}{2}\left(AH^2 + AB^2 - BH^2\right)
\end{equation*}
\item En déduire que $P = AB \times AH$. \emph{(Indication : écrire $BH$ en fonction de $AH$ et de $AB$)}
\item En déduire que $P = AB \times AC \times \cos(\alpha)$.
\item Que dire du cas où l'angle $\widehat{BAC}$ est obtu ?
\end{enumerate}
\end{document}