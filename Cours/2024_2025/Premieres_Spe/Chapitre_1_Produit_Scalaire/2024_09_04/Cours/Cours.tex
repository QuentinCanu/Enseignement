\documentclass{article}
\usepackage{main}

\title{Cours : Accueil, Produit Scalaire}
\date{4 Septembre 2024}
\author{Quentin Canu}

\begin{document}
\maketitle

\section{Accueil des élèves (15 min)}
\subsection{Matériel demandé}
\begin{itemize}
\item Porte-vue (ou classeur) pour les cours.
\item Cahier pour les exercices. 
\end{itemize}
\subsection{Notes}
Les notes comptent pour le contrôle continu du bac (coefficient $8$ si la spécialité est abandonnée à la fin de la première).
\begin{itemize}
\item On fait une évaluation de cours par semaine, avec une note sur $10$.
\item On fait un contrôle une fois toutes les trois semaines. Note sur $20$ avec coefficient $2$ ou $3$.
\end{itemize}
\subsection{Règles en classe}
\begin{itemize}
\item Une personne a la parole à la fois.
\item Il n'y a pas de question idiote. 
\item Créneau de deux heures : on reste en salle pendant la pause.
\end{itemize}
\section{Rappels (5-10 min)}
Conversation sur les vecteurs.

\section{Activité (30-40 min)}
\section{Cours (15 min)}
\section{Exercices (40 min)}
\section{Devoirs}

Finir la feuille d'exercices.

\end{document}