\documentclass{article}
\usepackage{mainPoly}

\title{Chapitre 2 : Second degré}
\date{}
\author{Premières Spécialité Mathématiques}

\begin{document}
\maketitle

\section{Fonctions polynomiales du second degré}

\begin{tcolorbox}
\begin{definition}
Une \textbf{fonction polynomiale du second degré} est une fonction $f$ définie sur les réels qui à tout nombre $x$ associe un réel $f(x)$ de la forme:
\begin{equation*}
ax^2+bx+c   
\end{equation*}
où $a$, $b$ et $c$ sont des réels avec $a \neq 0$. 
\end{definition}
\end{tcolorbox}
\begin{remark}
L'hypothèse $a \neq 0$ est essentielle, sinon la fonction est polynomiale de degré au plus $1$.    
\end{remark}
L'objectif de ce chapitre est d'étudier les fonctions polynomiales du second degré : l'allure de leur courbe représentative, leur extremum, leurs racines\dots
\subsection{Forme canonique}
\begin{tcolorbox}
\begin{proposition}
Soit $f$ une fonction polynomiale du second degré telle que $f(x) = ax^2 + bx + c$. Alors il existe $\alpha$ et $\beta$ tel que
\begin{equation*}
f(x) = a(x - \alpha)^2 + \beta
\end{equation*}
\end{proposition}
\end{tcolorbox}
\emptybox{10cm}
\end{document}