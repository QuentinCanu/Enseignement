\documentclass{article}
\usepackage{main}

\title{Secondes Mathématiques : Progression}
\author{Quentin Canu}
\date{2024-2025}

\begin{document}
\maketitle

\section{Informations}

\qty{4}{\hour} par semaine $\times$ 36 semaines = \qty{144}{\hour}

\section{Progression}

\begin{center}
    
\begin{tabular}{|l|p{4cm}|p{2cm}|p{2cm}|}
\hline
Chapitre
    & 
Titre
        & 
Nombre de semaines
            &
Calcul en demi-groupe\\
\hline
1   & 
Calcul littéral : Développement, Factorisation, identités remarquables, expressions fractionnaires
        &
            &
\\
\hline
2   &
Vecteurs : Translation, sommes, différence, colinéarité 
        &

            &
Fractions\\
\hline
3   &
Proportions, pourcentages, évolutions
        &

            &
Équations du premier degré\\
\hline
4   &
Ensembles de nombres, arithmétique, intervalles, valeurs absolues
        &

            &
Puissances\\
\hline
5   &
Géométrie repérée, coordonnées du milieu, distance
        &

            &
Racines carrés\\
\hline
6   &
Fonctions : Généralités, courbe représentative, tableau de signes
        &

            &
Inéquations\\
\hline
7   &
Probabilités
        &

            &
Géométrie simple\\
\hline
8   &
Coordonnées de vecteurs, déterminant
        &

            &
Équation produit\\
\hline
9   &
Etudes de fonctions, variations, signe, extremum, parité
        &

            &
Fonctions de référence\\
\hline
10  &
Statistiques descriptives
        &

            &
Systèmes\\
\hline
11  &
Equations de droites, focntions affines,vecteurs directeurs
        &

            &
\\
\hline
12  &
Échantillonage
        &

            &
\\
\hline
\end{tabular}
\end{center}


\end{document}