\documentclass{exam}
\usepackage{mainExam}

\title{Évaluation de cours}
\author{Seconde 9}
\date{10 Février 2024}


\begin{document}
\maketitle
\begin{questions}
\titledquestion{Puissances}[5] Simplifier les expressions suivantes sous la forme $a^n$.
\begin{parts}
\vspace*{0.2cm}
\part[1] $4^3 \times 4^5 =$ \answersline
\vspace*{0.2cm}
\part[1] $\dfrac{2^3 \times 8}{2^3} =$ \answersline
\vspace*{0.2cm}
\part[1] $\dfrac{(3^6)^2}{3^2 \times 3^7} = $ \answersline
\vspace*{0.2cm}
\part[1] $\dfrac{5 \times 5^2}{25} = $ \answersline
\vspace*{0.2cm}
\part[1] $\dfrac{b^7 \times b^3}{b^{-4}} = $ \answersline
\end{parts}
\vspace*{0.5cm}
\titledquestion{Intervalles}[3]
Pour chacune des inégalités suivantes, dire à quel intervalle appartient $x$ :
\begin{parts}
\part[1] $12 \leq x < 14$ : $x \in $ \answersline
\vspace*{0.2cm}
\part[1] $-6 < x < -3,5$ : $x \in $ \answersline
\vspace*{0.2cm}
\part[1] $25 \leq x$ : $x \in $ \answersline
\vspace*{0.2cm}
\end{parts}
\vspace*{0.5cm}
\titledquestion{Distance}[2]
\begin{parts}
\part[1] Rappeler la formule de la distance entre deux nombres réels $a$ et $b$ : \answersline
\vspace*{0.2cm}
\part[1] Donner deux nombres réels dont la distance vaut $9$ : \answersline
\end{parts}
\end{questions}
\end{document}