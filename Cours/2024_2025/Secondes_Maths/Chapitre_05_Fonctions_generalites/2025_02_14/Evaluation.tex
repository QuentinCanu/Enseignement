\documentclass{exam}
\usepackage{mainExam}

\title{Évaluation de cours}
\author{Seconde 9}
\date{14 Février 2024}


\begin{document}
\maketitle
\begin{questions}
\titledquestion{Puissances}[3] Simplifier les expressions suivantes sous la forme $a^n$.
\begin{parts}
\vspace*{0.2cm}
\part[1] $4^3 \times 4^5 =$ \answersline
\vspace*{0.2cm}
\part[1] $\dfrac{(3^6)^2}{3^2 \times 3^7} = $ \answersline
\vspace*{0.2cm}
\part[1] $\dfrac{b^7 \times b^3}{b^{-4}} = $ \answersline
\end{parts}
\vspace*{0.5cm}
\titledquestion{Intervalles}[4]
Pour chacune des inégalités suivantes, dire à quel intervalle appartient $x$ :
\begin{parts}
\part[1] $12 \leq x < 14$ : $x \in $ \answersline
\vspace*{0.2cm}
\part[1] $x < 8$ : $x \in$ \answersline
\vspace*{0.2cm}
\part[1] $-6 < x < -3,5$ : $x \in $ \answersline
\vspace*{0.2cm}
\part[1] $25 \leq x$ : $x \in $ \answersline
\vspace*{0.2cm}
\end{parts}
\vspace*{0.5cm}
\titledquestion{Fonctions}[3]
Pour chacune des fonctions suivantes, compléter les phrases réponses demandées.
\begin{parts}
\part[1] $\function{f}{\R}{\R}{x}{5x-6}$ :
\begin{itemize}
\item L'image de $3$ par $f$ est \answersline 
\item L'image de $-5$ par $f$ est \answersline
\end{itemize}
\vspace*{0.2cm}
\part[1] $\function{g}{\R}{\R}{x}{x^2 + 6}$
\begin{itemize}
\item L'image de $4$ par $g$ est \answersline
\item On en déduit qu'un antécédent de \answersline par $g$ est \answersline
\end{itemize}
\vspace*{0.2cm}
\part[1] $\function{h}{\R}{\R}{x}{\abs{3x}}$
\begin{itemize}
\item L'image de $-8$ par $h$ est \answersline
\item On en déduit qu'un antécédent de \answersline par $h$ est \answersline
\end{itemize}
\vspace*{0.2cm}
\end{parts}
\end{questions}
\end{document}