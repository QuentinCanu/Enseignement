\documentclass{exam}
\usepackage{mainExam}

\title{Évaluation de cours}
\author{Seconde 9}
\date{8 Novembre 2024}

\begin{document}
\maketitle

\begin{questions}
\question
Pour chacune des questions suivantes, donner la population et la sous-population à l'étude.
\begin{parts}
\part Dans une trousse à outils, il y a $51$ vis cruciformes, qui représentent $30\,\%$ du nombre total de vis.

Quel est le nombre total de vis dans la trousse à outils ?
\part Le cadeau commun que nous souhaitons faire à Karim coûte $60$\,\euro{}. Je participe à hauteur de $12$\,\euro{}.

Calculer la proportion en pourcentage de ma participation sur le prix total du cadeau.
\part Une fête forraine accueille $1\,700$ visiteurs. On dénombre $40~\%$ de visiteurs de moins de $18$ ans.

Quel est le nombre de mineurs ?
\part Lors d'un concert, il y a $86$ spectateurs de plus de $60$ ans, ce qui représente $10~\%$ du public.

Combien de spectateurs ont assisté au concert ?
\part Un mur en briques est constitué de $1\,450$ briques. On dénombre $20~\%$ de briques rouges.

Quel est le nombre de briques rouges ?
\end{parts}
\end{questions}

\newpage
\maketitle
\begin{questions}
\question
Pour chacune des questions suivantes, donner la population et la sous-population à l'étude.
\begin{parts}
\part Une réserve de protection de tigres contient $1\,200$ tigres. On dénombre $240$ tigres blancs. 

Calculer la proportion en pourcentage de tigres blancs dans la réserve.
\part Dans une collection de $160$ cartes à jouer, il y a  $10\,\%$ de cartes pokémon. 

Combien y a-t-il de cartes pokémon?

\part Lors d'un concert, il y a $615$ spectateurs de plus de $60$ ans, ce qui représente $30~\%$ du public.

Combien de spectateurs ont assisté au concert ?
\part Le cadeau commun que nous souhaitons faire à Karim coûte $150$\,\euro{}. Je participe à hauteur de $51$\,\euro{}.

Calculer la proportion en pourcentage de ma participation sur le prix total du cadeau.

\part $340$ habitants d'une ville ont voté pour un maire. $55~\%$ ont voté pour le maire sortant. 

Calculer le nombre de votes du maire sortant.
\end{parts}
\end{questions}
\end{document}