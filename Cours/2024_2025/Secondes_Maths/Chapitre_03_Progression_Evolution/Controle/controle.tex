\documentclass{exam}
\usepackage{mainExam}

\title{Contrôle : proportions, évolutions, équations}
\date{6 Décembre 2024}
\author{Seconde 9}

\begin{document}
\maketitle
\instructions{autorisée}

\begin{questions}
\titledquestion{Équations}[4] Résoudre les équations suivantes (aucune phrase de justification n'est nécessaire).
\begin{parts}
\part $3x - 2 = 0$
\part $5y + 4 = 12$
\part $4t - 11 = -3$
\part $-6b + 13 = -11$
\part $11z + 4 = 13z + 8$
\part $-5w - 2 = 3w + 8$
\end{parts}
\vspace*{0.5cm}

\titledquestion{Proportions et sport}[5]
\begin{parts}
\part Soit $P$ une population de $n$ individus et $S$ une sous-population de $P$ comportant $s$ individus. Rappeler la formule donnant la proportion de $S$ par rapport à $P$.
\part Un archer pratique son sport en vue des jeux olympiques. 
\begin{subparts}
\subpart Lors de son entraînement, il tire $34$ flèches. Seules $13$ flèches touchent leur cible. Donner le pourcentage de réussite de cet archer.
\subpart Combien de flèches doivent toucher la cible durant le \textsc{même} entrainement ($34$ flèches tirées), pour que le taux de réussite soit supérieur à $75\%$ ? 
\end{subparts}
\part Une lanceuse de poids pratique son sport en vue du championnat mondial.
\begin{subparts}
\subpart Après avec lancé $350$ poids durant la saison, seuls $46\%$ ont dépassé 20 mètres. Combien de poids ont dépassé 20 mètres ?
\subpart Cette lanceuse conserve son taux de réussite à $46\%$ lors de la saison suivante. Après avoir lancé tous ses poids, seuls $391$ ont dépassé les 20 mètres. Combien de poids a-t-elle lancée durant la saison ?
\end{subparts}
\end{parts}
\vspace*{0.5cm}

\titledquestion{Proportions de proportions}[5]
\begin{parts}
\part Soit $P$ une population, $S$ une sous-population de $P$ et $T$ une sous-population de $S$. On note:
\begin{itemize}
\item $p_S$ la proportion de $S$ par rapport à $P$
\item $p_T$ la proportion de $T$ par rapport à $S$
\item $p$ la proportion de $T$ par rapport à $T$
\end{itemize}

Rappeler la formule permettant de calculer $p$ en fonction de $p_S$ et $p_T$.
\part On s'interresse aux contrats d'une agence représentant des artistes musicaux. 
\begin{subparts}
\subpart $33\%$ des artistes représentés font du rap, et parmi eux, $12\%$ font du rap gangsta. Quelle est la proportion de rappeurs gangsta représentés par l'agence, parmi \textsc{tous} les artistes représentés ?
\subpart Même question sur les chanteurs de black metal. En effet, $27\%$ des artistes suivis par cette même maison de disque font du metal, et  parmi eux, $18\%$ d'entre eux sont des chanteurs de black metal.
\subpart Cette agence représente-t-elle le plus des rappeurs gangsta ou des chanteurs de black metal ?  
\end{subparts}
\part En plus des artistes de rap et des chanteurs de metal, cette agence représente des artistes de pop. Certains de ces artistes de pop en particulier sont des artistes de pop indie. Ces artistes-là représentent $20\%$ des clients de cette agence, par rapport à l'intégralité des artistes représentés. Quel est le pourcentage des artistes pop indie représentés parmi les artistes pop représentés ?
\end{parts}
\newpage

\titledquestion{Évolutions}[6]
\begin{parts}
\part Soit une quantité évoluant d'une valeur de départ $V_d$ vers une valeur finale $V_f$. Donner la formule du taux d'évolution $t$ de cette quantité.
\vspace*{0.5cm}

On joue à un jeu de rôle. La puissance de notre personnage est actuellement de $1246$. Heureusement, il est possible d'acheter un bonus parmi trois pour s'améliorer. 
\begin{enumerate}
\item Le premier bonus augmente la puissance de $300$. 
\item Le deuxième bonus permet d'augmenter la puissance de $15\%$.
\item Le troisième bonus est aussi une augmentation en pourcentage. Par exemple, durant la dernière partie, ce bonus a augmenté une puissance de $650$ en $767$.
\end{enumerate}
\part Quel est le meilleur bonus disponible pour notre personnage ?
\part Même question pour un personnage dont le niveau de puissance de départ est $1945$.
\part Quel est le niveau de puissance de départ nécessaire pour que le premier et le deuxième bonus aie le même effet ?
\end{parts}
\end{questions}
\end{document}