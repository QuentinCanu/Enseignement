\documentclass{exam}
\usepackage{mainExam}

\title{Évaluation de cours}
\date{22 Novembre 2024}
\author{Seconde 9}

\begin{document}
\maketitle

\instructions{autorisée}

\begin{questions}
\question Dans chacune des situations suivantes :
\begin{itemize}
\item \textbf{Expliciter la population, la sous-population et la sous-sous-population}
\item \textbf{Répondre à la question en écrivant clairement le calcul demandé}
\end{itemize}
\begin{parts}
\part Dans un club de boxe, $67\%$ des membres sont dans la catégorie poids lourd. Parmi eux, $76\%$ ont déjà gagné un combat durant la saison. Quelle est la proportion des membres de catégorie poids lourd ayant déjà gagné un combat parmi les membres du club ?
\part Un gâteau comprend dans sa recette $27\%$ de fruits secs. De plus, ce gâteau contient $5\%$ de noisettes. Quel est la proportion de noisettes parmi les fruits secs du gâteau ?
\part $49\%$ du budget d'un pays est consacrée à la culture, et $77\%$ du budget de la culture est dédié au financement des musées. Quel pourcentage du budget du pays finance les musées ? 
\end{parts}

\makeemptybox{5cm}

\question Résoudre les équations suivantes :
\begin{parts}
\item $2x - 14 = 0$
\item $17t + 6 = 0$
\item $12z + 18 = 11 - 18z$
\end{parts}

\makeemptybox{5cm}
\end{questions}

\newpage

\maketitle

\instructions{autorisée}

\begin{questions}
\question Dans chacune des situations suivantes :
\begin{itemize}
\item \textbf{Expliciter la population, la sous-population et la sous-sous-population}
\item \textbf{Répondre à la question en écrivant clairement le calcul demandé}
\end{itemize}
\begin{parts}
\part Dans un jardin botanique, $88\%$ des espèces sont des fleurs. Parmi ces fleurs, $23\%$ sont des roses. Quelle est la proportion de roses dans ce jardin ?
\part Dans le trésor d'un pirate, $36\%$ est constitué de pierres précieuses. De plus, le trésor contient $19\%$ de rubis (qui sont des pierres précieuses). Quel est le pourcentage de rubis dans ce trésor ?
\part $19\%$ des bénéfices d'une entreprises est consacrée à des œuvres caritatives. Par exemple, $8\%$ des bénéfices de cette entreprise est versée à la croix rouge. Quel pourcentage du budget dédié aux œuvres caritatives est versée à la croix rouge ?
\end{parts}

\makeemptybox{5cm}

\question Résoudre les équations suivantes :
\begin{parts}
\item $5y - 15 = 0$
\item $11 + 9u = 0$
\item $-6p + 9 = 4p - 1$
\end{parts}

\makeemptybox{5cm}
\end{questions}

\end{document}