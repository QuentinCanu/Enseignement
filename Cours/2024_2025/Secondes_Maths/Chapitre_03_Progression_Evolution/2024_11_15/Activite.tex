\documentclass{article}
\usepackage{main}

\title{Activité : Rentabilité de films}
\author{Seconde 9}
\date{5 Novembre 2024}

\begin{document}
\maketitle

Le tableau suivant présente le budget et les revenus mondiaux d'une vingtaine de films.
\begin{table}[h]
    \centering
    \begin{tabular}{|c|p{2cm}|p{2cm}|p{2cm}|}
    \hline
    Titre & Budget (en millions) & Revenus mondiaux (en millions) & \% du budget récupéré \\ \hline
    300 & 65 & 456 &  \\ \hline
    Ant-Man & 130 & 519 &  \\ \hline
    Avatar & 237 & 2923 & \\ \hline
    Captain America: The First Avenger & 140 & 371 & \\ \hline
    Dragon Ball Super: Broly & 8,5 & 103,6 & \\ \hline
    Inception & 160 & 837 & \\ \hline
    Indiana Jones \& the Last Crusade & 48 & 474 &  \\ \hline
    Jumanji: Welcome to the Jungle & 90 & 963 & 1069 \\ \hline
    La La Land & 30 & 471 & \\ \hline
    Lucy & 39 & 469 & \\ \hline
    Mad Max: Fury Road & 185 & 390 & \\ \hline
    Paranormal Activity & 0,015 & 194 & \\ \hline
    Parasite & 11 & 117,3 & \\ \hline
    Pirates of the Caribbean: On Stranger Tides & 379 & 1046 & \\ \hline
    Pokémon Detective Pikachu & 150 & 433 & \\ \hline
    Skyfall & 200 & 1109 & \\ \hline
    Sleight & 0,25 & 4 & \\ \hline
    Solo: A Star Wars Story & 300 & 393 & \\ \hline
    The Dark Knight & 185 & 1006 &  \\ \hline
    Toy Story & 30 & 394 &  \\ \hline
    \end{tabular}
    \caption{Tableau de rentabilité de 20 films.
    }
    \label{tab:my-table}
    \end{table}

\begin{enumquestions}
\item Par quelle formule déterminer les gains réels (c'est-à-dire une fois que les coûts ont été payés) d'un film ? Donner les gains réels de  
\end{enumquestions}
\end{document}