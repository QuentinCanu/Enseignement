\documentclass{article}
\usepackage{main}

\title{Calcul de proportions}

\begin{document}
\maketitle

\begin{enumquestions}
    \item Dans une entreprise, il y a $300$ salariés au total. Parmi eux, on dénombre  $120$ cadres. \\Calculer la proportion en pourcentage de cadres dans cette entreprise.
	\item Dans une réserve de protection d'oiseaux, il y a $198$ pipits farlouse, ce qui représente $22~\%$ du nombre total d'oiseaux. \\Quel est le nombre d'oiseaux de cette réserve ?
	\item $400$ personnes assistent à un concert. $36~\%$ ont moins de $18$ ans. \\Calculer le nombre de personnes majeures dans le public.
	\item Le cadeau commun que nous souhaitons faire à Rayan coûte $40$\,\euro{}. Je participe à hauteur de $8$\,\euro{}. \\Calculer la proportion en pourcentage de ma participation sur le prix total du cadeau.
	\item Dans un entreprise, il y a  $39$ cadres. Ils  représentent $30\,\%$ du nombre total de salariés. \\Quel est le nombre total de salariés dans cette entreprise ?
	\item Une réserve de protection d'oiseaux contient $2\,600$ individus d'oiseaux. On dénombre $28~\%$ de pipits farlouse.\\Quel est le nombre de pipits farlouse ?
	\item Pour le cadeau de Mehdi, j'ai donné $16$\,\euro{}. Cela représente $32~\%$ du prix total du cadeau. \\Quel est le montant du cadeau ?
	\item $1\,650$ personnes assistent à un concert. $54~\%$ ont moins de $18$ ans. \\Calculer le nombre de personnes mineures dans le public.
	\item Une réserve de protection d'oiseaux contient $1\,480$ individus d'oiseaux. On dénombre $74$ bruants des roseaux. \\Calculer la proportion en pourcentage de bruants des roseaux dans la réserve.
	\item Dans une entreprise de $150$ salariés, il y a  $28\,\%$ de cadres. \\Combien y a-t-il de cadres dans cette entreprise ?
\end{enumquestions}
\end{document}