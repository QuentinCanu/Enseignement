\documentclass{article}
\usepackage{mainPoly}

\title{Statistiques : Proportions, Évolutions}
\date{}
\author{Seconde 9}

\begin{document}
\maketitle

\section{Proportions et pourcentages}
\subsection{Populations}
\begin{tcolorbox}
\begin{definition}
En statistiques, on étudie des \textbf{populations}, c'est-à-dire des ensembles d'éléments appelés \textbf{individus}.  
\end{definition}
\end{tcolorbox}
\begin{example} 
Les ensembles suivants sont des populations pouvant faire l'objet d'études statistiques. 
\begin{itemize}
\item Le sport préféré des habitants de Villeneuve-Le-Roi;
\item Les initiales des élèves d'un lycée;
\item Le poids de pièces de métal fabriquées par une machine.
\item
\item 
\end{itemize}
\end{example}
\begin{tcolorbox}
\begin{definition}
On appelle \textbf{sous-population} d'une population $P$ une partie des individus de $P$.
\end{definition}
\end{tcolorbox}
\begin{example}
On donne des exemples de sous-population correspondant aux populations données ci-dessus:
\begin{itemize}
\item Les sports collectifs;
\item Les initiales commençant par des voyelles;
\item Les pièces pesant plus de $\qty{3,8}{\kilo\gram}$;
\item 
\item 
\end{itemize}
\end{example}
\begin{tcolorbox}
\begin{definition}
On conidère une population $P$ de $N$ individus et une sous-population $S$ de $P$ de $n$ individus. Alors la \textbf{proportion} de $S$ par rapport à $P$, notée $p$, est donné par
\begin{equation*}
p = \dfrac{n}{N}
\end{equation*}        
\end{definition}
\end{tcolorbox}
\begin{remark}
Pour obtenir la proportion d'une sous-population, on divise le nombre d'individus \textbf{concernés} par le nombre \textbf{total} d'individus. 
\end{remark}
\begin{example}
On vide une trousse de tous ses stylos (il y en a $15$), et on compte le nombre de stylos rouges (il y en a $3$). 
\begin{enumquestions}
\item Quelle est la population étudiée ? Et la sous-population ?
\item Quelle est la proportion de stylos rouges dans cette trousse ?
\end{enumquestions}

\emptybox{4cm}
\end{example}
\newpage

\subsection{Pourcentages}
\begin{tcolorbox}
\begin{remark}
Si l'on souhaite avoir la proportion $p$ sous la forme de \textbf{pourcentage}, il suffit de la multiplier par $100$.
\end{remark}
\end{tcolorbox}
\begin{example}
On considère les $56$ animaux d'un zoo : il y a $28$ lions, $12$ zèbres et $16$ alligators.
\begin{enumquestions}
\item Quelle est la population étudiée ?
\item Quelles sont les différentes sous-populations à l'étude ?
\item Donner la proportion de lions ($p_L$), de zèbres ($p_Z$) et d'alligators ($p_A$) \textbf{en pourcentage}.
\end{enumquestions}

\emptybox{4cm}
\end{example}

\begin{remark}
\hfill
\begin{itemize}
\item Si l'on connait la nombre total d'individus $N$ et la proportion $p$ de la sous-population $S$, alors on obtient le nombre d'individus $n$ de $S$ en faisant
\begin{equation*}
n = p \times N
\end{equation*}.
\item Autrement dit, prendre $p \%$ de $N$, c'est multiplier $N$ par $\dfrac{p}{100}$.
\item Si l'on connait $n$ et $p$, alors le nombre total d'individu $N$ est donné par
\begin{equation*}
N = \dfrac{n}{p}
\end{equation*}
\item Autrement dit, si $n$ représente $p \%$ de la population totale, alors le nombre total d'individu est donné par
\begin{equation*}
N = \dfrac{n}{p}100
\end{equation*}
\end{itemize}
\end{remark}
\begin{example}
\hfill
\begin{enumquestions}
\item Dans le lycée $A$, il y a $650$ élèves, dont $20\%$ de secondes. Combien y a-t-il de secondes ?
\item Il y a $50$ terminales dans le lycée $B$, et ils représentent $25\%$  de l'ensemble des élèves. Combien y a-t-il d'élèves au total dans le lycée $B$ ?
\end{enumquestions}

\emptybox{4cm}
\end{example}
\end{document}