\documentclass{article}
\usepackage{main}

\title{Activité : Rentabilité de films}
\author{Seconde 9}
\date{5 Novembre 2024}

\begin{document}
\maketitle

Le tableau suivant présente le budget et les revenus mondiaux d'une vingtaine de films.
\begin{table}[h]
\centering
\begin{tabular}{|c|p{2cm}|p{2cm}|p{2cm}|p{2cm}|}
\hline
Titre & Budget (en millions) & Revenus mondiaux (en millions) & Gains réels (en millions) & Rentabilité ($\%$ du budget récupéré) \\ \hline
300 & 65 & 456 &  &\\ \hline
Ant-Man & 130 & 519 & & \\ \hline
Avatar & 237 & 2923 & & \\ \hline
Captain America: The First Avenger & 140 & 371 &  & \\ \hline
Dragon Ball Super: Broly & 8,5 & 103,6 & & \\ \hline
Inception & 160 & 837 & & \\ \hline
Indiana Jones \& the Last Crusade & 48 & 474 &  &\\ \hline
Jumanji: Welcome to the Jungle & 90 & 963 & & \\ \hline
La La Land & 30 & 471 & & \\ \hline
Lucy & 39 & 469 & & \\ \hline
Mad Max: Fury Road & 185 & 390 & &\\ \hline
Paranormal Activity & 0,015 & 194 & &\\ \hline
Parasite & 11 & 117,3 & &\\ \hline
Pirates of the Caribbean: On Stranger Tides & 379 & 1046 & &\\ \hline
Pokémon Detective Pikachu & 150 & 433 & &\\ \hline
Skyfall & 200 & 1109 & &\\ \hline
Sleight & 0,25 & 4 & &\\ \hline
Solo: A Star Wars Story & 300 & 393 & &\\ \hline
The Dark Knight & 185 & 1006 &  &\\ \hline
Toy Story & 30 & 394 &  &\\ \hline
\end{tabular}
\caption{Tableau de rentabilité de 20 films.
}
\label{tab:my-table}
\end{table}

\begin{enumquestions}
\item Par quelle formule déterminer les gains réels d'un film, c'est-à-dire ce qu'a rapporté le film \textbf{une fois le budget dépensé} ?
\item En déduire les gains réels de tous les films présentés ici.
\item Quel ratio effectuer pour obtenir la rentabilité d'un film ?
\item Calculer la rentabilité de tous les films du tableau.
\item D'après l'article donné par le lien suivant \url{https://lucyfinance.fr/la-rentabilite-dun-film/}, quels sont les types de budget possibles pour un film ? Et les types de recettes possibles ?
\end{enumquestions}

\emptybox{5cm}
\end{document}