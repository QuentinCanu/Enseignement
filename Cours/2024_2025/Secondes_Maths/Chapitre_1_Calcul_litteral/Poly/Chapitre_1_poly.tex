\documentclass{article}
\usepackage{mainPoly}

\author{}
\date{}
\title{Calcul Littéral}

\begin{document}
\maketitle
\section{Introduction : Lire un cours de maths}
\begin{tcolorbox}
\begin{definition}
Une définition permet l'introduction d'un concept nouveau en mathématiques. Il utilise des définitions déjà connues pour construire quelque chose de nouveau.
\end{definition}
\end{tcolorbox}
Les définitions décrivent ce que sont les objets.
\begin{proposition}
Une proposition est un résultat à propos des objets introduits par le cours. La proposition est vraie parce qu'elle a été démontrée.
\end{proposition}
Les propositions décrivent ce que font les objets.
\begin{example}
Le nombre $\pi$ est défini comme la rapport entre le périmètre et le diamètre de n'importe quel cercle. C'est une \textbf{définition}.

L'aire d'un disque de rayon $r$ est donné par $\pi r^2$. C'est une \textbf{proposition}: c'est un résultat que l'on démontre grâce à la définition de $\pi$.
\end{example}
\begin{remark}
Pour bien apprendre un cours de maths, il faut identifier les différentes parties du cours: 
\begin{itemize}
\item Les \textbf{définitions} sont à connaître par \textbf{cœur}.
\item Les \textbf{propositions} sont à comprendre. Pour cela, il faut savoir \textbf{refaire} les \textbf{exemples} données par le cours.
\item Les \textbf{théorèmes} sont des proposition importantes, elle nécessitent d'être connues.
\item Les \textbf{remarques} permettent de mieux comprendre les concepts du cours, il ne faut pas les négliger lors de la lecture du cours.
\item Les \textbf{Exemples} illustrent directement les notions introduites. Il faut savoir les \textbf{refaire}.
\end{itemize}
\end{remark}
\newpage
\section{Développement et factorisation}
\begin{tcolorbox}
\begin{definition}
\begin{itemize}
\item Une expression littérale est sous forme \emph{développée} si elle correspond à une \textbf{somme} de termes.
\item Une expression littérale est sous forme \emph{factorisée} si elle correspond à un \textbf{produit} de facteurs.
\end{itemize}
\end{definition}
\end{tcolorbox}
\begin{example}
L'aire du rectangle suivant
\begin{center}
\begin{tikzpicture}
\coordinate (A) at (0,0);
\coordinate (B) at (5,0);
\coordinate (C) at (5,2);
\coordinate (D) at (0,2);
\coordinate (E) at (2,0);
\coordinate (F) at (2,2);

\draw (A) -- (B) -- (C) -- (D) -- cycle;
\draw (E) -- (F);
\draw[<->] (A) ++ (-0.25,0) -- ++(0,2) node[midway,left] {$a$};
\draw[<->] (A) ++ (0,-0.25) -- ++(2,0) node[midway, below] {$b$};
\draw[<->] (E) ++ (0,-0.25) -- ++(3,0) node[midway,below] {$c$};
\end{tikzpicture}
\end{center}
peut être calculée de deux façons.
\begin{itemize}
\item En \textbf{multipliant} sa largeur ($a$) et sa longueur ($b + c$):
\begin{equation*}
a(b + c)    
\end{equation*}
\item En \textbf{ajoutant} les aires des deux rectangles:
\begin{equation*}
ab + bc
\end{equation*}  
\end{itemize}
\end{example}
\subsection{Développement}
Pour développer un produit, on utilise la distributivité de la multiplication sur l'addition.


\begin{equation*}
\tikzmarknode{a}{a}(\tikzmarknode{b}{b}+\tikzmarknode{c}{c}) = ab + ac
\end{equation*}
\begin{tikzpicture}[overlay, remember picture]
\draw (a.north) to[bend left=75] (b.north);
\draw (a.north) to[bend left=75] (c.north);
\end{tikzpicture}

\begin{equation*}
(\tikzmarknode{a1}{a} + \tikzmarknode{b1}{b})(\tikzmarknode{c1}{c} + \tikzmarknode{d1}{d}) =
ac + ad + bc + bd
\end{equation*}
\begin{tikzpicture}[overlay, remember picture]
\draw (a1.north) to[bend left=75] (c1.north);
\draw (a1.north) to[bend left=75] (d1.north);
\draw (b1.south) to[bend right=75] (c1.south);
\draw (b1.south) to[bend right=75] (d1.south);
\end{tikzpicture}
\begin{tcolorbox}
Pour développer un produit de sommes, on \og distribue \fg chaque terme de la somme de gauche vers chaque terme de la somme de droite.
\end{tcolorbox}
\begin{example}
Développer chacune des expressions suivantes. On fera apparaître les traits de construction de la distributivité.
\begin{enumerate}[label=\alph*), parsep=0.5cm, topsep=0.5cm]
\item $4x(2y + 5z) = $ \answersline
\item $3x(-10x + 2) = $ \answersline
\item $-(-4a + 2b) = $ \answersline
\item $(17x - 5)(12x + 7) = $ \answersline
\item $(l + L)(l - L) = $ \answersline
\end{enumerate}
\end{example}

\newpage
\subsection{Factorisation}
\begin{tcolorbox}
Pour factoriser une somme, on peut chercher dans chaque terme de la somme un \textbf{facteur commun}.
\end{tcolorbox}
\begin{equation*}
\underline{a}b + \underline{a}c = \underline{a}(b + c)
\end{equation*}
\begin{example}
Factoriser les expressions suivantes :
\begin{enumerate}[label=\emph{\alph*)}]
\item $5a + 10b =$ \answersline
\item $-8y^2 + y =$ \answersline
\item $21x - 28x^2 =$ \answersline
\item $35p - 42q =$ \answersline
\item $x(3x - 2) + 10(3x - 2) =$ \answersline
\end{enumerate}
\end{example}
\section{Identités remarquables}
\begin{proposition}
Soient $a$ et $b$ deux nombre réels quelconques. Alors,
\begin{equation*}
\begin{aligned}
&(a + b)^2 = a^2 + 2ab + b^2\\ 
&(a - b)^2 = a^2 - 2ab + b^2\\ 
&(a + b)(a - b) = a^2 - b^2\\ 
\end{aligned}
\end{equation*}
\end{proposition}
\begin{example}
Développer les expression suivantes:
\begin{enumerate}[label=\emph{\alph*)}]
\item $(c-1)(c+1)=$ \answersline
\item $(x+4)^2=$ \answersline
\item $(x-4)^2=$ \answersline
\end{enumerate}    
\end{example}
\vspace*{0.5cm}
\begin{example}
Factoriser l'expression suivante. 

$y^2 - 64 =$ \answersline
\end{example}
\end{document}