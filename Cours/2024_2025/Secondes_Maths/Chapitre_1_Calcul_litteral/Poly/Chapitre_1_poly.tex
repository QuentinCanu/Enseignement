\documentclass{article}
\usepackage{mainPoly}

\author{}
\date{}
\title{Calcul Littéral}

\begin{document}
\maketitle
\section{Introduction : Lire un cours de maths}
\begin{tcolorbox}
\begin{definition}
Une définition permet l'introduction d'un concept nouveau en mathématiques. Il utilise des définitions déjà connues pour construire quelque chose de nouveau.
\end{definition}
\end{tcolorbox}
Les définitions décrivent ce que sont les objets.
\begin{proposition}
Une proposition est un résultat à propos des objets introduits par le cours. La proposition est vraie parce qu'elle a été démontrée.
\end{proposition}
Les propositions décrivent ce que font les objets.
\begin{example}
Le nombre $\pi$ est défini comme la rapport entre le périmètre et le diamètre de n'importe quel cercle. C'est une \textbf{définition}.

L'aire d'un disque de rayon $r$ est donné par $\pi r^2$. C'est une \textbf{proposition}: c'est un résultat que l'on démontre grâce à la définition de $\pi$.
\end{example}
\begin{remark}
Pour bien apprendre un cours de maths, il faut identifier les différentes parties du cours: 
\begin{itemize}
\item Les \textbf{définitions} sont à connaître par \textbf{cœur}.
\item Les \textbf{propositions} sont à comprendre. Pour cela, il faut savoir \textbf{refaire} les \textbf{exemples} données par le cours.
\item Les \textbf{théorèmes} sont des proposition importantes, elle nécessitent d'être connues.
\item Les \textbf{remarques} permettent de mieux comprendre les concepts du cours, il ne faut pas les négliger lors de la lecture du cours.
\end{itemize}
\end{remark}
\newpage
\section{Développement}
\begin{tcolorbox}
\begin{definition}
Une \emph{expression littérale} permet de représenter une quantité dépendant de nombres dont les valeurs ne sont pas connues.
\end{definition}
\end{tcolorbox}
\begin{example}
L'aire d'un triangle rectangle dont la longueur est donnée par $L$ et la largeur par $l$ correspond à l'expression
\begin{equation*}
lL
\end{equation*}
Si le prix $P$ d'une baguette de pain augmente de $20 \%$, alors le nouveau prix est donné par
\begin{equation*}
1,20P
\end{equation*}
\end{example}
Une expression littérale décrit donc comment calculer une quantité.
\end{document}