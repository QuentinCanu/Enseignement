\documentclass{article}
\usepackage{main}

\author{Seconde 9}
\date{4 Septembre 2024}
\title{Cours : accueil, développement}

\begin{document}
\maketitle

\section{Accueil}
\textbf{Faire l'appel}
\subsection{Matériel}
\begin{itemize}
\item Un porte-vue, ou un classeur, pour ranger d'un côté les polycopiés de cours, de l'autre les énoncés des exercices.
\item Un cahier pour rédiger les exercices.
\item Un cahier de brouillon.
\item Une calculatrice lycée avec un mode examen.
\end{itemize}
\subsection{Règles en classe}
\begin{itemize}
\item Une seule personne à la fois prend la parole en classe.
\item Il n'y a pas de mauvaise intervention en classe.
\end{itemize}
\subsection{Notes}
\begin{itemize}
\item Une evaluation de cours tous les vendredi (début le vendredi 20 Septembre).
\item Un contrôle toutes les trois semaines environ. Coeff $2$ ou $3$.
\end{itemize}
\section{Cours}
\subsection{Comment apprendre un cours de maths}
\begin{itemize}
\item Définitions : par cœur
\item Propopositions, théorèmes : à connaître et à comprendre
\item Exemples : à savoir refaire
\item 
\end{itemize}
\end{document}