\documentclass{article}
\usepackage{main}

\title{Cours : Addition de vecteurs}
\date{26 Septembre 2024}
\author{Quentin Canu}

\begin{document}
\maketitle

\section{Vie de classe}

Il faut qu'on parle de la séance de mercredi après-midi.

Tout d'abord, pour être transparent, je n'étais pas au meilleur de ma forme. Et vu la séance, j'ai laissé la frustration l'emporter, ce qui a nuis à la qualité du cours. Et pour cela, je reconnais ma responsabilité, et je vous présente mes excuses.

Parlons de vous maintenant. Car même face à un cours qui n'est pas à votre goût, à tort ou à raison, il y a des comportements qui ne sont pas acceptables. Lancer des stylos, ce n'est pas acceptable. Lancer des ciseaux, ce n'est pas acceptable. Lancer une gomme, ce n'est pas acceptable. Vous avez perdu de vue la règle dont on a parlé en début d'année : le respect. Vous n'avez pas respecté mon travail, vous n'avez pas respecté le reste de vos camarades, et vous avez choisi de renvoyer une image de vous qui est pitoyable. Et je parle de vous en temps que classe, car je sais qu'individuellement, vous n'êtes pas comme ça.

Donc, dans l'objectif de respect des règles, on va tous sortir et poser son carnet sur la table. À la moindre remarque, je le prends. Si je vous remarque encore alors que j'ai votre carnet, je vous mets une heure de colle.

Concernant les concessions et les arrangements, pour le moment, vous avez perdu le privilège que je vous les accorde. Donc les questions pour aller aux toilettes, pour entrer avec une minute de retard, pour ajouter quelques minutes aux minuteur du contrôle, ces questions, je vous dirais non systématiquement. Si vous me montrez par la suite que vous êtes capable de tenir votre part du marché, alors vous gagnerez à nouveau ce privilège.

Les choses ne peuvent pas rester sans conséquences. Je demande une lettre d'excuses de la part de la seconde 9, que les délégués me transmettront lundi.
\end{document}