\documentclass{article}
\usepackage{main}

\title{Fiche d'aide : Vecteurs}
\date{11 Octobre 2024}
\author{Seconde 9}

\begin{document}
\maketitle

\section{Définition}
\begin{itemize}
\item Direction, Sens, Norme
\item Montrer que deux vecteurs sont égaux : $\vect{AB} = \vect{DC}$ ?
\begin{enumerate}
\item \textbf{Car on sait que $ABCD$ est un parallélogramme}
\item \textbf{En lisant la figure}
\item \textbf{Par le calcul (voir après)}
\end{enumerate}
\end{itemize}
\section{Opérations}
\subsection{Somme de vecteurs}
Calculer $\vect{u} + \vect{v}$ :
\begin{itemize}
\item \textbf{Sur la figure (effectuer la translation de vecteur $\vect{u}$ puis la translation de vecteur $\vect{v}$)}
\item Dans la cas $\vect{AB} + \vect{BC}$ : \textbf{Relation de Chasles $\vect{A\underline{B}} + \vect{\underline{B}C} = \vect{AC}$}
\end{itemize}
\subsection{Opposé d'un vecteur}
\begin{itemize}
\item \textbf{Même direction, sens opposé, même norme} 
\item $\vect{AB} = - \vect{BA}$
\end{itemize}
\subsection{Multiplication par un nombre réel}
À propos de $\vect{v} = k \vect{u}$ :
\begin{itemize}
\item \textbf{Les vecteurs $\vect{u}$ et $\vect{v}$ sont colinéaires}
\item \textbf{Même direction, le sens dépend du signe de $k$, la norme est multiplié par la valeur absolue de $k$}
\item Dans un calcul : \textbf{$p \vect{u} + q \vect{u} = (p + q) \vect{u}$}
\end{itemize}
\section{Configuration géométrique}
\begin{itemize}
\item Parallélogramme $ABCD$ : \textbf{Car $\vect{AB} = \vect{DC}$}
\item Milieu $I$ de $[AB]$ : \textbf{Car $\vect{AI} = \vect{IB}$}
\item Droites parallèles $(AB) \parallel (CD)$ : \textbf{Car $\vect{AB}$ et $\vect{CD}$ sont colinéaire, pour passer de $\vect{AB}$ à $\vect{CD}$, il faut multiplier par un nombre $k$.}
\end{itemize}

\end{document}