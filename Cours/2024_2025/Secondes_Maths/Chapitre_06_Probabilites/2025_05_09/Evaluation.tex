\documentclass{exam}
\usepackage{mainExam}

\title{Évaluation de cours : Probabilités}
\date{9 Mai 2025}
\author{Seconde 9}

\begin{document}
\maketitle
\begin{questions}
\question[6]
Un magasin de cosmétiques fait l'inventaire de ses produits. On obtient le tableau de répartition suivant :
\begin{center}
\begin{tabular}{|c|c|c|}
\hline
&Rouge à lèvres&Maquillage\\
\hline
Européen&\num{68}&\num{56}\\
\hline
Asiatique&\num{49}&\num{12}\\
\hline
Américain&\num{90}&\num{53}\\
\hline
\end{tabular}
\end{center}
On tire au hasard un produit de ce magasin. On note les événements:
\begin{itemize}
\item $A$ : \og le produit sélectionné est américain \fg
\item $B$ : \og le produit sélectionné est du maquillage \fg
\end{itemize}
\vspace*{0.5cm}
\begin{parts}
\part[1] Calculer $P(A)$. $P(A) =$ \answersline 
\part[1] Calculer $P(B)$. $P(B) =$ \answersline 
\part[2] Donner la signification en français de l'événement $A \cap B$, puis calculer $P(A \cap B)$.
\begin{itemize}
\item $A \cap B$ : \og \answersline \fg 
\item $P(A \cap B) =$ \answersline 
\end{itemize}
\part[2] Donner la signification en français de l'événement $\overbar{B}$, puis calculer $P(\overbar{B})$.
\begin{itemize}
\item $\overbar{B}$ : \og \answersline \fg 
\item $P(\overbar{B}) =$ \answersline 
\end{itemize} 
\end{parts}
\vspace*{0.5cm}
\question[4]
On numérote deux urnes $1$ et $2$. Dans l'urne $1$, on dispose deux boules rouges et une boule bleue. Dans l'urne $2$, on dispose une boule rouge et deux boules bleues. On sélectionne au hasard une des deux urnes, puis on tire dans l'urne sélectionnée une boule au hasard.
\begin{parts}
\part Compléter l'arbre de dénombrement suivant, qui représente l'expérience aléatoire.
\begin{center}
\begin{tikzpicture}[level 1/.style={sibling distance=30mm},level 2/.style={sibling distance=10mm}]
\coordinate
    child {node {\dots}
        child {node {R}}
        child {node {R}}
        child {node {\dots}}}
    child {node {\dots}
        child {node {R}}
        child {node {B}}
        child {node {\dots}}};
\end{tikzpicture}
\end{center}
\part Quelle est la probabilité d'obtenir une boule rouge après cette expérience ?

\answersline
\end{parts}
\end{questions}
\newpage
\maketitle
\begin{questions}
\question[6]
Un magasin de bricolage fait l'inventaire de ses produits. On obtient le tableau de répartition suivant :
\begin{center}
\begin{tabular}{|c|c|c|c|}
\hline
&Marteau&Clou&Scie\\
\hline
En promotion&\num{17}&\num{80}&\num{9}\\
\hline
Sans Promotion&\num{88}&\num{20}&\num{113}\\
\hline
\end{tabular}
\end{center}
On tire au hasard un produit de ce magasin. On note les événements:
\begin{itemize}
\item $A$ : \og le produit sélectionné est en promotion \fg
\item $B$ : \og le produit sélectionné est une scie \fg
\end{itemize}
\vspace*{0.5cm}
\begin{parts}
\part[1] Calculer $P(A)$. $P(A) =$ \answersline 
\part[1] Calculer $P(B)$. $P(B) =$ \answersline 
\part[2] Donner la signification en français de l'événement $A \cap B$, puis calculer $P(A \cap B)$.
\begin{itemize}
\item $A \cap B$ : \og \answersline \fg 
\item $P(A \cap B) =$ \answersline 
\end{itemize}
\part[2] Donner la signification en français de l'événement $\overbar{B}$, puis calculer $P(\overbar{B})$.
\begin{itemize}
\item $\overbar{B}$ : \og \answersline \fg 
\item $P(\overbar{B}) =$ \answersline 
\end{itemize} 
\end{parts}
\vspace*{0.5cm}
\question[4]
On numérote deux urnes $1$ et $2$. Dans l'urne $1$, on dispose deux boules rouges et une boule bleue. Dans l'urne $2$, on dispose une boule rouge et deux boules bleues. On sélectionne au hasard une des deux urnes, puis on tire dans l'urne sélectionnée une boule au hasard.
\begin{parts}
\part Compléter l'arbre de dénombrement suivant, qui représente l'expérience aléatoire.
\begin{center}
\begin{tikzpicture}[level 1/.style={sibling distance=30mm},level 2/.style={sibling distance=10mm}]
\coordinate
    child {node {\dots}
        child {node {R}}
        child {node {R}}
        child {node {\dots}}}
    child {node {\dots}
        child {node {R}}
        child {node {B}}
        child {node {\dots}}};
\end{tikzpicture}
\end{center}
\part Quelle est la probabilité d'obtenir une boule rouge après cette expérience ?

\answersline
\end{parts}
\end{questions}
\end{document}