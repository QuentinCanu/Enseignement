\documentclass{exam}
\usepackage{mainExam}

\title{Calcul Littéral : feuille d'exercices}
\date{6 Septembre 2024}
\author{Seconde 9}

\begin{document}
\maketitle
\begin{questions}
\begin{minipage}{0.45\textwidth}
\titledquestion{Développement}
Développer et réduire les expressions suivantes:
\begin{parts}
\part $-4(4x-8)-(5-5x)$
\part $2-(11x-10)(11x-8)$
\part $10x+(-11x+2)(3x-11)$
\part $(-2x\times 3)(-x-2)$
\part $-5x-8(-2x+7)$
\part $(-3t-2)^2+(3t-3)^2$
\part $(-5z+1)^2+(-z+4)^2$
\part $(2x-5)^2+(2x+5)^2$
\part $(-2z-5)^2+(-2z+4)^2$
\part $(-2x-1)^2+(-2x-4)^2$
\end{parts} 
\end{minipage}
\hfill
\begin{minipage}{0.45\textwidth}
\titledquestion{Factorisation}
Factoriser les expressions suivantes :
\begin{parts}
\part $9a-21b$
\part $56a+63b$
\part $x(x+5)-3(x+5)$
\part $3(3x-4)-x(3x-4)$
\part $9x^2-64$
\part $64x^2-49$
\part $x^2+16x+64$
\part $x^2-2x+1$
\part $(9x+5)^2-4$
\part $(9x-7)^2-16$
\end{parts}
\end{minipage}

\vspace*{0.5cm}
\titledquestion{Pour aller plus loin}
\begin{parts}
\part
Soient $3$ nombres réels $a$, $b$ et $c$. Développer l'expression
\begin{equation*}
\left(a+b+c\right)^2
\end{equation*}
\part Soient $2$ nombres réels $a$ et $b$. Développer l'expression
\begin{equation*}
\left(a + b\right)^3    
\end{equation*}
\part Développer et réduire l'expression :
\begin{equation*}
(x-1)^2 + x^2 + (x+1)^2    
\end{equation*}
En déduire trois entiers consécutifs dont la somme des carrés vaut $4802$.
\end{parts}


\end{questions}
\end{document}