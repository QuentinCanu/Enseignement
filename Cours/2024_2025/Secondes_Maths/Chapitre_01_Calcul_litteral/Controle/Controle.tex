\documentclass{exam}
\usepackage{mainExam}

\title{Contrôle : calcul littéral, vecteurs}
\date{4 Octobre 2024}
\author{Seconde 9}

\begin{document}
\maketitle
\instructions{Interdite}
\vspace*{1cm}

\begin{questions}
\titledquestion{Les identités remarquables}[4]
\begin{parts}
\part Compléter les identités suivantes:
\begin{subparts}
\subpart $(a + b)^2 =$
\subpart $a^2 - b^2 =$
\end{subparts}
\part Appliquer les identités remarquables pour simplifier les expressions suivantes :
\begin{subparts}
\subpart $x^2 - 4$
\subpart $(3y - 4)^2$
\subpart $16z^2 + 40z + 25$
\subpart $36c^2 - 81d^2$
\end{subparts}
\end{parts}
\vspace*{1cm}
\titledquestion{Vecteurs}[4]
\begin{parts}
\part Rappeler les trois composantes définissant un vecteur.
\part Dans la figure ci-après, les triangles $ABC$, $BDE$ et $CEF$ sont équilatéraux.

\begin{center}
\begin{tikzpicture}
\draw (0,0) node[below left] {$A$} -- (0:3) node[below] {$B$} -- (60:3) node[left] {$C$} -- cycle;
\draw (0:3) -- ++(0:3) node[below right] {$D$} -- ++(120:3) node[right] {$E$} -- cycle;
\draw (60:3) -- ++(0:3) -- ++(120:3) node[above] {$F$} -- cycle;
\end{tikzpicture}
\end{center}
Donner (sans justifier) un vecteur correspondant à chacun des critères suivants :
\begin{subparts}
\subpart Un vecteur égal à $\vect{AB}$;
\subpart Un vecteur égal à $\vect{CF}$;
\subpart Un vecteur égal à $\vect{DB}$;
\subpart Un représentant de $\vect{BC}$ d'origine $E$;
\subpart Un représentant de $\vect{FE}$ d'extrémité $E$;
\subpart Un vecteur de même direction que $\vect{DE}$ mais pas de même norme ni de même sens.
\end{subparts}
\end{parts}

\newpage
\titledquestion{Sciences expérimentales}[5]
\begin{parts}
\part Pour déterminer une masse volumique $M$ d'un échantillon, en \unit{\kilo\gram\per \centi\meter}, on divise sa masse $m$ par son volume $V$ :
\begin{equation*}
M = \dfrac{m}{V}    
\end{equation*}
\begin{subparts}
\subpart Exprimer le volume $V$ d'un échantillon en fonction de sa masse $m$ et de sa masse volumique $M$.
\subpart En déduire le volume total de deux échantillons, le premier de masse $m = \qty{36}{\kilo\gram}$ et de masse volumique $M = \qty{48}{\kilo\gram\per\centi\meter}$, le deuxième de masse $m' = \qty{5}{\kilo\gram}$ et de masse volumique $M' = \qty{25}{\kilo\gram\per\centi\meter}$.
\end{subparts}
\part La force de poussée permettant la flotaison des bateaux est appelée la poussée d'Archimède. Son intensité est donnée par la formule :
\begin{equation*}
F = M \times v \times g
\end{equation*}
où $F$ est la force, $M$ est la masse volumique de l'objet immergé, $v$ le volume immergé et $g$ l'accélération de la pesanteur.
\begin{subparts}
\subpart Nous nous plaçons dans le cas où l'objet est entièrement immergé (dans ce cas, $v = V$). En utilisant l'expression de la masse volumique $M$ de la question précédente, montrer que l'intensité de la force peut être simplifié en $F = mg$, où $m$ est la masse de l'objet immergé.
\subpart En déduire une expression de $g$ en fonction de $m$ et de $F$.   
\end{subparts}
\part La \emph{loi des gaz parfaits} est une égalité mettant en relation différents paramètres caractérisant un gaz dit \og parfait \fg : la pression ($P$), la température ($T$), la quantité de matière ($n$) et le volume occupé ($V$). Elle dépend de la \emph{constante des gaz parfaits}, notée $R$, et est donnée par :
\begin{equation*}
P \times V = n \times R \times T\,.
\end{equation*}
Exprimer $R$ en fonction de $P$, $V$, $n$ et $T$. 
\end{parts}

\vspace*{1cm}
\titledquestion{Développements, factorisations}[5]
\begin{parts}
\part Développer et simplifier les expressions suivantes :
\begin{subparts}
\subpart $(5a + 3)(9a + 4)$
\subpart $(y - 2)(y + 3) + (2y + 1)(y + 3)$
\end{subparts}
\part Factoriser les expressions suivantes :
\begin{subparts}
\subpart $35p + 42q$
\subpart $(y - 2)(y + 3) + (2y + 1)(y + 3)$
\end{subparts}
\part Simplifier $(x + 1)^2 - (x - 1)^2$. En déduire $10001^2 - 9999^2$.
\part Montrer que les trois expressions suivantes sont égales, quelque soit la valeur de $t$.
\begin{subparts}
\subpart $A = (2t - 4)^2 + 12$
\subpart $B = 4(t - 2)^2 + 12$
\subpart $C = 4(t - 3)(t - 1) + 16$
\end{subparts}
\end{parts}
\vspace*{1cm}
\titledquestion{Bonus}[2]
Développer et simplifier $(a+b)^3$. En déduire $101^3$.
\end{questions}
\end{document}