\documentclass{exam}
\usepackage{mainExam}

\title{Évaluation de cours}
\date{27 Janvier 2025}
\author{Seconde 9}

\begin{document}
\maketitle

\begin{questions}
\titledquestion{Démonstration}[5]
Démontrer la proposition suivante : \og $\dfrac{1}{3}$ n'est pas décimal.\fg On utilisera un raisonnement par l'absurde.
\vspace*{0.1cm}
\makeemptybox{8cm}
\vspace*{0.5cm}
\titledquestion{Fractions}[3]
Dire si les fractions suivantes sont décimales ou non. \textbf{Justifier}.
\begin{parts}
\part[0.5] $\dfrac{1}{9}$ \answersline
\vspace*{0.2cm}
\part[0.5] $\dfrac{64}{25}$ \answersline
\vspace*{0.2cm}
\part[0.5] $\dfrac{99}{24}$ \answersline
\vspace*{0.2cm}
\part[0.5] $\dfrac{105}{45}$ \answersline
\vspace*{0.2cm}
\part[0.5] $\dfrac{78}{32}$ \answersline
\vspace*{0.2cm}
\part[0.5] $\dfrac{110}{70}$ \answersline
\vspace*{0.2cm}
\end{parts}
\vspace*{0.5cm}
\titledquestion{Puissances}[2]
Écrire les expressions suivantes sous la forme $a^n$, avec $a \in \R$ et $n \in \Z$.
\begin{parts}
\part[0,5] $3^2 \times 3^4 = $ \answersline
\vspace*{0.2cm}
\part[0,5] ${(4^5)}^3 = $ \answersline
\vspace*{0.2cm}
\part[0,5] $11^8 \times 5^8 = $ \answersline
\vspace*{0.2cm}
\part[0,5] $\dfrac{2^4}{4^3} = $ \answersline
\vspace*{0.2cm}
\end{parts}
\end{questions}

\newpage
\maketitle

\begin{questions}
\titledquestion{Démonstration}[5]
Démontrer la proposition suivante : \og $\dfrac{1}{3}$ n'est pas décimal.\fg On utilisera un raisonnement par l'absurde.
\vspace*{0.1cm}
\makeemptybox{8cm}
\vspace*{0.5cm}
\titledquestion{Fractions}[3]
Dire si les fractions suivantes sont décimales ou non. \textbf{Justifier}.
\begin{parts}
\part[0.5] $\dfrac{64}{25}$ \answersline
\vspace*{0.2cm}
\part[0.5] $\dfrac{1}{9}$ \answersline
\vspace*{0.2cm}
\part[0.5] $\dfrac{105}{45}$ \answersline
\vspace*{0.2cm}
\part[0.5] $\dfrac{99}{24}$ \answersline
\vspace*{0.2cm}
\part[0.5] $\dfrac{110}{70}$ \answersline
\vspace*{0.2cm}
\part[0.5] $\dfrac{78}{32}$ \answersline
\vspace*{0.2cm}
\end{parts}
\vspace*{0.5cm}
\titledquestion{Puissances}[2]
Écrire les expressions suivantes sous la forme $a^n$, avec $a \in \R$ et $n \in \Z$.
\begin{parts}
\part[0,5] $7^3 \times 7^8 = $ \answersline
\vspace*{0.2cm}
\part[0,5] ${(9^4)}^2 = $ \answersline
\vspace*{0.2cm}
\part[0,5] $5^2 \times 12^2 = $ \answersline
\vspace*{0.2cm}
\part[0,5] $\dfrac{3^5}{9^3} = $ \answersline
\vspace*{0.2cm}
\end{parts}
\end{questions}
\end{document}