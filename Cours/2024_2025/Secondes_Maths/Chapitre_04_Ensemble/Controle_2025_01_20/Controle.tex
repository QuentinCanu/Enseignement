\documentclass{exam}
\usepackage{mainExam}

\title{Contrôle : évolutions, arithmétique}
\date{20 Janvier 2025}
\author{Seconde 9}

\begin{document}
\maketitle
\instructions{Autorisée}

\begin{questions}
\vspace*{1cm}
\titledquestion{Évolutions successives, évolutions réciproques}[4]
\begin{parts}
\part[1] La population d'une ville augmente de $5\%$ entre $2022$ et $2023$ puis augmente de $20\%$ entre $2023$ et $2024$. Quel est le taux d'évolution de cette population entre $2022$ et $2024$ ?
\part[1] Le prix de vente d'un livre a été soldé de $40\%$ avant d'être remonté de $12\%$. Le prix du livre est-il plus bas ou plus haut qu'initialement ?
\part[1] La température moyenne d'une planète a augmenté de $3\%$. De quel pourcentage cette température doit diminuer pour retourner à son état initial ?
\part[1] Un objet de collection a pris de la valeur, au point de voir son prix augmenter de $15\%$. Quel est le taux de diminution nécessaire en pourcentage pour que cet objet retrouve son prix initial ? 
\end{parts}
\vspace*{1cm}
\titledquestion{Diviseurs}[5]
\begin{parts}
\part Soit trois entiers relatifs $a, b$ et $k$ vérifiant :
\begin{equation*}
a = k \times b
\end{equation*}
Recopier et compléter les phrases suivantes :
\begin{subparts}
\subpart[0,5] $a$ est un \dots{} de $b$
\subpart[0,5] $b$ est un \dots{} de $a$
\end{subparts}
\part Donner la liste des diviseurs des nombres suivants :
\begin{subparts}
\subpart[0,5] $70$
\subpart[0,5] $110$
\subpart[0,5] $182$
\subpart[0,5] $1050$
\end{subparts}
\part[2] Lister tous les nombres \textbf{impairs}, \textbf{à $3$ chiffres}, \textbf{divisibles par $3$ et $5$} et dont \textbf{le chiffre des centaines est plus grand que la somme du chiffre des unités et du chiffre des dizaines}.
\end{parts}
\newpage
\titledquestion{Nombres premiers}[5]
\begin{parts}
\part[1] Rappeler la définition de nombre premier, puis donner \num{5} nombres premiers.
\part[1] Les nombres \num{503} et \num{507} sont-ils premiers ? Détaillez ce que vous avez fait pour obtenir votre réponse.
\part[1] Donner la décomposition en facteurs premiers de \num{304920}.
\part En $1777$, Leonhard Euler propose la formule suivante pour produire des nombres premiers :
\begin{equation*}
n^2 - n + 41  
\end{equation*}
où $n$ est un nombre entier naturel.
\begin{subparts}
\subpart[1] Choisir trois valeurs possibles inférieures à $10$ pour $n$. Pour chacune d'entre elle, vérifier que la formule donne bien un nombre premier.
\subpart[1] Justifier pourquoi la formule ne donne pas un nombre premier pour $n = 41$.
\end{subparts}
\end{parts}
\vspace*{1cm}
\titledquestion{Démonstrations}[5]
\begin{parts}
\part[2] Rappeler la démonstration de la démonstration suivante : le carré d'un nombre impair est un nombre impair.
\part
\begin{subparts}
\subpart[0,5] Effectuer les additions suivantes : $3+4$; $6+7$; $99+100$.

La somme de deux nombres consécutifs semble-t-elle toujours paire ou toujours impaire ?
\subpart[1]
Recopier et compléter la démonstration suivante :
\begin{tcolorbox}
\begin{quote}
Soit $n$ un entier relatif. On considère la somme $n + (n+1)$ (c'est la somme de deux entiers consécutifs).

Alors,
\begin{equation*}
n + (n+1) = \text{ \dots}
\end{equation*}

On en déduit que cette somme est impaire.
\end{quote}
\end{tcolorbox}
\end{subparts}
\part 
\begin{subparts}
\subpart[0,5] Effectuer les sommes suivantes : $3+4+5$; $20+21+22$; $203+204+205$. Vérifier la divisibilité par $3$ de chaque résultat. Que remarquez-vous ?
\subpart[1] Recopier et compléter la démonstration suivante :
\begin{tcolorbox}
\begin{quote}
Soit $n$ un entier relatif. On considère la somme $n + (n+1) + (n+2)$ (c'est la somme de trois entiers consécutifs).

Alors,
\begin{equation*}
n + (n + 1) + (n + 2) = \text{ \dots}
\end{equation*}

On en déduit que cette somme est divisible par $3$.
\end{quote}
\end{tcolorbox} 
\end{subparts}
\end{parts}
\vspace*{1cm}
\end{questions}
\end{document}