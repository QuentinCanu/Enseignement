\documentclass{exam}
\usepackage{mainExam}

\title{Évaluation de cours}
\author{Seconde 9}
\date{13 Décembre 2024}

\begin{document}
\maketitle
\instructions{Autorisée}

\begin{questions}
\titledquestion{Évolutions successives}
\begin{parts}
\part La population d'une ville a augmenté de $6~\%$ en $2021$ puis a diminué de $1~\%$ en $2022$.

Quel est le taux d'évolution global ?
\part Le prix d'un article subit une hausse de $27~\%$ puis une hausse de $23~\%$.

Déterminer le taux d'évolution global du prix de cet article.
\part Le nombre d'adhérents d'une association a diminué de $25~\%$ entre $2020$ et $2021$ puis a augmenté de $4~\%$ entre $2021$ et $2022$.

Quel est le taux d'évolution global du nombre d'adhérents ?
\end{parts}
\titledquestion{Évolutions reciproques}
\begin{parts}
\part Le prix d'un article subit une hausse de $32\,\%$.

Quelle évolution devra-t-il subir pour revenir à son prix initial ?

On donnera le taux d'évolution en pourcentage, éventuellement arrondi à $0,01\,\%$ près.
\part Une luthière a décidé d'augmenter son tarif horaire de $3\,\%$.

Quelle évolution devra-t-il subir pour revenir à son niveau de départ ?

On donnera le taux d'évolution en pourcentage, éventuellement arrondi à $0,01\,\%$ près.
\end{parts}
\end{questions}

\newpage
\maketitle
\instructions{Autorisée}

\begin{questions}
\titledquestion{Évolutions successives}
\begin{parts}
\part Le nombre d'adhérents d'une association a augmenté de $27~\%$ entre $2020$ et $2021$ puis a augmenté de $3~\%$ entre $2021$ et $2022$.

Quel est le taux d'évolution global du nombre d'adhérents ?
\part La population d'une ville a augmenté de $14~\%$ en $2021$ puis a augmenté de $1~\%$ en $2022$.

Quel est le taux d'évolution global ?
\part Le prix d'un article subit une hausse de $16~\%$ puis une baisse de $48~\%$.

Déterminer le taux d'évolution global du prix de cet article.
\end{parts}
\titledquestion{Évolutions réciproques}
\begin{parts}
\part Une informaticienne a décidé d'augmenter son tarif horaire de $31\,\%$.
    
Quelle évolution devra-t-il subir pour revenir à son niveau de départ ?
    
On donnera le taux d'évolution en pourcentage, éventuellement arrondi à $0,01\,\%$ près.
\part Le nombre de commerciaux d'une entreprise a baissé de $35\,\%$.
    
Quelle évolution permettrait de retrouver le nombre de départ ?
    
On donnera le taux d'évolution en pourcentage, éventuellement arrondi à $0,01\,\%$ près. 
\end{parts}
\end{questions}

\newpage
\maketitle
\instructions{Autorisée}

\begin{questions}
\titledquestion{Évolutions successives}
\begin{parts}
\part La population d'une ville a diminué de $17~\%$ en $2021$ puis a augmenté de $12~\%$ en $2022$.

Quel est le taux d'évolution global ?
\part Le prix d'un article subit une baisse de $61~\%$ puis une baisse de $4~\%$.

Déterminer le taux d'évolution global du prix de cet article.
\part Le nombre d'adhérents d'une association a augmenté de $13~\%$ entre $2020$ et $2021$ puis a augmenté de $29~\%$ entre $2021$ et $2022$.

Quel est le taux d'évolution global du nombre d'adhérents ?
\end{parts}
\titledquestion{Évolutions réciproques}
\begin{parts}
\part Le prix d'un article subit une hausse de $39\,\%$.

Quelle évolution devra-t-il subir pour revenir à son prix initial ?

On donnera le taux d'évolution en pourcentage, éventuellement arrondi à $0,01\,\%$ près.
\part Le nombre de stagiaires d'une entreprise a baissé de $23\,\%$.

Quelle évolution permettrait de retrouver le nombre de départ ?

On donnera le taux d'évolution en pourcentage, éventuellement arrondi à $0,01\,\%$ près.

\end{parts}
\end{questions}

\newpage
\maketitle
\instructions{Autorisée}

\begin{questions}
\titledquestion{Évolutions successives}
\begin{parts}
\part Le prix d'un article subit une baisse de $71~\%$ puis une hausse de $6~\%$.

Déterminer le taux d'évolution global du prix de cet article.
\part La population d'une ville a diminué de $19~\%$ en $2021$ puis a diminué de $12~\%$ en $2022$.

Quel est le taux d'évolution global ?
\part Le nombre d'adhérents d'une association a augmenté de $4~\%$ entre $2020$ et $2021$ puis a augmenté de $31~\%$ entre $2021$ et $2022$.

Quel est le taux d'évolution global du nombre d'adhérents ? 
\end{parts}
\titledquestion{Évolutions réciproques}
\begin{parts}
\part Le prix d'un article subit une hausse de $15\,\%$.

Quelle évolution devra-t-il subir pour revenir à son prix initial ?

On donnera le taux d'évolution en pourcentage, éventuellement arrondi à $0,01\,\%$ près.
\part Le nombre d'employés d'une entreprise a augmenté de $2\,\%$.

Quelle évolution permettrait de retrouver le nombre de départ ?

On donnera le taux d'évolution en pourcentage, éventuellement arrondi à $0,01\,\%$ près.
\end{parts}
\end{questions}
\end{document}