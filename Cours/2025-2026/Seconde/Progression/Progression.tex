\documentclass{article}
\usepackage{main}

\title{Progression Seconde 3}
\date{2025-2026}
\author{Quentin Canu}

\begin{document}
\maketitle
\section{Premier Trimestre}
\subsection*{Chapitre 1 : Calcul littéral, développement, factorisation, identités remarquables, équations}
\paragraph{Début le 3 Septembre}
\begin{itemize}
\item Egalité et substitution : Si $x = y$, alors $P(x) = P(y)$. (1h)
\item Propriété des égalités : Si $a = b$ alors $a + c = b + c$, $a \times c = b \times c$ et $\dfrac{a}{c} = \dfrac{b}{c}$. Exprimer une variable en fonction d'une autre. (1h) 
\item Equations polynomiales de degré 1 (demi-groupe)
\item Développement, factorisation (2h)
\item Identités remarquables (Evaluation de cours)
\end{itemize}
\paragraph{Fin le 12 Septembre}
\subsection*{Chapitre 2 : Information chiffrée : Proportions et Evolutions}
\paragraph{Début le 15 Septembre}
\begin{itemize}
\item Exercices bilans + fractions (demi-groupe)
\item Populations, sous-populations, proportions (2h)
\item Proportions de proportions (Evaluation de cours) (1h)
\item Rappels sur les fractions (demi-groupe)
\item Variation absolue, variation relative, taux d'évolution (2h)
\item Controle
\item Exercices bilan (demi-groupe)
\item Taux d'évolutions successifs, taux d'évolution réciproques
\end{itemize}
\paragraph{Fin le 1\ier{} Octobre}
\subsection*{Chapitre 3 : Géométrie repérée}
\end{document}