\documentclass{article}
\usepackage{main}

\title{Présentation de rentrée}
\date{3 Septembre 2025}
\author{Quentin Canu}

\begin{document}
\maketitle
\section{Entrée en classe}
\begin{itemize}
\item Appel par ordre alphabétique (prévoir une liste d'élèves);
\item Rangement des téléphones dans la malette dans l'ordre
\end{itemize}
\section{Vérification de l'emploi du temps}
\section{Explication du matériel}
\begin{itemize}
\item 2 Cahiers $24 \times 32$: cours + exercice. On collera la feuille du jour sur une page de gauche, la page de droite servira à prendre des notes.
\item Calculatrice: uniquement si on en a déjà une, et si on compte faire spé maths. Sinon, pas la peine.
\end{itemize}
\section{Explication des notes}
\begin{itemize}
\item Note d'investissement, coefficient $0,5$: commence à $10$, monte de $0,5$ en cas de bon travail durant une séance, baisse de $0,5$ en cas de manquement aux règles. Pour le premier trimestre, je vérifie les cahiers.
\item Évaluations, coefficient $0,5$: tous les vendredis, dure un quart d'heure, exercices déjà vus en cours, commence le 12 Septembre.
\item Contrôles, coefficient $2$: toutes les trois semaines/un mois: le premier est prévu le 26 Septembre.
\end{itemize}
\section{Règles du cours}
\begin{enumerate}
\item En cours de maths, on fait des maths.
\item Une seule personne à la fois prend la parole.
\item Si je ne comprends pas, je dois demander de l'aide.
\end{enumerate}
\end{document}