\documentclass{article}
\usepackage{main}

\title{Art abstrait}
\author{Seconde 3}
\date{}

\begin{document}
\maketitle
\section{Tableau}
Une fresque d'art abstrait fait figurer quatre zones différentes, toutes rectangulaires:
\begin{itemize}
\item La zone délimitée par le rectangle $HIFG$ sera coloriée en rouge.
\item La zone délimitée par le rectangle $BCDI$ sera coloriée en bleu.
\item La zone délimitée par le rectangle $ABIH$ contiendra un mélange de bleu et de rouge.
\item La dernière zone, délimité par $IDEF$ restera en blanc.
\end{itemize}
\begin{center}
\begin{tikzpicture}
\coordinate (A) at (0,0);
\coordinate (B) at (2,0);
\coordinate (C) at (5,0);
\coordinate (D) at (5,2);
\coordinate (E) at (5,4);
\coordinate (F) at (2,4);
\coordinate (G) at (0,4);
\coordinate (H) at (0,2);
\coordinate (I) at (2,2);
\draw (A) -- (C) -- (E) -- (G) -- cycle;
\draw (B) -- (F);
\draw (D) -- (H);
\draw (A) node[below left] {$A$};
\draw (B) node[below] {$B$};
\draw (C) node[below right] {$C$};
\draw (D) node[right] {$D$};
\draw (E) node[above right] {$E$};
\draw (F) node[above] {$F$};
\draw (G) node[above left] {$G$};
\draw (H) node[left] {$H$};
\draw (I) node[above right] {$I$};
\draw[<->] (A) ++ (0,-0.5) -- ++ (2,0) node[below, midway] {$q$};
\draw[<->] (A) ++ (0,-1) -- ++ (5,0) node[below, midway] {\qty{50}{\meter}};
\draw[<->] (A) ++ (-0.5, 0) -- ++ (0,2) node[left, midway] {$p$};
\draw[<->] (A) ++ (-1, 0) -- ++ (0,4) node[left, midway] {\qty{40}{\meter}};
\end{tikzpicture}
\end{center}
\begin{alphaquestions}
\item Ajouter de la couleur à la figure ci-dessus pour correspondre aux zones du tableau.
\item\label{perim} Proposer une formule pour calculer les périmètres des zones suivantes, en fonction des indéterminées $p$ et $q$:
\begin{itemize}
\item La zone coloriée avec le mélange de bleu et de rouge 
\item La zone totale coloriée avec du bleu
\item La zone totale coloriée avec du rouge
\item La zone restant blanche
\end{itemize}
\item\label{val} En déduire la valeur du périmètre de ces rectangles en fonction des valeurs données à $p$ et $q$:
\begin{center}
\begin{tabular}{|l|c|c|c|c|}
\hline
&Mélange&Rouge&Bleu&Blanc\\
\hline
$p = 20$ et $q = 20$&&&&\\
\hline
$p = 5$ et $q = 7$&&&&\\
\hline
$p = 15$ et $q = 2$&&&&\\
\hline
\end{tabular}
\end{center}
\item Même question que \ref{perim} et \ref{val}, mais en remplaçant le périmètre par \textbf{l'aire} des rectangles considérés.
\begin{center}
\begin{tabular}{|l|c|c|c|c|}
\hline
&Mélange&Rouge&Bleu&Blanc\\
\hline
$p = 20$ et $q = 20$&&&&\\
\hline
$p = 5$ et $q = 7$&&&&\\
\hline
$p = 15$ et $q = 2$&&&&\\
\hline
\end{tabular}
\end{center}
\end{alphaquestions}
\hfill\vline\hfill
\section{Collection}
Un dyptique est constitué de deux tableaux liés par leur taille autour d'un nombre $x$ indéterminé : le premier est un rectangle de longueur $x + 1$ et de largeur $x - 1$, le deuxième est un carré de côté $x$ dont on a retiré un petit carré de côté $1$.
\begin{center}
\begin{tikzpicture}
\coordinate (A) at (0,0);
\coordinate (B) at (3,0);
\coordinate (C) at (3,1);
\coordinate (D) at (0,1);
\coordinate (E) at (4,0);
\coordinate (F) at (6,0);
\coordinate (G) at (6,2);
\coordinate (H) at (5,2);
\coordinate (I) at (5,1);
\coordinate (J) at (4,1);
\coordinate (K) at (4,2);
\draw (A) -- (B) -- (C) -- (D) -- cycle;
\draw (E) -- (F) -- (G) -- (H) -- (I) -- (J) -- cycle;
\draw[dashed] (J) -- (K) -- (H);
\draw[<->] (A) ++ (0,-0.25) -- ++(3,0) node[midway,below] {$x+1$};
\draw[<->] (A) ++ (-0.25,0) -- ++(0,1) node[midway,left] {$x-1$};
\draw[<->] (E) ++ (0,-0.25) -- ++(2,0) node[midway,below] {$x$};
\draw[<->] (K) ++ (-0.25,0) -- ++(0,-1) node[midway,left] {$1$};
\end{tikzpicture}
\end{center}
\begin{alphaquestions}
\item Donner la formule permettant de calculer l'aire de ces deux tableaux.
\item Le nombre $x$ peut-il prendre n'importe quelle valeur ?
\item Calculer l'aire de chacun des deux tableaux, en choisissant plusieurs valeurs de $x$.  
\item Que remarquez-vous ?
\end{alphaquestions}
\end{document}