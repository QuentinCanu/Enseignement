\documentclass{controle}
\usepackage{main}

\title{Contrôle n°3 : Fonctions inverses; suites géométriques}
\date{9 Janvier 2026}
\author{TSTMG1}

\begin{document}
\maketitle

\instructions[Autorisée]

\begin{questions}
\titledquestion{Sommes géométrique}[4]
\begin{parts}
\part[1] Somme 1
\part[1] Somme 2
\part[1] Somme 3
\part[1] Somme 4
\end{parts}
\titledquestion{Fonction inverse}[8]
\begin{parts}
\part[1] Cout unitaire $C_m(q) = \dfrac{C(q)}{q}$; vérifier une image
\part[2] Vérifier $C_m'(q)$ = polynôme;
\part[2] Vérifier $C_m'(q)$ = fraction;
\part[2] Construire tableau de signes + tableau de variations;
\part[1] En déduire coût unitaire extremal
\end{parts}
\titledquestion{Suite arithmético-géométrique}[8]
\begin{parts}
\part[1] Premiers termes de $(u_n)$
\part[1] Est-elle géométrique ? Arithmétique ?
\part[2] Montrer que la suite auxiliaire $(v_n)$ est géométrique.
\part[2] En déduire la formule explicite de $v_n$.
\part[1] En déduire la formule explicite de $u_n$.
\part[1] Calculer $u_{20}$
\end{parts}
\end{questions}
\end{document}