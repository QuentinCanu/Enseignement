\documentclass{poly}
\usepackage{main}

\title{Variables Aléatoires}
\date{}
\author{Terminale STMG1}

\begin{document}
\maketitle
\section{Espérance d'une variable aléatoire}
\begin{definition}
Soit une expérience aléatoire d'univers $\Omega$. Une \textbf{variable aléatoire} est une fonction définie sur $\Omega$ et à valeur dans $\R$. 
\end{definition}
\begin{example}
On donne des exemples de variables aléatoires :
\begin{center}
\begin{tabular}{|c|c|}
\hline
Expérience&Variable aléatoire $X$\\
\hline
On lance un dé à six faces & Le gain $X$ correspondant au double du résultat\\
\hline
On sélectionne un élève du lycée au hasard & L'âge de l'élève sélectionné\\
\hline
On choisit un mot au hasard dans le dictionnaire & Le nombre de lettres du mot choisi\\
\hline
&\\
\hline
&\\
\hline
\end{tabular}
\end{center}
\end{example}-
\begin{definition}
Soit $X$ une variable aléatoire d'une expérience aléatoire d'univers $\Omega$ \textbf{fini}. Déterminer \textbf{la loi de $X$} correspond à établir la probabilité de chaque valeur possible de $X$.
\end{definition}
\begin{example}
Dans une urne on dispose une boule rouge, deux boules vertes, et trois boules bleues. On mise 1€, et on tire au hasard une boule dans cette urne.
\begin{itemize}
\item Si la boule est rouge, on remporte le double de la mise.
\item Si la boule est verte, on récupère sa mise.
\item Si la boule est bleue, la mise est perdue.
\end{itemize}
On note $X$ la mise obtenue. Déterminer la loi de $X$ :
\begin{center}
\begin{tabular}{|c|c|c|c|}
\hline
Valeur de $X$&$X=-1$&$X=0$&$X=1$\\
\hline
Probabilité&&&\\
\hline
\end{tabular}
\end{center}
\end{example}
\newpage
\begin{definition}
Soit $X$ une variable aléatoire d'une expérience aléatoire d'univers $\Omega$ \textbf{fini}. On suppose que $X$ peut prendre les valeurs $x_1; x_2; \dots; x_n$. On appelle alors \textbf{espérance de $X$}, notée $E(X)$, la quantité
\begin{equation*}
E(X) = x_1 \times P(X=x_1) + x_2 \times P(X = x_2) + \dots + x_n \times P(X = x_n)
\end{equation*} 
\end{definition}
\begin{remark}
L'espérance de $X$ correspond à la valeur moyenne que l'on peut espérer si l'on répète l'expérience un grand nombre de fois.
\end{remark}
\begin{example}
Sur un jeu à gratter coûtant \num{1}€, il est indiqué que sur \num{1000000} tickets :
\begin{itemize}
\item \num{1} permet de gagner \num{10000}€; 
\item \num{5} permettent de gagner \num{1000}€; 
\item \num{100} permetten de gagner \num{500}€; 
\item \num{1000} permettent de gagner \num{200}€; 
\item \num{1500} permettent de gagner \num{100}€; 
\item \num{3000} permettent de gagner \num{10}€;
\item \num{4000} permettent de gagner \num{1}€;
\item Le reste ne fait rien gagner du tout.
\end{itemize}
On appelle $X$ le gain en € gagné après achat et grattage d'un ticket à gratter.
\begin{alphaquestions}
\item Déterminer la loi de $X$.
\item Calculer son espérance.
\end{alphaquestions}

\end{example}
\newpage
\section{Schéma de Bernoulli et loi binomiale}
\subsection{Épreuve de Bernoulli}
\begin{example}
On lance une pièce équilibrée. On gagne 1€ si l'on fait face et rien si on fait pile. Quelle est la probabilité de gagner un euro ? Si on note $X$ le gain possible suite à cette expérience, calculer $E(X)$.

\end{example}
\begin{definition}
Soit une expérience aléatoire d'univers $\Omega$, et $X$ une variable aléatoire d'$\Omega$. On dit que $X$ \textbf{suit une loi de Bernoulli de paramètre $p \in [0;1]$}, ce que l'on note $X \hookrightarrow \mathcal{B}(p)$, quand la loi de $X$ est donnée par le tableau suivant :
\begin{center}
\begin{tabular}{|c|c|c|}
\hline
$x_i$&$1$&$0$\\
\hline
$P(X = x_i)$&$p$&$1 - p$\\
\hline
\end{tabular}
\end{center}
\end{definition}
\begin{proposition}
Soit une expérience aléatoire d'univers $\Omega$, et $X$ une variable aléatoire d'$\Omega$ suivant une loi de Bernoulli de paramètre $p$. Alors,
\begin{equation*}
E(X) = p
\end{equation*}
\end{proposition}
\begin{remark}
Une expérience de Bernoulli est donc une expérience durant laquelle on est confronté à un succès (le $1$) ou à un échec (le $0$).
\end{remark}

\subsection{Schéma de Bernoulli}
\begin{example}
On lance dix pièces équilibrées. On note $X$ le nombre de pile obtenus après tirage.
\end{example}
\begin{remark}
Dans l'expérience précédente, $X$ compte le nombre de succès de $n = 10$ expériences de Bernoulli indépendantes, ayant toute la même probabilité de succès $p = \dfrac{1}{2}$. On dit que c'est un \textbf{schéma de Bernoulli de paramètres $n$ et $p$}.
\end{remark}    
\begin{definition}
Soit une expérience aléatoire d'univers $\Omega$, ainsi que $n \in \N$ et $p \in [0;1]$. On dit que $X$ \textbf{suit une loi binomiale de paramètres $n$ et $p$}, ce qui l'on note $X \hookrightarrow \mathcal{B}(n;p)$, quand $X$ comptabilise le nombre de succès de $n$ dans un schéma de Bernoulli de paramètres $n$ et $p$. 
\end{definition}
\begin{proposition}
Soit une expérience aléatoire d'univers $\Omega$, et $X \hookrightarrow \mathcal{B}(n;p)$. Alors, on a
\begin{equation*}
E(X) = n \times p
\end{equation*}
\end{proposition}
\begin{example}
Pour chacun des exemples suivants, donner l'espérance de $X$ :
\begin{alphaquestions}
\item On tire trois fois une boule dans une urne contenant deux boules vertes et quatres boules rouges, avec remise après chaque tirage. $X$ est le nombre de boules vertes tirées. 
\item $X$ est le nombre de match gagnés par une équipe de foot sur une saison de $10$ matchs. Les matchs sont tous très équitables, et il n'y a pas de match nul.
\item On lance quatre fois un dé à 10 faces. $X$ est le nombre de fois où l'on obtient un multiple de $3$.
\end{alphaquestions}

\end{example}
\end{document}