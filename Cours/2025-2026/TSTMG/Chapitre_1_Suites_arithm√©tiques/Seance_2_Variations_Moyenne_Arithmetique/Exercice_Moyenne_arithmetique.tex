\documentclass{exos}
\usepackage{main}

\begin{document}
\begin{exercize*}
Soit $(u_n)_{n \in \N}$ une suite arithmétique. Dans tous les cas présentés ci-après, donner la raison $r$ de $(u_n)_{n \in \N}$ ainsi que le terme $u_10$.
\begin{alphaquestions}
\item $u_0 = 5$ et $u_1 = 10$;
\item $u_9 = 28$ et $u_{11} = 12$
\item $u_8 = -45$ et $u_{12} = -15$
\item $u_5 = 49$ et $u_7 = 63$
\item $u_{25} = 120$ et $u_{27}=122$
\end{alphaquestions} 
\end{exercize*}
\vspace*{3cm}
\begin{exercize*}
Soit $(u_n)_{n \in \N}$ une suite arithmétique. Dans tous les cas présentés ci-après, donner la raison $r$ de $(u_n)_{n \in \N}$ ainsi que le terme $u_10$.
\begin{alphaquestions}
\item $u_0 = 5$ et $u_1 = 10$;
\item $u_9 = 28$ et $u_{11} = 12$
\item $u_8 = -45$ et $u_{12} = -15$
\item $u_5 = 49$ et $u_7 = 63$
\item $u_{25} = 120$ et $u_{27}=122$
\end{alphaquestions} 
\end{exercize*}
\end{document}