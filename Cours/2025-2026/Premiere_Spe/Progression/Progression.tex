\documentclass{article}

\usepackage{enumitem}

\title{Progression}
\author{Quentin Canu}
\date{2025-2026}

\begin{document}
\maketitle
\section{Premier Trimestre}
\subsection*{Chapitre 1 : Second degré : étude de fonction}
\paragraph*{Début le 4 Septembre}
\begin{enumerate}
\item Fonctions affines : courbe, expression algébrique, variation, calcul du coefficient directeur, ordonnée à l'origine (2h)
\item Fonctions polynomiales de degré 2 : expression algébrique, courbe, variation (2h)
\item Forme canonique : expression, méthode de calcul (2h)
\item Forme canonique : extremum (2h, Evaluation)
\end{enumerate}
\paragraph*{Fin le 15 Septembre}
\subsection*{Chapitre 2 : Dérivation locale}
\paragraph*{Début le 22 Septembre}
\begin{enumerate}
\item Notions de sécantes, taux de variation (2h)
\item Tangente, Nombre dérivé (2h)
\item Nombre dérivé de fonctions classiques (2h)
\item Controle 
\end{enumerate}
\paragraph*{Fin le 2 Octobre}
\subsection*{Chapitre 3 : Second degré : équations et inéquations}
\subsection*{Chapitre 4 : Fonction dérivées}
\section{Second Trimestre}
\begin{enumerate}[resume]
\item Dérivation
\item Trigonométrie
\item Produit scalaire
\item Géométrie
\end{enumerate}
\section{Troisième Trimestre}
\begin{enumerate}[resume]
\item Probabilités Conditionnelles
\item Variables aléatoires
\end{enumerate}
\end{document}