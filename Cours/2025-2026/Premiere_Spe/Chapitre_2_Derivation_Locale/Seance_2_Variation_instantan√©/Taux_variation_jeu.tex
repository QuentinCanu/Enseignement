\documentclass{exos}
\usepackage{main}

\title{\vspace{-2cm}Jeu de la variation}
\date{}
\author{Première Spécialité Mathématiques}

\begin{document}
\maketitle
\begin{exercize*}
\hfill

\begin{center}
\begin{tikzpicture}
\draw[help lines] (-7.25,-4.25) grid[step=0.5] (7.25,4.25);
\draw[axis] (-7.25,0) -- (7.25,0) node[right] {$x$};
\draw[axis] (0,-4.25) -- (0,4.25) node[above] {$y$};
\draw (0,0) node[below left] {$O$};
\draw (1,0.1) -- (1,-0.1) node[below] {$1$};
\draw (0.1,1) -- (-0.1,1) node[left] {$1$};
\end{tikzpicture}
\end{center}

\paragraph*{Règles du jeu}

\begin{alphaquestions}
\item Choisissez lequel/laquelle d'entre vous est le/la joueur/se $1$ et le/la joueur $2$. Le rôle de n°1 est de tracer des courbes représentatives de fonctions, tandis que le rôle de n°2 est de tracer des droites sécantes à cette courbe.
\item N°1 choisit un nombre $a$ dans l'intervalle $[-7;0]$ qu'il note dans l'encadré suivant :
\begin{equation*}
\tcboxmath{a = \quad} 
\end{equation*}
\item \textbf{Au crayon de papier}, n°1 trace la courbe représentative d'une fonction \og un peu complexe \fg.
\item\label{Alice} \textbf{au crayon de papier}, n°2 trace une droite sécante à la courbe à l'abscisse $a$ puis par un autre point $b > a$. \textbf{Sur l'intervalle $[a;b]$, la fonction associée à la courbe doit être monotone}.
\item\label{Bob} N°1 efface alors la courbe, et en redessine une autre de sorte que la droite de n°2 soit toujours sécante à la courbe aux points d'abscisse $a$ et $b$. \textbf{Sur l'intervalle $[a;b]$, la fonction représentée par la courbe ne doit pas être monotone}.
\item Continuer le jeu en répétant les étapes \ref{Alice} et \ref{Bob}. Le jeu peut-il continuer indéfiniment ? Que remarquez-vous concernant le nombre $b$ ?
\end{alphaquestions}

\end{exercize*}
\end{document}