\documentclass{exos}
\usepackage{main}

\begin{document}
\begin{exercize*}
Pour chacune des quantités suivantes dépendant de $h$, dire s'il existe une limite finie pour ces quantités quand $h$ tend vers $0$. Donner la valeur de cette limite le cas échéant.
\begin{alphaquestions}
\item $10 - h + h^2$
\item $\dfrac{1}{\sqrt{h}}$
\item $\dfrac{h}{\sqrt{h}}$
\item $\sqrt{h - 2}$
\end{alphaquestions}
\end{exercize*}
\vspace*{1.5cm}
\begin{exercize*}
Pour chacune des quantités suivantes dépendant de $h$, dire s'il existe une limite finie pour ces quantités quand $h$ tend vers $0$. Donner la valeur de cette limite le cas échéant.
\begin{alphaquestions}
\item $10 - h + h^2$
\item $\dfrac{1}{\sqrt{h}}$
\item $\dfrac{h}{\sqrt{h}}$
\item $\sqrt{h - 2}$
\end{alphaquestions}
\end{exercize*}
\vspace*{1.5cm}
\begin{exercize*}
Pour chacune des quantités suivantes dépendant de $h$, dire s'il existe une limite finie pour ces quantités quand $h$ tend vers $0$. Donner la valeur de cette limite le cas échéant.
\begin{alphaquestions}
\item $10 - h + h^2$
\item $\dfrac{1}{\sqrt{h}}$
\item $\dfrac{h}{\sqrt{h}}$
\item $\sqrt{h - 2}$
\end{alphaquestions}
\end{exercize*}
\end{document}