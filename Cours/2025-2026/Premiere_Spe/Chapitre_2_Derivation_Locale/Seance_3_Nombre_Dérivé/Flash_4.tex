\documentclass{exos}
\usepackage{main}

\begin{document}
\begin{exercize*}
Soit $f \colon x \mapsto \sqrt{x^2 + 9}$.
\begin{alphaquestions}
\item Justifier que cette fonction est définie sur $\R$.
\item Calculer le taux d'acroissement de $f$ entre $0$ et $4$. 
\end{alphaquestions}
\end{exercize*}
\vspace*{3cm}
\begin{exercize*}
Soit $f \colon x \mapsto \sqrt{x^2 + 9}$.
\begin{alphaquestions}
\item Justifier que cette fonction est définie sur $\R$.
\item Calculer le taux d'acroissement de $f$ entre $0$ et $4$. 
\end{alphaquestions}
\end{exercize*}
\vspace*{3cm}
\begin{exercize*}
Soit $f \colon x \mapsto \sqrt{x^2 + 9}$.
\begin{alphaquestions}
\item Justifier que cette fonction est définie sur $\R$.
\item Calculer le taux d'acroissement de $f$ entre $0$ et $4$. 
\end{alphaquestions}
\end{exercize*}
\vspace*{3cm}
\begin{exercize*}
Soit $f \colon x \mapsto \sqrt{x^2 + 9}$.
\begin{alphaquestions}
\item Justifier que cette fonction est définie sur $\R$.
\item Calculer le taux d'acroissement de $f$ entre $0$ et $4$. 
\end{alphaquestions}
\end{exercize*}
\vspace*{3cm}
\end{document}