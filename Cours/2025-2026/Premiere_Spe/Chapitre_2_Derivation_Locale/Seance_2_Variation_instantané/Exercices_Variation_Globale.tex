\documentclass{exos}
\usepackage{main}

\begin{document}
\begin{exercize}
Pour chacune des fonctions $f$ suivantes, ainsi que pour chaque $a$ et $b$ donné, calculer le taux d'accroissement de $f$ entre $a$ et $b$:
\begin{alphaquestions}
\item $f : x \mapsto 3x - 4$; $a = 0$ et $b = 5$
\item $f : x \mapsto x^2 + 2x + 6$; $a = -2$ et $b = 4$
\item $f : x \mapsto \sqrt{x + 1}$; $a = 10$ et $b = 15$
\end{alphaquestions}
\end{exercize}
\vspace*{1cm}
\begin{exercize}
Sans le calculer, dire s'il est possible de déterminer le signe des taux de variations suivants. Si oui, donner le signe correspondant.
\begin{alphaquestions}
\item Taux de variation de $f \colon x \mapsto 4(x-1)^2 + 3$ entre $a = 2$ et $b = 5$
\item Taux de variation de $f \colon x \mapsto x^2 + 6x + 8$ entre $a = -5$ et $b = 10$
\item Taux de variation de $f \colon x \mapsto -7x^2 -14x - 7$ entre $a = -3$ et $b = -2$
\end{alphaquestions}
\end{exercize}
\vspace*{1cm}
\begin{exercize}
Soit $h \neq 0$. Pour chacune des fonctions $f$ suivantes, et chaque réel $a$ donné, exprimer le taux de variation de $f$ entre $a$ et $a + h$ en fonction de $h$.
\begin{alphaquestions}
\item $f : x \mapsto -6x + 2$ et $a = 1$
\item $f : x \mapsto -3x^2 + 8$, $a = 4$
\end{alphaquestions}
\end{exercize}
\vspace*{1cm}
\begin{exercize}
Soit $f : px + q$ une fonction affine, et $a < b$ deux nombres réels. Montrer que le taux de variaition de $f$ entre $a$ et $ b$ vaut $p$.
\end{exercize}
\end{document}