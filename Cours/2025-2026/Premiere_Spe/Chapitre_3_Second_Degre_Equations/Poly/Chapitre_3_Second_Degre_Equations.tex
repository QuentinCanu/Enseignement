\documentclass{poly}
\usepackage{main}

\title{Chapitre 3 : Équations et inéquations du second degré}
\author{Première Spécialité Mathématiques}
\date{}


\begin{document}
\maketitle

\section{Racines}

\subsection{Définition}

\begin{definition}
Soit $f$ une fonction. On appelle \textbf{racine} de la fonction $f$ un nombre $r$ tel que $f(r) = 0$.
\end{definition}
\begin{exercize*}
Vérifier que $r_1 = 1$ et $r_2 = - 3$ sont deux racines de la fonction $f : x \mapsto 2x^2 + 4x - 6$.
\end{exercize*}
\vspace*{2cm}

\begin{proposition}
Soit $f : x \mapsto ax^2 + bx + c$ une fonction polynomiale du second degré. Alors, seuls trois cas sont à considérer :
\begin{alphaquestions}
\item $f$ n'admet aucune racine réelle, c'est-à-dire que pour tout réel $x$, on a $f(x) \neq 0$.
\item $f$ admet une unique racine notée $r$. Dans ce cas, $f$ peut être factorisée en $f(x) = a (x - r)^2$ pour tout $x$.
\item $f$ admet deux racines, notées $r_1$ et $r_2$. Dans ce cas, $f$ peut être factorisée en $f(x) = a(x - r_1)(x - r_2)$ pour tout $x$. 
\end{alphaquestions} 
\end{proposition}
\begin{exercize*}
Sur le repère suivant, tracer la courbe représentative de trois fonctions polynomiale du second degré correspondant à chacun des cas exposés dans la proposition précédente.
\begin{center}
\begin{tikzpicture}
\newcommand{\Le}{-3.25};
\newcommand{\Ri}{3.25};
\newcommand{\To}{3.25};
\newcommand{\Bo}{-3.25};
\coordinate (LB) at (\Le,\Bo);
\coordinate (RT) at (\Ri,\To);
\draw[help lines] (LB) grid[step = 0.5] (RT);
\draw[->,axis] (\Le,0) -- (\Ri,0) node[right] {$x$};
\draw[->,axis] (0,\Bo) -- (0,\To) node[above] {$y$};
\end{tikzpicture}
\end{center}
\end{exercize*}
\subsection{Calcul des racines}
\subsubsection{En identifiant une racine évidente}
\begin{exercize*}
Soit $f(x) = -x^2 + 6x$ pour $x \in \R$. Cette fonction possède-t-elle des racines évidentes ? Essayer avec des entiers comme $0; 1; -1; \dots$. 
\end{exercize*}
\vspace*{2cm}
\subsubsection{En utilisant une identité remarquable}
\begin{exercize*}
Soit $f(x)= 2x^2 - 128$ pour $x \in \R$.
\begin{alphaquestions}
\item Factoriser $f(x)$ par $2$.
\item À l'aide d'une identité remarquable bien choisie, factoriser $f(x)$.
\item En déduire les racines de $f(x)$.
\end{alphaquestions}
\end{exercize*}
\vspace*{2cm}

\subsubsection{Avec le produit et la somme des racines}
\begin{proposition}
Soit $f : x \mapsto ax^2 + bx + c$ une fonction polynomiale du second degré. Si $r_1$ et $r_2$ sont les deux racines (possiblement confondues) de $f$, alors
\begin{equation*}
r_1 + r_2 = \dfrac{-b}{a} \qquad r_1 \times r_2 = \dfrac{c}{a}
\end{equation*}
\end{proposition}
\begin{example}
Soit $f(x) = x^2 + x - 20$. On remarque que $4$ est une racine de $f$. En déduire une autre racine de $f$, puis une factorisation de $f$.
\vspace*{0.5cm}

\end{example}
\newpage
\subsection{Discriminant}
\begin{definition}
Soit $f : x \mapsto ax^2 + bx + c$ une fonction polynomiale du second degré. Alors on appelle \textbf{discriminant de f}, noté $\Delta$, la quantité
\begin{equation*}
\Delta = b^2 - 4ac
\end{equation*} 
\end{definition}
\begin{theorem}
Soit $f : x \mapsto ax^2 + bx + c$ une fonction polynomiale de second degré, et $\Delta$ son discriminant. Alors:
\begin{alphaquestions}
\item Si $\Delta < 0$, alors $f$ n'admet pas de racine réelle.
\item Si $\Delta = 0$, alors $f$ admet une unique racine réelle $r$, telle que
\begin{equation*}
r = - \dfrac{b}{2a}
\end{equation*}
\item Si $\Delta > 0$, alors $f$ admet deux racines réelles distinctes $r_1 < r_2$, telles que
\begin{equation*}
r_1 = \dfrac{- b - \sqrt{\Delta}}{2a} \qquad r_2 = \dfrac{- b + \sqrt{\Delta}}{2a}
\end{equation*}
\end{alphaquestions}
\end{theorem}
\paragraph{Démonstration}
\hfill
\newpage
\section{Signe}
\begin{proposition}
Soit $f : x \mapsto ax^2 + bx + c$ une fonction polynomiale du second degré. Alors:
\begin{alphaquestions}
\item Si $f$ n'admet pas de racine, alors $f$ est du même signe que $a$ sur $\R$.
\item Si $f$ admet une unique racine $r$, alors $f$ est du même signe que $a$ sur $\left]-\infty;r\right[$ et sur $\left]r;+\infty\right[$.
\item Si $f$ admet deux racines distinctes $r_1 < r_2$, alors $f$ est du même signe que $a$ sur $\left]-\infty;r_1\right[$ et sur $\left]r_2;+\infty\right[$, et est du signe opposé à $a$ sur $\left]r_1;r_2\right[$
\end{alphaquestions}
\end{proposition}
\begin{remark}
Une phrase pour retenir cette proposition :
\begin{tcolorbox}
\begin{quote}
Une fonction polynomiale du second degré est du même signe que $a$ à \textbf{l'extérieur} de ses racines, et est de signe opposé à $a$ à \textbf{l'intérieur} de ses racines.
\end{quote}
\end{tcolorbox}
\end{remark}
\begin{example}
En reprenant l'exemple précédent, donner le tableau de signes des fonctions $f$, $g$ et $h$.
\end{example}

\end{document}