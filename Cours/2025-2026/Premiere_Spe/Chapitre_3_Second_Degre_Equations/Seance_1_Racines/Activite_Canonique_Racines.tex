\documentclass{exos}
\usepackage{main}

\title{Forme canonique : Le retour}
\author{Première Spécialité Mathématiques}
\date{}


\begin{document}
\maketitle
\section{Déjà vu}
Soit la fonction polynomiale du second degré $f \colon x \mapsto x^2 + 4x + 5$.

\begin{alphaquestions}
\item Exprimer $f(x)$ sous forme canonique.
\item L'équation $f(x) = 0$ admet-elle des solutions ?
\end{alphaquestions}
\section{Retour case départ}
Soit la fonction polynomiale du second degré $g \colon x \mapsto x^2 - 8x + 16$.
\begin{alphaquestions}
\item Exprimer $g(x)$ sous forme canonique. On pourra utiliser une identité remarquable bien choisie.
\item Montrer que l'équation $g(x) = 0$ admet une solution. En admet-t-elle d'autres ?
\end{alphaquestions} 
\section{Finalement\dots}
Soit la fonction polynomiale du second degré $h \colon x \mapsto x^2 - 2x - 3$.
\begin{alphaquestions}
\item Exprimer $h(x)$ sous forme canonique.
\item L'équation $h(x)=0$ admet-t-elle des solutions ?
\item Proposer une forme factorisée de $h(x)$. En déduire les solutions de l'équation $h(x) = 0$.
\end{alphaquestions}


\end{document}