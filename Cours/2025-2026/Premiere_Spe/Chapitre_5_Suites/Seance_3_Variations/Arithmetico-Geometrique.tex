\documentclass{controle}
\usepackage{main}

\title{Exercice : Suite Arithmético-Géométrique}
\author{Première Spécialité Mathématiques}
\date{Chapitre 5}

\begin{document}
\titleandname{}

Soit $(u_n)$ la suite définie pour $n \in \N$ par :
\begin{equation*}
\begin{cases}
u_0 &= 14\\
u_{n+1} &= 2u_n - 8
\end{cases}
\end{equation*}
\begin{parts}
\part Donner les valeurs de $u_0$, $u_1$ et $u_2$.
\part La suite $(u_n)$ est-elle arithmétique ? Est-elle géométrique ?
\part On pose $(w_n)$ la suite définie pour $n \in \N$ par
\begin{equation*}
w_n = u_n - 8
\end{equation*}
Calculer $w_0$, $w_1$ et $w_2$.
\part Démontrer que $(w_n)$ est une suite géométrique dont on précisera la raison.
\part En déduire une expression de $w_n$ en fonction de $n$.
\part Donner la valeur de $w_{15}$.
\part Donner une expresison de $u_n$ en fonction de $n$.
\part En déduire une valeur de $u_{17}$.
\end{parts}
\end{document}