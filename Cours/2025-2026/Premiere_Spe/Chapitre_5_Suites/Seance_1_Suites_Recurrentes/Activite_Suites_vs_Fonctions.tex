\documentclass{exos}
\usepackage{main}

\title{Modélisation discrète ou continue ?}
\author{Première Spécialité Mathématiques}
\date{}

\begin{document}
\maketitle
\begin{tcolorbox}
Pour chacune des situations suivantes, dire s'il est possible d'utiliser une fonction pour modéliser le phénomène décrit. Si oui, proposer une fonction plausible : on précisera l'ensemble de définition et l'ensemble image de la fonction. 
\end{tcolorbox}
\begin{alphaquestions}
\item Le prix d'un litre d'essence est de \num{1,609} €. On s'intéresse au prix dépensé \textbf{en fonction de} la quantité achetée, en litres.
\item Le chiffre d'affaire d'une banque vérifie un phénomène intéressant: chaque année, le 1\ier{} Janvier, ce chiffre représente \num{95}\% du chiffre d'affaire au 1\ier{} Janvier de l'année précédente. On s'intéresse au chiffre d'affaire du 1\ier{} Janvier \textbf{en fonction de} l'année $x$ donnée.
\item On fait chauffer l'extrémité gauche d'une barre en métal de \qty{10}{\centi\meter}, tandis que l'extrémité droite est laissée à l'air libre. On s'intéresse à la température de la barre \textbf{en fonction de} la distance $x$ par rapport à l'extrémité gauche.
\item Les performances d'une sprinteuse sont chronométrées chaque jour. On s'intéresse au temps obtenu par la sprinteuse \textbf{en fonction du} numéro du jour.
\item On compte le nombre de possibilité de ranger ma collection de figurines numérotées. On s'intéresse à ce nombre de possibilités \textbf{en fonction du} nombre de figurines dans ma collection.
\item On mesure l'intensité d'un dipôle \textbf{en fonction de} sa tension.
\item On cherche à savoir la valeur en bourse d'une action d'une célèbre entreprise de supermarchés \textbf{en fonction de} l'instant.
\item On verse de l'eau le long d'une pente. On s'interesse à la vitesse en bas de la pente \textbf{en fonction de} la quantité d'eau versée (en litres).
\item On verse de l'eau le long d'une pente. On s'interesse à la vitesse en bas de la pente \textbf{en fonction du} nombre de bouteilles d'eau versées.
\end{alphaquestions}
\end{document}