\documentclass{exos}
\usepackage{main}

\begin{document}
On souhaite mesurer la portée d'une catapulte. Pour cela, on installe la catapulte au début d'une piste graduée, et l'on projette une pierre. La pierre atterrit \qty{300}{\meter} plus loin, sur la graduation \num{300}.

\begin{alphaquestions}
\item Soit un repère orthonormée $(O;I;J)$. On assimile la piste graduée à l'axe des abscisses, et le point $O$ à l'emplacement de la catapulte. Tracer l'allure de la trajectoire de la pierre.
\item Quand elle était à son altitude maximale, au-dessus de quelle graduation la pierre se trouvait ?
\item Soit $y = ax^2 + bx + c$ la formule de l'altitude $y$ de la pierre en fonction de la graduation $x$ qu'elle survole. On admet que $a = -10$. En déduire l'altitude maximale atteinte par la pierre.
\end{alphaquestions}

\vspace*{2cm}
On souhaite mesurer la portée d'une catapulte. Pour cela, on installe la catapulte au début d'une piste graduée, et l'on projette une pierre. La pierre atterrit \qty{300}{\meter} plus loin, sur la graduation \num{300}.

\begin{alphaquestions}
\item Soit un repère orthonormée $(O;I;J)$. On assimile la piste graduée à l'axe des abscisses, et le point $O$ à l'emplacement de la catapulte. Tracer l'allure de la trajectoire de la pierre.
\item Quand elle était à son altitude maximale, au-dessus de quelle graduation la pierre se trouvait ?
\item Soit $y = ax^2 + bx + c$ la formule de l'altitude $y$ de la pierre en fonction de la graduation $x$ qu'elle survole. On admet que $a = -10$. En déduire l'altitude maximale atteinte par la pierre.
\end{alphaquestions}

\vspace*{2cm}
On souhaite mesurer la portée d'une catapulte. Pour cela, on installe la catapulte au début d'une piste graduée, et l'on projette une pierre. La pierre atterrit \qty{300}{\meter} plus loin, sur la graduation \num{300}.

\begin{alphaquestions}
\item Soit un repère orthonormée $(O;I;J)$. On assimile la piste graduée à l'axe des abscisses, et le point $O$ à l'emplacement de la catapulte. Tracer l'allure de la trajectoire de la pierre.
\item Quand elle était à son altitude maximale, au-dessus de quelle graduation la pierre se trouvait ?
\item Soit $y = ax^2 + bx + c$ la formule de l'altitude $y$ de la pierre en fonction de la graduation $x$ qu'elle survole. On admet que $a = -10$. En déduire l'altitude maximale atteinte par la pierre.
\end{alphaquestions}

\vspace*{2cm}
\end{document}