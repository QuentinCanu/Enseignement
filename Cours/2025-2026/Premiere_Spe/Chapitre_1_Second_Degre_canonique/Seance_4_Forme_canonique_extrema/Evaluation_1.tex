\documentclass{controle}
\usepackage{main}

\title{Evaluation n°1 : Fonctions affines et fonction polynomiales du second degré}
\author{Première Spécialité Mathématiques}
\date{15 Septembre 2025}

\begin{document}
\maketitle
\begin{questions}
\titledquestion{Fonctions affines}[4]
Soit $f$ une fonction affine dont la courbe représentative est appelée $\mathcal{C}_f$. Dans chacun des cas suivants, donner le coefficient directeur et l'ordonnée à l'origine de $f$. \textbf{Justifier brièvement votre réponse.}
\vspace*{0.5cm}
\begin{parts}
\begin{minipage}{0.45\textwidth}
\part\hfill

\begin{tikzpicture}
\draw[help lines] (-3.25,-3.25) grid[step=0.5] (3.25,3.25);
\draw[axis, ->] (-3.25,0) -- (3.25,0) node[right] {$x$}; 
\draw[axis, ->] (0,-3.25) -- (0,3.25) node[above] {$y$};
\draw (0.5,0.1) -- (0.5,-0.1) node[below] {$1$};
\draw (0.1,0.5) -- (-0.1,0.5) node[left] {$1$};
\clip (-3.25,-3.25) rectangle (3.25,3.25);
\draw (-4,-1) -- (4,3);
\end{tikzpicture}
\end{minipage}
\begin{minipage}{0.45\textwidth}
\part\hfill

\begin{tikzpicture}
\draw[help lines] (-3.25,-3.25) grid[step=0.5] (3.25,3.25);
\draw[axis, ->] (-3.25,0) -- (3.25,0) node[right] {$x$}; 
\draw[axis, ->] (0,-3.25) -- (0,3.25) node[above] {$y$};
\draw (0.5,0.1) -- (0.5,-0.1) node[below] {$1$};
\draw (0.1,0.5) -- (-0.1,0.5) node[left] {$1$};
\clip (-3.25,-3.25) rectangle (3.25,3.25);
\draw (-4,-0.5) -- (4,-2.5); 
\end{tikzpicture}
\end{minipage}
\end{parts}
\vspace*{1cm}
\titledquestion{Fonctions polynomiales du second degré}[6]
Pour chacune des fonctions suivantes:
\begin{subparts}
\subpart Décrire l'allure de sa courbe représentative;
\subpart Calculer sa forme canonique.
\end{subparts}
\begin{parts}
\part $f(x) = x^2 - 6x + 4$
\part $g(x) = -4x^2 + 24x - 5$
\end{parts}
\end{questions}

\newpage
\maketitle

\begin{questions}
\titledquestion{Fonctions affines}[4]
Soit $f$ une fonction affine dont la courbe représentative est appelée $\mathcal{C}_f$. Dans chacun des cas suivants, donner le coefficient directeur et l'ordonnée à l'origine de $f$. \textbf{Justifier brièvement votre réponse.}
\vspace*{0.5cm}
\begin{parts}
\begin{minipage}{0.45\textwidth}
\part\hfill

\begin{tikzpicture}
\draw[help lines] (-3.25,-3.25) grid[step=0.5] (3.25,3.25);
\draw[axis, ->] (-3.25,0) -- (3.25,0) node[right] {$x$}; 
\draw[axis, ->] (0,-3.25) -- (0,3.25) node[above] {$y$};
\draw (0.5,0.1) -- (0.5,-0.1) node[below] {$1$};
\draw (0.1,0.5) -- (-0.1,0.5) node[left] {$1$};
\clip (-3.25,-3.25) rectangle (3.25,3.25);
\draw (-4,4) -- (4,0);
\end{tikzpicture}
\end{minipage}
\begin{minipage}{0.45\textwidth}
\part\hfill

\begin{tikzpicture}
\draw[help lines] (-3.25,-3.25) grid[step=0.5] (3.25,3.25);
\draw[axis, ->] (-3.25,0) -- (3.25,0) node[right] {$x$}; 
\draw[axis, ->] (0,-3.25) -- (0,3.25) node[above] {$y$};
\draw (0.5,0.1) -- (0.5,-0.1) node[below] {$1$};
\draw (0.1,0.5) -- (-0.1,0.5) node[left] {$1$};
\clip (-3.25,-3.25) rectangle (3.25,3.25);
\draw (-4,-3) -- (4,3); 
\end{tikzpicture}
\end{minipage}
\end{parts}
\vspace*{1cm}
\titledquestion{Fonctions polynomiales du second degré}[6]
Pour chacune des fonctions suivantes:
\begin{subparts}
\subpart Décrire l'allure de sa courbe représentative;
\subpart Calculer sa forme canonique.
\end{subparts}
\begin{parts}
\part $f(x) = x^2 + 2x + 7$
\part $g(x) = 12x^2 - 48x - 9$
\end{parts}
\end{questions}
\end{document}