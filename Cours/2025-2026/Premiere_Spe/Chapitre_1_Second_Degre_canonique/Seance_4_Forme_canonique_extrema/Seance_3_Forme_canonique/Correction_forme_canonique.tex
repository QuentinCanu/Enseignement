\documentclass{article}
\usepackage{main}

\title{Exemple de passage à la forme canonique}
\author{}
\date{}

\begin{document}
\maketitle

\begin{tcolorbox}
On présente les deux méthodes pour passer la fonction polynomiale du second degré $f \colon x \mapsto 2x^2 + 20x + 45$.
\end{tcolorbox}
\paragraph*{Méthode par identification}
On utilise les formules du cours pour déterminer $\alpha$ et $\beta$. Ici, 
\begin{equation*}
\begin{cases}
a = 2\\
b = 20\\
c = 45
\end{cases}
\end{equation*}

On en déduit que 
\begin{equation*}
\begin{aligned}
\alpha &= \dfrac{-b}{2a}\\
&= \dfrac{-(20)}{2 \times (2)}\\
&= \dfrac{-20}{4}\\
&= -5
\end{aligned}  
\end{equation*}

Ainsi que
\begin{equation*}
\begin{aligned}
\beta &= f(\alpha)\\
&= 2\alpha^2 + 20\alpha + 45\\
&= 2 \times (-5)^2 + 20 \times (-5) + 45\\
&= 50 - 100 + 45\\
&= -5\\
\end{aligned}
\end{equation*}
On remplace ainsi les valeurs correspondante pour conclure que
\begin{equation*}
f(x) = a(x - \alpha)^2 + \beta = 2(x - (-5))^2 + (-5) = 2(x+5)^2 - 5
\end{equation*}
\paragraph*{Méthode par identité remarquable \og limitée \fg}
On s'inspire de la démonstration vue en cours. Soit $x \in \R$.

Alors,
\begin{align*}
f(x) &= 2x^2 + 20x + 45\\
&= 2(\dfrac{2}{2}x^2 + \dfrac{20}{2}x) + 45 \tag{On factorise par $a$}\\
&= 2(x^2 + 10x) + 45\\ 
\end{align*}
L'expression $x^2 + 10x = x^2 + 2 \times x \times 5$ peut être reconnue comme le début d'une identité remarquable $p^2 + 2pq + q^2$ avec $p = x$ et $q = 5$. On sait que $x^2 + 2 \times x \times 5 + 5^2 = (x + 5)^2$, et par conséquent,
\begin{equation*}
x^2 + 2 \times x \times 5 = (x + 5)^2 - 5^2
\end{equation*}
On utilise cette formule pour effectuer une substitution dans l'expression de $f(x)$:
\begin{align*}
f(x) &= 2(x^2 + 10x) + 45\\
&= 2((x + 5)^2 - 5^2) + 45\\
&= 2((x + 5)^2 - 25) + 45\\
&= 2(x+5)^2 - 50 + 45 \tag{Par distributivité}\\
&= 2(x+5)^2 - 5
\end{align*}
Ainsi, les deux méthodes permettent d'aboutir à la forme canonique de la fonction polynomiale du second degré $f$.
\end{document}