\documentclass{exos}
\usepackage{main}

\title{Production}
\date{8 Septembre 2025}
\author{Première Spécialité Mathématiques}

\begin{document}
\maketitle

\begin{tcolorbox}
Un constructeur automobile décide de commercialiser des voitures à bas coût : chaque voiture doit être vendue \qty{6000}{\text{€}}. Sa production $q$ peut varier entre \num{0} et \num{100} milliers de voitures. Suite à une étude réalisée, les coûts de production (en millions d'euros) sont donnés par la formule suivante :
\begin{equation*}
C(q) = 0,05q^2 + q + 20
\end{equation*}
où $q$ est exprimé en millier d'unités.
\end{tcolorbox}
\begin{alphaquestions}
\item Exprimer la recette $R(q)$ en fonction de la production $q$. On exprimera $R(q)$ en millions d'euros.
\item En déduire, en fonction de $q$, le bénéfice de l'entreprise $B(q)$.
\item Vérifier que $B(q) = -0,05(q - 50)^2 + 45$.
\item Dans quel intervalle doit se situer la quantité de voitures produites pour réaliser un bénéfice positif ?
\item Quel est le nombre d'automobiles à produire pour obtenir un bénéfice maximal ?
\end{alphaquestions}
\end{document}