\documentclass{article}
\usepackage{main}

\title{Présentation de rentrée}
\date{4 Septembre 2025}
\author{Quentin Canu}

\begin{document}
\maketitle
\section{Entrée en classe}
\begin{itemize}
\item Appel par ordre alphabétique (prévoir une liste d'élèves);
\item Rangement des téléphones dans la malette dans l'ordre
\end{itemize}
\section{Vérification de l'emploi du temps}
\section{Explication du matériel}
\begin{itemize}
\item 2 Cahiers $24 \times 32$: cours + exercice. On collera la feuille du jour sur une page de gauche, la page de droite servira à prendre des notes.
\item Calculatrice
\end{itemize}
\section{Explication des notes}
\begin{itemize}
\item Évaluations, coefficient $0,5$: tous les lundi, dure un quart d'heure, exercices déjà vus en cours, commence le 15 Septembre.
\item Contrôles, coefficient $2$: toutes les trois semaines/un mois: le premier est prévu le 2 Octobre.
\end{itemize}
\section{Règles du cours}
\begin{enumerate}
\item En cours de maths, on fait des maths.
\item Une seule personne à la fois prend la parole.
\item Si je ne comprends pas, je dois demander de l'aide.
\end{enumerate}
\end{document}