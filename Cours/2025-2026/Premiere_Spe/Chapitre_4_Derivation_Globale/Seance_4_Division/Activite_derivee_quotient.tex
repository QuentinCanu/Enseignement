\documentclass{exos}
\usepackage{main}

\title{Dérivée d'un quotient}
\date{27 Novembre 2025}
\author{Premières Spécialité Mathématiques}

\begin{document}
\maketitle
\begin{tcolorbox}
Durant cette activité, on ne demande pas les ensembles de dérivabilité des fonctions étudiées.
\end{tcolorbox}
\section{Introduction}
On donne la formule de la dérivée de l'inverse d'une fonction $v$:
\begin{equation*}
\left(\dfrac{1}{v}\right)' = - \dfrac{v'}{v^2}  
\end{equation*}
\begin{alphaquestions}
\item Tester sur la fonction $f \colon x \mapsto \dfrac{1}{x}$.
\item Calculer la fonction dérivée de $g \colon x \mapsto \dfrac{1}{x^2}$ et $h \colon x \mapsto \dfrac{1}{\sqrt{x}}$.
\item Calculer la fonction dérivée de $p \colon x \mapsto \dfrac{1}{3x^2 - 2x + 1}$ 
\end{alphaquestions}
\section{Dérivée de la division}
On souhaite démontrer la règle de dérivation d'un quotient de fonctions. Pour cela, on pose $u$ et $v$ deux fonctions.

En remarquant que $\left(\dfrac{u}{v}\right) = \left(u \times \dfrac{1}{v}\right)$, en déduire que
\begin{equation*}
\left(\dfrac{u}{v}\right)' = \dfrac{u'v - uv'}{v^2}
\end{equation*}
\section{Application}
En déduire la fonction dérivée de chacune des fonctions suivantes:
\begin{alphaquestions}
\item $f \colon x \mapsto \dfrac{x^2 - 2x + 1}{5x + 1}$
\item $g \colon x \mapsto \dfrac{2x + 11}{-x^3 + 4x^2 - 5x + 7}$
\item $h \colon x \mapsto \dfrac{\sqrt{x}}{x^3}$
\end{alphaquestions}
\end{document}