\documentclass{exos}
\usepackage{main}

\title{Vers la fonction dérivée}
\date{17 Novembre 2025}
\author{Première Spécialité Mathématiques}


\begin{document}
\maketitle

\begin{exercize}
Soit $f \colon x \mapsto x^3$. 
\begin{alphaquestions}
\item Calculer le nombre dérivé de $f$ en $a = 2$. On utilisera l'identité remarquable suivante :
\begin{equation*}
(a + b)^3 = a^3 + 3a^2b + 3ab^2 + b^3 
\end{equation*}
\item Calculer le nombre dérivé de $f$ en $a = 3$.
\item Calculer le nombre dérivé de $f$ pour un $a$ quelconque. Est-ce possible pour tout $a$ ?
\item En déduire une formule permettant de calculer immédiatement $f'(a)$ pour n'importe quel nombre $a$.
\end{alphaquestions}
\end{exercize}
\begin{exercize}
Soit $f \colon x \mapsto \sqrt{}$. 
\begin{alphaquestions}
\item Calculer le nombre dérivé de $f$ en $a = 2$.
\item Calculer le nombre dérivé de $f$ en $a = 0$. Est-ce possible ?
\item Calculer le nombre dérivé de $f$ pour un $a$ quelconque dans $\R_+$. Est-ce possible pour tout $a \in \R_+$ ?
\item En déduire une formule permettant de calculer immédiatement $f'(a)$ pour n'importe quel nombre $a \in \R_+$.
\end{alphaquestions}
\end{exercize}
\end{document}