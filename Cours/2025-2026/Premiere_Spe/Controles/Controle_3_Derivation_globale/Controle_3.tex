\documentclass{controle}
\usepackage{main}

\title{Contrôle n°3 : Dérivation globale}
\author{Première Spécialité Mathématiques}
\date{8 Janvier 2026}

\begin{document}
\instructions[Interdite]
\begin{questions}
\titledquestion{Automatismes}[6]
\begin{parts}
\part[3] Résoudre sur $\R$ les équations polynomiale du second degré suivantes:
\begin{subparts}
\subpart $x^2 - 7x + 12 = 0$
\subpart $-2x^2 +16x - 42 = 0$
\end{subparts}
\part[3] Dériver les fonctions suivantes sur les ensembles donnés, en justifiant correctement l'ensemble de dérivabilité des fonctions.
\begin{subparts}
\subpart $f : x \mapsto \sqrt{x}$
\subpart $g : x \mapsto 6x^4 - 3x^2 + 19x - 5$
\subpart $h : x \mapsto (x^3 - 5x)(x-2)$
\end{subparts}
\end{parts}
\titledquestion{Bottes}[7]
Un fabricant de chaussures fait un peu de comptabilité. Il vend chaque paire de chaussures \num{201}€. On appelle $C(x)$ le coût de production de $x$ paires de chaussures.

On définit, pour $x \in [0;30]$,
\begin{equation*}
C(x) = x^3 - 30x^2 + 309x + 500
\end{equation*}
\begin{parts}
\part[0.5] Justifier que le bénéfice $B(x)$ associé à la production et à la vente de $x$ paires de chaussures est défini par
\begin{equation*}
B(x)=-x^3+30x^2-108x-500
\end{equation*}
\part[0.5] Justifier que $B$ est dérivable sur $[0;30]$ et donner l'expression de sa dérivée $B'$.
\part[2] En déduire le tableau de variations de $B$.
\part[2] Combien de chaussures doit vendre ce fabricant afin de réaliser un bénéfice maximal ?
\part[2] On appelle coût marginal de production la fonction $C_m$ définie par, pour tout $x \in [0;30]$,
\begin{equation*}
C_m(x)=C'(x)
\end{equation*}
\begin{subparts}
\subpart Démontrer que pour tout $x \in [0;30]$, $C_m(x) = 3(x-10)^2+9$.
\subpart En déduire combien de paires doit produire le fabricant pour observer un coût marginal minimal.
\end{subparts}  
\end{parts}
\titledquestion{Courbes de Lorenz}[7]
On appelle \textbf{courbe de Lorenz} la courbe représentative d'une fonction $L$ vérifiant les propriétés suivantes:
\begin{enumerate}
\item \label{Lorenz_incr} $L$ est définie et croissante sur $[0;1]$;
\item \label{Lorenz_eq} $L(0) = 0$ et $L(1) = 1$;
\item \label{Lorenz_high} Pour tout $x \in [0;1]$, on a $L(x) \leq x$. 
\end{enumerate}
Soit $f : x \mapsto \dfrac{3}{2}x+\dfrac{1}{x+1}-1$, définie sur $[0;1]$. On souhaite montrer que $f$ est une courbe de Lorenz.
\begin{parts}
\part[1] Montrer que $f(0)=0$ et $f(1)=1$. En déduire que le critère \ref{Lorenz_eq} est respecté.
\part[1] On admet que la fonction est dérivable sur $[0;1]$. Montrer que pour tout $x \in [0;1]$, $f'(x)=\dfrac{3}{2}-\dfrac{1}{(x+1)^2}$.
\part[1] En déduire pour tout $x \in [0;1]$, on a $f'(x)=\dfrac{3(x+1)^2-2}{(x+1)^2}$.
\part[2] Dresser le tableau de signes de $f'$, et en déduire le tableau de variations de $f$. La fonction $f$ vérifie-t-elle le critère \ref{Lorenz_incr} ?
\part[1] Justifier que pour montrer que $f$ vérifie le critère \ref{Lorenz_high}, il suffit de vérifier que $f(x) - x \leq 0$ pour tout $x \in [0;1]$.
\part[2] On admet que pour montrer que $f(x) - x \leq 0$, il suffit de montrer que $x^2 - x \leq 0$ pour tout $x \in [0;1]$. Montrer alors que $f(x) - x \leq 0$. Conclure que $f$ vérifie le critère \ref{Lorenz_high}
\end{parts}
\end{questions}
\end{document}