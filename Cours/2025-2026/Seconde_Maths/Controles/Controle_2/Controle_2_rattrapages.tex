\documentclass{controle}
\usepackage{main}


\title{Contrôle n°2 : Évolutions}
\author{Seconde 3}
\date{12 Novembre 2025}

\begin{document}
\maketitle
\instructions[Autorisée]

\begin{questions}
\titledquestion{Géométrie repérée}[4]
Soit $(O;I;J)$ un repère orthonormé que l'on représente ci-dessous.
\begin{center}
\begin{tikzpicture}
\tikzmath{\x=4.25;};
\draw[help lines] (-\x,-\x) grid[step=0.5] (\x,\x);
\draw[axis] (-\x,0) -- (\x,0); 
\draw[axis] (0,-\x) -- (0,\x);
\draw (0,0) node[below left] {$O$};
\draw (1,0.1) -- (1,-0.1) node[below] {$I$};  
\draw (0.1,1) -- (-0.1,1) node[left] {$J$};

\draw (2,3) node[below left] {$A$} node {$\bullet$};
\draw (-2,-1) node[below left] {$B$} node {$\bullet$};
\draw (-3,1.5) node[below left] {$C$} node {$\bullet$}; 
\end{tikzpicture}
\end{center}
\begin{parts}
\part[2] Quelles sont les coordonnées des points $A$, $B$ et $C$ ?
\part[2] Placer les points suivants sur le repère:
\begin{subparts}
\subpart $D(-2;1)$
\subpart $E(0,5;1,5)$
\subpart $F$ a la même ordonnée que $A$ et la même abscisse que $B$.
\end{subparts}
\end{parts}

\titledquestion{Taux d'évolution}[6]
\begin{parts}
\part[1] Soit une quantité évoluant d'une valeur de départ $V_d$ vers une valeur finale $V_f$. Donner la formule du taux d'évolution $t$ de cette quantité.
\vspace*{0.5cm}

On joue à un jeu de rôle. La puissance de notre personnage est actuellement de $1246$. Heureusement, il est possible d'acheter un bonus parmi trois pour s'améliorer. 
\begin{enumerate}
\item Le premier bonus augmente la puissance de $300$. 
\item Le deuxième bonus permet d'augmenter la puissance de $15\%$.
\item Le troisième bonus est aussi une augmentation en pourcentage. Par exemple, durant la dernière partie, ce bonus a augmenté une puissance de $650$ en $767$.
\end{enumerate}
\part[2] Quel est le meilleur bonus disponible pour notre personnage ?
\part[2] Même question pour un personnage dont le niveau de puissance de départ est $1945$.
\part[1] Quel est le niveau de puissance de départ nécessaire pour que le premier et le deuxième bonus aient le même effet ?
\end{parts}
\vspace*{0.5cm}
\titledquestion{Évolutions successives}[5]
\begin{parts}
\part[1] Donner le taux d'évolution global correspondant à une augmentation de $+15\%$ puis une augmentation de $10\%$. 
\part[1] Donner le taux d'évolution global correspondant à une augmentation de $+36\%$ puis une diminution de $5\%$.
\part[3] Une boutique en ligne de chaussure souhaitant devenir rentable décide d'augmenter progressivement ses prix. Chaque mois, les prix augmentent de $12\%$.
\begin{subparts}
\subpart[1] De quel pourcentage les prix ont augmenté au bout de $1$ mois ? $2$ mois ?
\subpart[1] Montrer que le taux d'évolution des prix au bout de $4$ mois est d'environ $57\%$.
\subpart[1] L'entreprise décide de solder sa marchandise de $42\%$ après avoir le $4$\ieme{} mois. Quel est le taux d'évolution des prix après ces soldes ?
\end{subparts}
\end{parts}
\vspace*{0.5cm}
\titledquestion{Évolutions réciproques}[4]
\begin{parts}
\part[1] Quel est le taux d'évolution réciproque d'une diminution de $45\%$ ? On arrondira la réponse au centième près.
\part[3] Un chimiste teste la concentration de liquide dans différents mélanges :
\begin{subparts}
\subpart[1] Il commence avec le mélange $A$ et observe sa concentration initiale de chlorure de potassium. Il ajoute un premier réactif et observe alors une augmentation de $40\%$. Il sait que le deuxième réactif qu'il ajoute a pour effet de réduire de $35\%$ la concentration de chlorure de potassium. Le deuxième réactif va-t-il rendre à ce mélange sa concentration initiale ? Justifier votre réponse.
\subpart[2] Pour le mélange $B$, l'assistant du chimiste a déjà ajouté un premier réactif nommé \og reactium \fg. Par chance, le chimiste sait qu'en ajoutant un inhibiteur, la concentration du mélange diminuera de $39\%$, ce qui lui rendra sa concentration initiale. De quel pourcentage la concentration a augmenté après ajout du \og reactium \fg ?
\end{subparts}
\end{parts}
\end{questions}


\end{document}