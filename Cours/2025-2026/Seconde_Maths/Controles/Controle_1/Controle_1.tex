\documentclass{controle}
\usepackage{main}

\title{Contrôle n°1 : Calcul littéral, Proportions}
\author{Seconde 3}
\date{3 Octobre 2025}

\begin{document}
\maketitle
\instructions[Autorisée]

\vspace*{1cm}

\begin{questions}
\titledquestion{Proportions}[4]
Pour chacune des situations suivantes :
\begin{subparts}
\subpart Préciser la population et la sous-population à l'étude;
\subpart Calculer la valeur demandé, en précisant le calcul.
\end{subparts}
\begin{parts}
\part[1] Parmi les \num{180} abeilles d'une ruche, \num{45} d'entre elles sont des ouvrières. Quel est le pourcentage d'ouvrières dans cette ruche ?
\part[1] Dans un univers de science-fiction, \num{31}\% des personnages sont d'origine extraterrestre. On suppose qu'il y a \num{500} personnages dans cet univers, combien y a-t-il d'extraterrestres parmi eux ?
\part[2] J'ai participé au cadeau d'une amie à hauteur de \num{52}€, ce qui représente \num{13}\% du prix total du cadeau. Quel est le prix total du cadeau ?
\end{parts}
\vspace*{1cm}
\titledquestion{Équations}[4]
\begin{parts}
\part[2]
Résoudre dans $\R$ les équations suivantes :
\begin{subparts}
\subpart[1] $-6x + 9 = 18 - x$
\subpart[1] $x(x - 6) = (x + 10)(x - 1)$
\end{subparts}
\part[2] Une sortie scolaire coûte initialement \num{11}€ à tous les élèves. Cependant, le jour de la sortie, \num{4} élèves sont absents, et le prix de la sortie coûte \num{13}€ à tous les participants restants. En posant $n$ le nombre d'élèves initialement prévus, exprimer la situation sous forme d'équation et la résoudre dans $\R$.
\end{parts}

\vspace*{1cm}

\titledquestion{Carré d'entier successifs}[6]
\begin{parts}
\part[1] Vérifier les égalités suivantes :
\begin{equation*}
\begin{aligned}
3^2 - 2^2 &= 3 + 2\\
10^2 - 9^2 &= 10 + 9\\
25^2 - 24^2 &= 25 + 24\\ 
\end{aligned}
\end{equation*}
\part[1] On souhaite montrer l'affirmation suivante :
\begin{quote}
La différence des carrés de deux nombres entiers positifs correspond à la somme de ces deux entiers. 
\end{quote}
Pour cela, on pose $n$ un entier quelconque. Justifier que l'énoncé se traduit par l'égalité $(n+1)^2 - n^2 = (n + 1) + n$
\part[2] Démontrer que l'égalité précédente est vraie.
\part[2] Montrer de la même manière une méthode pour calculer facilement $(n + 1)^2 - (n - 1)^2$, et en déduire la valeur de $101^2 - 99^2$.
\end{parts}
\vspace*{1cm}
\titledquestion{Factorisation}[6]
\begin{parts}
\part[2] Factoriser les expressions suivantes, par la méthode du facteur commun ou à l'aide d'une identité remarquable :
\begin{subparts}
\subpart $48a + 52b$
\subpart $x^2 + 14x + 7$
\subpart $81p^2 - 45p$
\subpart $64 - m^2$
\end{subparts}
\part[2] On s'interresse à la méthode de résolution d'équations du second degré comme $x^2 + 4x - 5 = 0$.
\begin{subparts}
\subpart En développant, montrer que $x^2 + 4x - 5 = (x + 2)^2 - 9$.
\subpart En factorisant, montrer alors que $x^2 + 4x - 5 = (x + 5)(x - 1)$.
\end{subparts}
On en déduit dans ce cas que les solutions de l'équation $x^2 + 4x - 5 = 0$ sont $-5$ et $1$ (en prenant l'opposé des nombres apparaissant dans la forme factorisée).
\part[2] En s'inspirant de la méthode précédente, factoriser les expressions permettant de résoudre les équations suivantes. On utilisera l'indication sans avoir besoin de la justifier.
\begin{subparts}
\subpart $x^2 - 8x - 9 = 0$ (Indication : On a $x^2 - 8x - 9 = (x - 4)^2 - 25$)
\subpart $x^2 - 18x + 80 = 0$ (Indication : On a $x^2 - 18x + 80 = (x - 9)^2 - 1$)
\end{subparts}
\end{parts}
\end{questions}
\end{document}