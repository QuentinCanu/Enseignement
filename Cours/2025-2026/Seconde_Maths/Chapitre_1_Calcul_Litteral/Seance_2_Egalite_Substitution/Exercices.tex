\documentclass{article}
\usepackage{main}

\title{Conversion d'unités}
\author{Seconde 3}
\date{8 Septembre 2025}

\begin{document}
\maketitle
\begin{alphaquestions}
\item Rappeler la relation entre les mètres (\unit{\meter}) et les kilomètres (\unit{\kilo\meter}) en complétant l'égalité suivante :
\begin{equation*}
\qty{1}{\kilo\meter} = \dots
\end{equation*}
\item À l'aide d'une substitution, en déduire le nombre de mètres carrés (\unit{\meter^2}) qu'il y a dans \num{20} kilomètre carré (\unit{\kilo\meter^2}) :
\begin{equation*}
\qty{20}{\kilo\meter^2} = 20 \times \qty{1}{\kilo\meter\squared} = 20 \times (\qty{1}{\kilo\meter})^2 = \dots
\end{equation*}
\item Même question, mais sur le nombre de mètres cube dans \num{60} kilomètre cube.
\item Combien y a-t-il de secondes dans une heure ?
\item Une voiture roule à \qty[per-mode = symbol]{72}{\kilo\meter\per\hour}, quelle est sa vitesse en \unit[per-mode = symbol]{\meter\per\second} ? On pourra compléter l'égalité suivante :
\begin{equation*}
\qty[per-mode = fraction]{72}{\kilo\meter\per\hour} = 72 \times \dfrac{\qty{1}{\kilo\meter}}{\qty{1}{\hour}} = \dots 
\end{equation*}
\item Une moto double la voiture en roulant à \qty[per-mode = symbol]{25}{\meter\per\second}. Quelle est sa vitesse en \unit[per-mode = symbol]{\kilo\meter\per\hour} ? \emph{(Indication : Si $\qty{1}{\kilo\meter} = \qty{1000}{\meter}$, alors $\qty{1}{\meter} = \qty[parse-numbers = false]{1/1000}{\kilo\meter}$)}
\item La vitesse de la lumière est de \qty[per-mode = symbol]{299 792 458}{\meter\per\second}, quelle est-elle en \unit[per-mode = symbol]{\kilo\meter\per\hour}? 
\end{alphaquestions}
\end{document}