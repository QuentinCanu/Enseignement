\documentclass{poly}
\usepackage{main}

\author{Seconde 3}
\date{}
\title{Calcul Littéral}

\begin{document}
\maketitle
\section{Introduction : Lire un cours de maths}
En mathématiques, un cours est composé de \textbf{définitions}, de résultats (comme des \textbf{propositions}, des \textbf{propriétés} ou des \textbf{théorèmes}), de \textbf{remarques} ainsi que d'\textbf{exemples} ou \textbf{exercices}. 

Les définitions décrivent ce que sont les objets.
\begin{tcolorbox}
Une \textbf{définition} permet l'introduction d'un concept nouveau en mathématiques. Il utilise des définitions déjà connues pour construire quelque chose de nouveau.
\tcblower
\textbf{Il faut connaitre ses définitions pour comprendre les énoncés des exercices.}
\end{tcolorbox}
Les propositions, les propriétés et les théorèmes décrivent ce que font les objets.
\begin{tcolorbox}[arc=0mm, colback=black!10]
Une \textbf{proposition}, une \textbf{propriété} ou un \textbf{théorème} est un \textbf{résultat} à propos des objets définis plus tôt. Ce résultat est vrai parce qu'il a été démontré.
\tcblower
\textbf{Il faut connaître les résultat du cours pour pouvoir résoudre les exercices.}
\end{tcolorbox}
\begin{example}
\hfill
\begin{itemize}
\item Un triangle rectangle est un triangle dont l'un des angles est droit. C'est une \textbf{définition}.
\item Le \textbf{Théorème de Pythagore} est un résultat à propos des triangles rectangles. Il nous en apprend un peu plus sur l'objet \og triangle rectangle \fg.
\end{itemize}
\end{example}
\begin{tcolorbox}[colback=white]
Dans un cours, il y a aussi des remarques pour bien aider à comprendre, il ne faut pas les négliger.
\end{tcolorbox}

% \begin{remark}
% Pour bien apprendre un cours de maths, il faut identifier les différentes parties du cours: 
% \begin{itemize}
% \item Les \textbf{définitions} sont à connaître par \textbf{cœur}.
% \item Les \textbf{propositions} sont à comprendre. Pour cela, il faut savoir \textbf{refaire} les \textbf{exemples} données par le cours.
% \item Les \textbf{théorèmes} sont des proposition importantes, elle nécessitent d'être connues.
% \item Les \textbf{remarques} permettent de mieux comprendre les concepts du cours, il ne faut pas les négliger lors de la lecture du cours.
% \item Les \textbf{Exemples} illustrent directement les notions introduites. Il faut savoir les \textbf{refaire}.
% \end{itemize}
% \end{remark}
\newpage
\section{Expressions et égalité}
\subsection{Expressions}
\begin{definition}
Une \textbf{expression algébrique} est composée de nombres et d'opérations mathématiques ($+, -, \times, \dots$).
\end{definition}
Les expressions servent à écrire des formules. Les nombres sont soit connus (on utilise la plupart du temps des chiffres pour les écrire), soit indéterminés (on utilise des lettres pour les désigner).
\begin{remark}
On ignore les divisions ($/$) pour le moment : on préfèrera travailler avec les expressions fractionnaires plus tard.
\end{remark}
\begin{example}
\hfill
\begin{itemize}
\item L'expression $2x + 5$ signifie que le nombre indéterminé $x$ est multiplié par $2$, et que l'on ajoute $5$ au résultat.
\item La formule $\pi r^2$ permet de calculer l'aire d'un disque de rayon $r$. Ici, $r$ est indéterminé (car cette formule marche pour n'importe quel disque quelque soit son rayon), et $\pi$ est une constante (c'est la même quantité quoiqu'il arrive).
\end{itemize}
\end{example}
\begin{exercize}
Pour chacune des situations suivantes, donner l'expression correspondante :
\begin{alphaquestions}
\item On achète une orange à $10$€ et trois pommes qui coûtent $p$€ chacune. Quel est le prix total $P$ ? $P = \answerline{}$
\item On considère un rectangle de longueur $12$ et de largeur $l$. Que vaut son aire $A$ ? $A = \answerline$
\item Le marié envoie $n$ dragées à ses $47$ invités, et la mariée en envoie $m$ à ses $35$ invités. Soudain, il réalise qu'il a oublié ses parents, et il envoie un paquet de $100$ dragées pour se faire pardonner. Combien de dragées ont été envoyées en tout ? \answerline
\item Dans un jeu de société, il y chaque joueur possède deux dés transparents ainsi qu'autant de dés noirs qu'il y a de joueurs dans ce jeu. On suppose qu'il y a $m$ joueurs, combien y a-t-il de dés en tout ? \answerline 
\end{alphaquestions}
\end{exercize}

\newpage
\subsection{Egalité : Substitution}

\begin{proposition}
Soient deux nombres $x$ et $y$ tels que $x = y$. Alors toute expression comportant $x$ peut-être transformée en la même expression en remplaçant les $x$ par $y$. On parle de \textbf{substitution}.
\end{proposition}
\begin{example}
On suppose que $x = y$. Alors l'expression $2x + 1$ peut-être transformé en $2y + 1$.
\end{example}
\begin{remark}
\hfill
On pourrait aussi substituer $2y + 1$ en $2x + 1$: \textbf{Une égalité marche dans les deux sens}
\end{remark}
On peut aussi faire des substitutions à partir d'expressions plus avancées.
\begin{method}[Substitution d'expression sans se tromper avec les parenthèses]
\end{method}
On suppose $x = a + 5$, comment transformer l'expression $x^2 +3x$ ?
\begin{enumerate}
\item On souhaite transformer $x$, on met tous les $x$ entre parenthèses: $(x)^2 + 3(x)$
\item On efface $x$: $()^2 + 3()$
\item On remplace par $a + 5$: $(a + 5)^2 + 3(a + 5)$ 
\end{enumerate}

\begin{exercize}
En utilisant les égalités données, transformer l'expression correspondante:
\begin{alphaquestions}
\item $e = f$: $84e + 62$ devient \answerline
\item $x = 2z + 4y$: $-x^2 + 5z$ devient \answerline
\item $l + L = P$: $\sqrt{l + L} + \dfrac{1}{l + L}$ devient \answerline
\item $a = b + d$ et $b = c + d$: $5a + 2e$ devient \answerline
\item $2c + 3a = n - 5u$: $2(2c + 3a)^2 - 3(n - 5u)^2$ devient \answerline
\end{alphaquestions}
\end{exercize}

\newpage
\section{Développement et factorisation}
\subsection{Développement}
\begin{definition}
Une expression littérale est sous forme \textbf{développée réduite} si elle correspond à une \textbf{somme de termes} avec le moins de termes possible.
\end{definition}
\begin{method}
Pour développer et réduire une expression, on procède comme suit :
\begin{enumerate}
\item On utilise la distributivité de la multiplication sur l'addition pour transformer les produits contenant des \og parenthèses de sommes \fg 
\begin{equation*}
\tikzmarknode{a}{a}(\tikzmarknode{b}{b}+\tikzmarknode{c}{c}) = ab + ac
\end{equation*}
\begin{tikzpicture}[overlay, remember picture]
\draw (a.north) to[bend left=75] (b.north);
\draw (a.north) to[bend left=75] (c.north);
\end{tikzpicture}

\begin{equation*}
(\tikzmarknode{a1}{a} + \tikzmarknode{b1}{b})(\tikzmarknode{c1}{c} + \tikzmarknode{d1}{d}) =
ac + ad + bc + bd
\end{equation*}
\begin{tikzpicture}[overlay, remember picture]
\draw (a1.north) to[bend left=75] (c1.north);
\draw (a1.north) to[bend left=75] (d1.north);
\draw (b1.south) to[bend right=75] (c1.south);
\draw (b1.south) to[bend right=75] (d1.south);
\end{tikzpicture}
\item Une fois toutes les distributions finies, on ajoute entre eux tous les termes ayant les mêmes produits d'indéterminées.
\begin{equation*}
5x + 12 + 7rx -3x + rx + r^2 = 2x + 12 + 8rx + r^2
\end{equation*}
\end{enumerate}
\end{method}
\begin{example}
Développer et réduire les expressions suivantes :
\begin{alphaquestions}
\item $4(5a + 2b)$
\item $(7x + 2)(3x - 4)$
\item $-4(2x - 1) + (-5 + 4x)(8 + 5x)$
\item $(a - b)(a + b)$
\end{alphaquestions}
\end{example}

\newpage
\subsection{Factorisation}
\begin{definition}
Une expression est \textbf{factorisée} s'il s'agit d'un produit.
\end{definition}
\begin{method}
Pour factoriser une somme, la première méthode consiste à identifier un facteur commun dans chaque terme à factoriser
\begin{equation*}
\underline{a}b + \underline{a}c = \underline{a}(b + c)
\end{equation*}
Parfois il faut chercher ce facteur commun:
\end{method}
\begin{example}
On souhaite factoriser l'expression
\begin{equation*}
10x + 6y
\end{equation*}
On remarque que $10 = 2 \times 5$ et $6 = 2 \times 3$. Alors,
\begin{equation*}
\underline{2} \times 5x + \underline{2} \times 3 y = 2 (5x + 3y)
\end{equation*}
\end{example}
\begin{exercize}

Factoriser les expressions suivantes :
\begin{enumerate}[label=\emph{\alph*)}]
\item $5a + 10b =$ \answerline
\item $-8y^2 + y =$ \answerline
\item $21x - 28x^2 =$ \answerline
\item $35p - 42q =$ \answerline
\item $x(3x - 2) + 10(3x - 2) =$ \answerline
\end{enumerate}
\end{exercize}

\section{Identités remarquables}
\begin{proposition}
Soient $a$ et $b$ deux nombre réels quelconques. Alors,
\begin{equation*}
\begin{aligned}
&(a + b)^2 = a^2 + 2ab + b^2\\ 
&(a - b)^2 = a^2 - 2ab + b^2\\ 
&(a + b)(a - b) = a^2 - b^2\\ 
\end{aligned}
\end{equation*}
\end{proposition}
\begin{example}
Développer les expression suivantes:
\begin{alphaquestions}
\item $(c-1)(c+1)=$ \answerline
\item $(x+4)^2=$ \answerline
\item $(x-4)^2=$ \answerline
\item $y^2 - 64 =$ \answerline
\end{alphaquestions}    
\end{example}
\newpage
\section{Expressions Fractionnaires}
\begin{definition}[Expression fractionnaire]
\begin{equation*}
\text{Expression Fractionnaire} = \dfrac{\text{Numérateur}}{\text{Dénominateur} \neq 0}
\end{equation*}
\end{definition}
\begin{example}

Les expressions $\dfrac{3}{4}$; $\dfrac{x}{3-x}$ et $\dfrac{a + b}{c - d}$ sont des expressions fractionnaires.

L'expression $\dfrac{x^2 - 1}{0}$ ne l'est pas à cause du $0$.
\end{example}

\subsection{Simplification de fractions}
\begin{example}
Simplifier les fractions suivantes :
\begin{enumerate}[label=\emph{\alph*})]
\item $\dfrac{35}{42} =$  \answerline
\item $\dfrac{10a^2(1 + b)}{5a(2 + b)} = $ \answerline
\item $\dfrac{a^2 + 2a + 1}{a^2 - 1} =$ \answerline
\item $\dfrac{49 - x^2}{x^2-14x+49} =$ \answerline
\end{enumerate}
\end{example}
\subsection{Dénominateurs communs}
\begin{proposition}[Formule universelle]
Soient $\dfrac{a}{b}$ et $\dfrac{c}{d}$ deux expressions fractionnaires. Alors les expressions fractionnaires
\begin{equation*}
\dfrac{ad}{bd} \text{ et } \dfrac{bc}{bd}    
\end{equation*}
ont le même dénominateur.
\end{proposition}
\begin{remark}
Cette formule est à utiliser \textbf{en dernier recours}, si vous ne voyez pas comment mettre deux expressions fractionnaires au même dénominateur.
\end{remark}
\begin{example}
Simplifier les expressions suivantes : 
\begin{enumerate}[label=\emph{\alph*)}]
\item $\dfrac{2}{3} + \dfrac{13}{6} =$ \answerline
\item $\dfrac{9}{15} - \dfrac{35}{25} = $
\answerline
\item $\dfrac{3}{p(q-1)} + \dfrac{5}{(q-1)} = $ \answerline 
\item $\dfrac{y + 1}{y - 2} - \dfrac{y - 3}{y^2 - 4y + 4} =$ \answerline
\end{enumerate}
\end{example}

\subsection{Égalité d'expressions fractionnaires}
\begin{remark}
Pour comparer deux expressions fractionnaires $\dfrac{a}{b}$ et $\dfrac{c}{d}$ :
\begin{enumerate}
\item On les mets au même dénominateur, et on compare les numérateurs;
\item On vérifie que $ad = bc$. 
\end{enumerate}
\end{remark}
% \begin{proposition}
% \hfill
% \begin{itemize} 
% \item Pour ajouter deux expressions fractionnaires, leurs dénominateurs doivent être \textbf{identiques}. La formule générale d'addition de fractions est donnée par :
% \begin{equation*}
% \dfrac{a}{b} + \dfrac{c}{d} = \dfrac{ad + cb}{bd}   
% \end{equation*}
% \item Pour multiplier deux fractions, on multiplie leur numérateurs et leur dénominateurs :
% \begin{equation*}
% \dfrac{a}{b} \times \dfrac{c}{d} = \dfrac{ac}{bd}    
% \end{equation*}
% \item Deux fractions sont égales si et seulement si leur produit en croix donne deux quantités égales.
% \begin{equation*}
% \dfrac{a}{b} = \dfrac{c}{d} \Leftrightarrow ad = bc
% \end{equation*}
% \end{itemize}
% \end{proposition}
% \begin{remark}
% Ces trois résultats sont à utiliser si vous ne voyez pas d'autres solutions.
% \end{remark}
% \begin{example}
% Différents cas que vous allez croiser\newline
% \begin{tcolorbox}
% \begin{minipage}{0.45\textwidth}
% Somme de deux fractions avec \textbf{une} fraction à changer :
% \begin{equation*}
% \dfrac{1+y}{1+x} + \dfrac{x}{y(1+x)} = \dfrac{y(1+y) + x}{y(1+x)}     
% \end{equation*}
% Multiplication d'une fraction par une expression non fractionnaire :
% \begin{equation*}
% (1+x)\dfrac{2}{y} = \dfrac{2(1+x)}{y}   
% \end{equation*}    
% \end{minipage}
% \hfill\vline\hfill
% \begin{minipage}{0.45\textwidth}
% Somme de deux fractions avec \textbf{deux} fractions à changer:
% \begin{equation*}
% \dfrac{1+a}{a} - \dfrac{a}{1+a} = \dfrac{(1+a)^2 - a^2}{a(1+a)}
% \end{equation*}
% Produit en croix pour isoler une variable :
% \begin{equation*}
% V = \dfrac{d}{t} \Leftrightarrow d = \dfrac{t}{V}
% \end{equation*}
% \end{minipage}
% \end{tcolorbox}
% \end{example}
\end{document}