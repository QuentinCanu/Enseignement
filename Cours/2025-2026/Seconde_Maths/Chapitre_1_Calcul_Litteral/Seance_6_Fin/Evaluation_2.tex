\documentclass{controle}
\usepackage{main}

\title{Évaluation n°2 : Développement, Factorisation}
\author{Seconde 3}
\date{19 Septembre 2025}

\begin{document}
\maketitle
\begin{questions}
\titledquestion{Identités remarquables}[2]
Compléter l'égalité suivante :

$a^2 - b^2 =$
\begin{solution}[1cm]
$a^2 - b^2 = (a - b)(a + b)$
\end{solution}
\titledquestion{Développements}[6]
Développer et réduire les expressions suivantes :
\begin{parts}
\part[2] $11 - (a - 3)(4 + a) =$
\begin{solution}[2cm]
$11 - (a - 3)(4 + a) = 11 - (4a + a^2 - 12 - 3a) = 11 - 4a - a^2 + 12 + 3a = 23 - a^2 - a$
\end{solution}
\part[2] $2x + (x + 10)(x - 7) =$
\begin{solution}[2cm]
$2x + (x + 10)(x - 7) = 2x + (x^2 - 7x + 10x - 70) = 2x + x^2 - 7x + 10x - 70 = 5x + x^2 - 70$  
\end{solution}
\part[2] $(v - 6)(w - 7) =$
\begin{solution}[2cm]
$(v - 6)(w - 7) = vw - 7v - 6w + 42$
\end{solution}
\end{parts}
\titledquestion{Factorisation}[2]
Factoriser l'expression suivante :

$42x^2 + 56x =$
\begin{solution}[2cm]
$42x^2 + 56x = 7 \times 6 x x + 7 \times 8 x = 7x(6x + 8)$
\end{solution}
\end{questions}

\newpage
\maketitle
\begin{questions}
\titledquestion{Identités remarquables}[2]
Compléter l'égalité suivante :

$(a - b)^2 =$
\begin{solution}[1cm]
$(a - b)^2 = a^2 - 2ab + b^2$
\end{solution}
\titledquestion{Développements}[6]
Développer et réduire les expressions suivantes :
\begin{parts}
\part[2] $-6 + (h - 9)(8 + h) =$
\begin{solution}[2cm]
$-6 + (h - 9)(8 + h) = -6 + (8h + h^2 - 72 - 9h) = -6 + 8h + h^2 - 72 - 9h = -78 - h + h^2$
\end{solution}
\part[2] $4y - (-y + 6)(y + 3) =$
\begin{solution}[2cm]
$4y - (-y + 6)(y + 3) = 4y - (-y^2 - 3y + 6y + 18) = 4y + y^2 + 3y - 6y - 18 = y + y^2 - 18$  
\end{solution}
\part[2] $(p - 2)(q - 10) =$
\begin{solution}[2cm]
$(p - 2)(q - 10) = pq - 10p - 2q + 20$
\end{solution}
\end{parts}
\titledquestion{Factorisation}[2]
Factoriser l'expression suivante :

$27y^2 - 45y =$
\begin{solution}[2cm]
$27y^2 - 45y = 9 \times 3 y y - 9 \times 5 y = 9y(3y - 9)$
\end{solution}
\end{questions}

\end{document}