\documentclass{controle}
\usepackage{main}


\title{Évaluation n°7 : Fonctions, images, antécédents}
\author{Seconde 3}
\begin{document}
\maketitle
\begin{questions}
\titledquestion{Définitions}[2]
Soit $f$ une fonction numérique à valeurs réelles, et $a,b$ deux nombres tels que:
\begin{equation*}
f(a) = b
\end{equation*}
Compléter la phrase suivante :
\begin{center}
\begin{quote}
$a$ est \dotfill de $b$ par la fonction $f$.
\end{quote}
\end{center}
\titledquestion{Abonnements}[8]
Un service de transport urbain met en place un abonnement : une carte coûtant \num{70}€, mais où chaque trajet coûte seulement \num{11}€. Cependant, le nombre de trajets est limité à \num{120} durant la durée de l'abonnement.

On note $a(x)$ le prix total dépensé en ayant effectué un nombre $x$ de trajets.
\begin{parts}
\part[2] Quel est l'ensemble de définition de la fonction $a$ ?
\part[2] Exprimer $a(x)$ en fonction de $x$.
\part[4] Résoudre l'équation $a(x) = 752$. Interpréter votre résultat dans le contexte de l'exercice.
\end{parts} 
\end{questions}
\newpage
\maketitle
\begin{questions}
\titledquestion{Définitions}[2]
Soit $f$ une fonction numérique à valeurs réelles, et $a,b$ deux nombres tels que:
\begin{equation*}
f(a) = b
\end{equation*}
Compléter la phrase suivante :
\begin{center}
\begin{quote}
$b$ est \dotfill de $a$ par la fonction $f$.
\end{quote}
\end{center}
\titledquestion{Baignoire}[8]
Une baignoire pleine de \qty{180}{\liter} se vide de son eau : chaque minute, elle perd \qty{2}{\liter} de son contenu. Par chance, la fuite est réparée au bout de \num{55} minutes. 

On note $c(t)$ le contenu en litres de la baignoire après qu'il se soit passé $t$ minutes.
\begin{parts}
\part[2] Quel est l'ensemble de définition de la fonction $c$ ?
\part[2] Exprimer $c(t)$ en fonction de $t$.
\part[4] Résoudre l'équation $c(t) = 106$. Interpréter votre résultat dans le contexte de l'exercice.
\end{parts} 
\end{questions}
\end{document}