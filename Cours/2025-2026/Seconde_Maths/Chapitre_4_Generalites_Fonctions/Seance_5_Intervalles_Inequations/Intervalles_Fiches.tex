\documentclass{poly}
\usepackage{main}

\title{Intervalles}
\author{Seconde 3}
\date{26 Novembre 2025}

\begin{document}
\maketitle

\begin{definition}
Un \textbf{intervalle} est un ensemble de nombres $x$ compris entre une ou deux bornes.

On distingue plusieurs types d'intervalles :
\begin{itemize}
\item L'intervalle $[a;b]$ est l'ensemble des nombres $x$ tels que $a \leq x \leq b$;
\item L'intervalle $]-\infty;b]$ est l'ensemble des nombres $x$ tels que $x \leq b$;
\item L'intervalle $[a;+\infty[$ est l'ensemble des nombres $x$ tels que $a \leq x$ 
\end{itemize}
\end{definition}
\begin{example}
\hfill
\begin{itemize}
\item L'intervalle $[1;5]$ est l'ensemble de tous les nombres compris entre $1$ et $5$.
\item L'intervalle $[1;+\infty[$ est l'ensemble de tous les nombres supérieurs ou égaux à $1$. 
\end{itemize}
\end{example}
\begin{exercize}
\hfill
\begin{alphaquestions}
\item Écrire l'ensemble de tous les nombres inférieurs ou égaux à $-2$ sous forme d'intervalle.

\item Donner l'exemple de $5$ nombres appartenant à l'intervalle $[4;7]$.
\end{alphaquestions}
\end{exercize}

\begin{definition}
Les exemples précédents peuvent être modifiés pour qu'une des bornes ne soit pas comprise dans l'intervalle. Par exemple, $[a;b[$ est l'ensemble de tous les nombres $x$ vérifiant $a \leq x < b$.
\end{definition}
\begin{remark}
\hfill
\begin{itemize}
\item Ainsi, un crochet \textbf{intérieur} représente l'inclusion d'une borne (comme une main qui attrape le nombre), alors qu'un crochet \textbf{extérieur} représente l'exclusion de la borne (la main refuse d'attraper le nombre).
\item Les symbole $-\infty$ et $+\infty$ sont toujours exclus: le crochet devant ces symboles doit être tourné vers l'\textbf{extérieur}.
\end{itemize}
\end{remark}
\begin{exercize}
Décrire l'ensemble de tous les nombres $x$ vérifiant l'inégalité $11 < x < 17$ sous la forme d'intervalle.
\end{exercize}

\end{document}