\documentclass{controle}
\usepackage{main}

\title{Évaluation n°8 : Intervalles, Résolution graphique d'équation}
\author{Seconde 3}
\date{28 Novembre 2025}

\begin{document}
\maketitle
\version
\begin{questions}
\titledquestion{Intervalles}[4]
Pour chacune des inégalités suivantes, dire à quel intervalle appartient la variable $x$ :
\begin{parts}
\part[1] $-3 \leq x < 7$
\part[1] $18 \leq x$
\part[1] $x < -5$
\part[1] $x \geq 9$
\end{parts}
\vspace*{1cm}
\titledquestion{Équations}[6]
Soit $f$ et $g$ deux fonctions définies sur l'intervalle $[-10;10]$ dont les courbes représentatives $\mathcal{C}_f$ et $\mathcal{C}_g$ sont représentées ci-après:
\begin{center}
\begin{tikzpicture}
\tikzmath{\xmax = 5.25; \ymax = 3.25;};
\draw[help lines] (-\xmax,-\ymax) grid[step=0.5] (\xmax,\ymax);

\draw[axis] (-\xmax,0) -- (\xmax,0) node[right] {$x$};
\draw[axis] (0,-\ymax) -- (0,\ymax) node[above] {$y$};
\draw (0.5,0.1) -- (0.5,-0.1) node[below] {$1$};
\draw (0.1,0.5) -- (-0.1,0.5) node[left] {$1$};

\draw (-5,2) node {$\bullet$};
\draw (-5,2) parabola bend (-2.5,1) (-0.5,2);
\draw (-0.5,2) parabola bend (1,3) (2,2);
\draw (2,2) -- (5,-0.5) node {$\bullet$} node[below right] {$\mathcal{C}_f$};

\draw (-5,-2.5) node {$\bullet$};
\draw (-5,-2.5) parabola bend (-2,-3) (-0.5,-1); 
\draw (-0.5,-1) parabola bend (2.5,2) (5,0.5);
\draw (5,0.5) node {$\bullet$} node[above right] {$\mathcal{C}_g$}; 
\end{tikzpicture}
\end{center}
\begin{parts}
\part[1] Donner graphiquement l'image de $-5$ par $f$.
\part[1] Donner graphiquement l'image de $5$ par $f$.
\part[2] Résoudre graphiquement l'équation $f(x) = 4$.
\part[2] Résoudre graphiquement l'équation $g(x) = -2$.
\end{parts}
\end{questions}
\newpage
\maketitle
\version
\begin{questions}
\titledquestion{Intervalles}[4]
Pour chacune des inégalités suivantes, dire à quel intervalle appartient la variable $x$ :
\begin{parts}
\part[1] $4 < x < 13$
\part[1] $-6 \leq x$
\part[1] $x \leq 8$
\part[1] $x > 9$
\end{parts}
\vspace*{1cm}
\titledquestion{Équations}[6]
Soit $f$ et $g$ deux fonctions définies sur l'intervalle $[-10;10]$ dont les courbes représentatives $\mathcal{C}_f$ et $\mathcal{C}_g$ sont représentées ci-après:
\begin{center}
\begin{tikzpicture}
\tikzmath{\xmax = 5.25; \ymax = 3.25;};
\draw[help lines] (-\xmax,-\ymax) grid[step=0.5] (\xmax,\ymax);

\draw[axis] (-\xmax,0) -- (\xmax,0) node[right] {$x$};
\draw[axis] (0,-\ymax) -- (0,\ymax) node[above] {$y$};
\draw (0.5,0.1) -- (0.5,-0.1) node[below] {$1$};
\draw (0.1,0.5) -- (-0.1,0.5) node[left] {$1$};

\draw (-5,-2) node {$\bullet$};
\draw (-5,-2) parabola bend (-2.5,-1) (-0.5,-2);
\draw (-0.5,-2) parabola bend (1,-3) (2,-2);
\draw (2,-2) -- (5,0.5) node {$\bullet$} node[below right] {$\mathcal{C}_f$};

\draw (-5,-2.5) node {$\bullet$};
\draw (-5,-2.5) parabola bend (-2,-3) (-0.5,-1); 
\draw (-0.5,-1) parabola bend (2.5,2) (5,0.5);
\draw (5,0.5) node {$\bullet$} node[above right] {$\mathcal{C}_g$}; 
\end{tikzpicture}
\end{center}
\begin{parts}
\part[1] Donner graphiquement l'image de $-5$ par $f$.
\part[1] Donner graphiquement l'image de $5$ par $f$.
\part[2] Résoudre graphiquement l'équation $f(x) = -4$.
\part[2] Résoudre graphiquement l'équation $g(x) = -2$.
\end{parts}
\end{questions}
\end{document}