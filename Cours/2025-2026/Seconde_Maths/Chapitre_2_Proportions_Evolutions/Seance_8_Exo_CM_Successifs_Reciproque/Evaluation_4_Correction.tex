\documentclass{controle}
\usepackage{main}

\title{Évaluation n°4 : Évolutions \& Coefficients multiplicateurs}
\date{10 Octobre 2025}
\author{Seconde 3}

\printanswers

\begin{document}
\maketitle
\version
\begin{questions}
\titledquestion{Répondre aux questions suivantes}[10]
\begin{parts}
\part[2] Multiplier par $1,23$ revient à faire quelle évolution en pourcentage ?
\begin{solution}[2cm]
Multiplier par $1,23$ revient à faire une augmentation de $t = (CM - 1) \times 100 = (1,23 - 1) \times 100 = 23\%$.
\end{solution}

\part[2] Une ville de \num{52000} habitants en 2024 observe un nombre d'habitants de \num{43000} en 2025. Quel est le taux d'évolution en pourcentages du nombre d'habitants dans cette ville ?
\begin{solution}[2cm]
On calcule le taux d'évolution $t = \dfrac{V_f - V_d}{V_d} \times 100 = \dfrac{43000 - 52000}{52000} \times 100 $
\end{solution}

\part[2] Ce matin, la bouteille de sirop de Léa contient \qty{0,35}{\liter} de médicament. Le soir, elle vide son contenu de \num{87}\%. Quel est le contenu en litres de sa bouteille ?
\begin{solution}[2cm]
On calcule le coefficient multiplicateur associé à une diminution de \num{87}\% : $CM = 1 + \dfrac{t}{100} = 1 - \dfrac{87}{100} = 0,13$

On multiplie la valeur initiale par le coefficient multiplicateur : $V_f = CM \times V_d = 0,13 \times 0,35$
\end{solution} 

\part[2] Le nombre d'abonnements à un mensuel de sport, initialement de \num{4200} par mois, a augmenté de $33\%$ durant le mois. Quel est ce nombre après augmentation ?
\begin{solution}[2cm]
On calcule le coefficient multiplicateur associé à une augmentation de \num{33}\% : $CM = 1 + \dfrac{t}{100} = 1 + \dfrac{33}{100} = 1,33$

On multiplie la valeur initiale par le coefficient multiplicateur : $V_f = CM \times V_d = 1,33 \times 4200$
\end{solution} 

\part[2] Le prix d'un téléphone, après augmentation de $15\%$, est de \num{450}€. Quel est son prix avant l'augmentation ?
\begin{solution}[2cm]
On calcule le coefficient multiplicateur associé à l'augmentation de $15\%$ : $CM = 1 + \dfrac{t}{100} = 1 + \dfrac{15}{100} = 1,15$.

On divise la valeur finale par ce coefficient : $V_d = \dfrac{V_f}{CM} = \dfrac{450}{1,15}$.
\end{solution} 
\end{parts}
\end{questions}

\newpage
\maketitle
\version
\begin{questions}
\titledquestion{Répondre aux questions suivantes}[10]
\begin{parts}
\part[2] Multiplier par $0,8$ revient à faire quelle évolution en pourcentage ?
\begin{solution}[2cm]
Multiplier par $0,8$ revient à faire une diminution de $t = (CM - 1) \times 100 = (0,8 - 1) \times 100 = -20\%$.
\end{solution}

\part[2] On observe une température moyenne de \qty{34}{\celsius} le matin dans une salle. Le soir, cette température moyenne de \qty{38}{\celsius}. Quel est le taux d'évolution en pourcentages de la température de la salle ?
\begin{solution}[2cm]
On calcule le taux d'évolution $t = \dfrac{V_f - V_d}{V_d} \times 100 = \dfrac{38 - 34}{34} \times 100$
\end{solution}

\part[2] Les actions d'une entreprise du CAC40 valent \num{77500}€. Leur valeur observe une augmentation de $3\%$ durant la nuit. Quelles est leur valeur après la nuit ?
\begin{solution}[2cm]
On calcule le coefficient multiplicateur associé à une augmentation de \num{3}\% : $CM = 1 + \dfrac{t}{100} = 1 - \dfrac{3}{100} = 1,03$

On multiplie la valeur initiale par le coefficient multiplicateur : $V_f = CM \times V_d = 1,03 \times 77500$
\end{solution} 

\part[2] Le nombre de membres d'une réseau social est de \num{68000} en Janvier. Entre Janvier et Mars, ce nombre a diminué de $41\%$. Quel est le nombre de membres en Mars ?
\begin{solution}[2cm]
On calcule le coefficient multiplicateur associé à une diminution de \num{41}\% : $CM = 1 + \dfrac{t}{100} = 1 - \dfrac{41}{100} = 0,59$

On multiplie la valeur initiale par le coefficient multiplicateur : $V_f = CM \times V_d = 0,59 \times 68000$
  \end{solution} 

\part[2] Le prix d'une tablette, après diminution de $15\%$, est de \num{540}€. Quel est son prix avant la diminution ?
\begin{solution}[2cm]
On calcule le coefficient multiplicateur associé à la diminution de $15\%$ : $CM = 1 + \dfrac{t}{100} = 1 - \dfrac{15}{100} = 0,85$.

On divise la valeur finale par ce coefficient : $V_d = \dfrac{V_f}{CM} = \dfrac{540}{0,85}$.
\end{solution}
\end{parts}
\end{questions}
\end{document}