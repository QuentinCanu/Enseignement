\documentclass{exos}
\usepackage{main}

\begin{document}
\begin{exercize*}
Dans une salle de cinéma, il y a \num{500} enfants, ce qui représente \num{46}\% des spectateurs de la salle.
\begin{alphaquestions}
\item Préciser la population et la sous-population étudiée.
\item Rappeler la formule de calcul d'une proportion en fonction des effectifs de la population et de la sous-population.
\item En déduire le nombre total de spectateurs dans la salle.
\end{alphaquestions}
\end{exercize*}
\vspace*{3cm}
\begin{exercize*}
Dans une salle de cinéma, il y a \num{500} enfants, ce qui représente \num{46}\% des spectateurs de la salle.
\begin{alphaquestions}
\item Préciser la population et la sous-population étudiée.
\item Rappeler la formule de calcul d'une proportion en fonction des effectifs de la population et de la sous-population.
\item En déduire le nombre total de spectateurs dans la salle.
\end{alphaquestions}
\end{exercize*}
\vspace*{3cm}
\begin{exercize*}
Dans une salle de cinéma, il y a \num{500} enfants, ce qui représente \num{46}\% des spectateurs de la salle.
\begin{alphaquestions}
\item Préciser la population et la sous-population étudiée.
\item Rappeler la formule de calcul d'une proportion en fonction des effectifs de la population et de la sous-population.
\item En déduire le nombre total de spectateurs dans la salle.
\end{alphaquestions}
\end{exercize*}
\vspace*{3cm}
\end{document}