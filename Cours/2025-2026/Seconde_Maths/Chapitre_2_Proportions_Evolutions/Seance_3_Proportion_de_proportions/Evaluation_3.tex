\documentclass{controle}
\usepackage{main}

% \printanswers
\title{Evaluation n°3 : Équations, Proportions}
\author{Seconde 3}
\date{26 Spetembre 2025}
\begin{document}
\maketitle
\begin{questions}
\titledquestion{Équations}[5]
Résoudre les équations suivantes. \textbf{Il n'est pas nécessaire de rédiger une phrase de réponse.}
\begin{parts}
\part $6x + 3 = 0$
\begin{solution}[2cm]
\begin{align*}
6x + 3 = 0& \Leftrightarrow 6x + 3 - 3 = 0 -3\\
&\Leftrightarrow 6x = -3\\
&\Leftrightarrow \dfrac{6x}{6} = \dfrac{-3}{6}\\
&\Leftrightarrow x = \dfrac{-1}{2}
\end{align*}
\end{solution}
\part $-4x + 8 = -6$
\begin{solution}[2cm]
\begin{align*}
-4x + 8 = -6& \Leftrightarrow -4x + 8 - 8 = -6 - 8\\
&\Leftrightarrow -4x = - 14\\
&\Leftrightarrow \dfrac{-4x}{-4} = \dfrac{-14}{-4}\\
&\Leftrightarrow x = \dfrac{7}{2}
\end{align*}
\end{solution}
\part $ 5x + 9 = -6x + 2$
\begin{solution}[2cm]
\begin{align*}
5x + 9 = -6x + 2& \Leftrightarrow 5x + 9 + 6x - 9 = -6x + 2 + 6x - 9\\
&\Leftrightarrow 11x = - 7\\
&\Leftrightarrow \dfrac{11x}{11} = \dfrac{-7}{11}\\
&\Leftrightarrow x = \dfrac{-7}{11}
\end{align*}
\end{solution}
\end{parts}
\titledquestion{Proportions}[5]
Pour chacune des situations suivantes, préciser la population et la sous-population étudiés, puis répondre à la question. \textbf{Indiquez clairement le calcul nécessaire au résultat}.
\begin{parts}
\part Un troupeau de bufles est composé de \num{789} individus, dont \num{330} sont des femelles. Quelle est la proportion \textbf{en pourcentages} de femelles dans ce troupeau ?
\begin{itemize}
\item Population : \fillin[Les bufles du troupeau]
\item Sous-Population : \fillin[Les femelles]
\end{itemize}
\begin{solution}[2cm]
\begin{equation*}
\dfrac{330}{789} \times 100
\end{equation*}
\end{solution}
\part Un stade accueille \num{1600} spectateurs. Parmi eux, \num{25}\% a consommé une boisson de la buvette. Combien de spectateurs ont consommé une boisson de la buvette ?
\begin{itemize}
\item Population : \fillin[Les spectateurs]
\item Sous-Population : \fillin[Les consommateurs de boissons]
\end{itemize}
\begin{solution}[2cm]
\begin{equation*}
1600 \times \dfrac{25}{100}
\end{equation*}
\end{solution}
\end{parts}
\end{questions}
\newpage
\maketitle
\begin{questions}
\titledquestion{Équations}[5]
Résoudre les équations suivantes. \textbf{Il n'est pas nécessaire de rédiger une phrase de réponse.}
\begin{parts}
\part $2x + 8 = 0$
\begin{solution}[2cm]
\begin{align*}
2x + 8 = 0& \Leftrightarrow 2x + 8 - 8 = 0 - 8\\
&\Leftrightarrow 2x = -8\\
&\Leftrightarrow \dfrac{2x}{2} = \dfrac{-8}{2}\\
&\Leftrightarrow x = -4
\end{align*}
\end{solution}
\part $3x - 12 = -4$
\begin{solution}[2cm]
\begin{align*}
3x - 12 = -4& \Leftrightarrow 3x -12 + 12 = -4 + 12\\
&\Leftrightarrow 3x = 8\\
&\Leftrightarrow \dfrac{3x}{3} = \dfrac{8}{3}\\
&\Leftrightarrow x = \dfrac{8}{3}
\end{align*}
\end{solution}
\part $ 22x + 13 = -2x - 4$
\begin{solution}[2cm]
\begin{align*}
22x + 13 = -2x - 4& \Leftrightarrow 22x + 13 + 2x -13 = -2x - 4 + 2x - 13\\
&\Leftrightarrow 24x = - 9\\
&\Leftrightarrow \dfrac{24x}{24} = \dfrac{-9}{24}\\
&\Leftrightarrow x = \dfrac{-3}{8}
\end{align*}
\end{solution}
\end{parts}
\titledquestion{Proportions}[5]
Pour chacune des situations suivantes, préciser la population et la sous-population étudiés, puis répondre à la question. \textbf{Indiquez clairement le calcul nécessaire au résultat}.
\begin{parts}
\part Une machine produit \num{567} crayon de couleur, dont \num{129} sont rouges. Quelle est la proportion \textbf{en pourcentages} de crayon rouges parmi ces crayons de couleur ? 
\begin{itemize}
\item Population : \fillin[Les crayons de couleur]
\item Sous-Population : \fillin[Les crayons rouges]
\end{itemize}
\begin{solution}[2cm]
\begin{equation*}
\dfrac{129}{567} \times 100
\end{equation*}
\end{solution}
\part Dans un roman comportant \num{10500} mots, \num{60}\% sont des verbes. Combien y-a-t-il de verbes dans ce roman
\begin{itemize}
\item Population : \fillin[Les mots du roman]
\item Sous-Population : \fillin[Les verbes]
\end{itemize}
\begin{solution}[2cm]
\begin{equation*}
10500 \times \dfrac{60}{100}
\end{equation*}
\end{solution}
\end{parts}
\end{questions}
\end{document}