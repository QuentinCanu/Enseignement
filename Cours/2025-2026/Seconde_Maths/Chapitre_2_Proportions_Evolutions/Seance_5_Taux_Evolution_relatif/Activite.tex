\documentclass{exos}
\usepackage{main}

\title{Évolutions : Coefficient multiplicateur}
\author{Seconde 3}
\date{1\ier{} Octobre}
\begin{document}
\maketitle

\section*{Introduction : Taux d'évolution}
\begin{exercize}
Une entreprise de produits cosmétiques observe un chiffre d'affaire de \num{800}€ en Septembre. En Octobre, ce chiffre d'affaire devient \num{1500}€.
\begin{alphaquestions}
\item De combien ce chiffre d'affaire à augmenté ?
\item À quel pourcentage de la somme de départ correspond cette augmentation ?
\end{alphaquestions} 
\end{exercize}
\section*{Etude de cas géométrique : augmentation en pourcentages}
\begin{exercize}
On pose un rectangle $ABCD$ tel que $AB = 1$ et $AD = a$.

\begin{center}
\begin{tikzpicture}
\coordinate (A) at (0,0);
\coordinate (B) at (5,0);
\coordinate (C) at (5,4);
\coordinate (D) at (0,4);

\draw (A) -- (B) -- (C) -- (D) -- cycle;
\draw (A) node[below left] {$A$};
\draw (B) node[below right] {$B$};
\draw (C) node[above right] {$C$};
\draw (D) node[above left] {$D$};
\draw (A) -- (B) node[midway, below] {$1$};
\draw (A) -- (D) node[midway, left] {$a$};
\end{tikzpicture}
\end{center}

\begin{alphaquestions}
\item Exprimer l'aire $\mathcal{A}$ de ce rectangle.
\item À quel pourcentage correspond un cinquième ?
\item Tracer un rectangle $BCEF$. On suppose que son aire vaut un cinquième de $\mathcal{A}$. Qu'en déduire des longueurs $EF$ et $CE$ ?
\item Exprimer l'aire du rectangle $ADEF$. Cette aire correspond à l'augmentation de l'aire $A$.
\item Bilan : Si je souhaite augmenter l'aire $\mathcal{A}$ de $35\%$, par quelle valeur dois-je multiplier $\mathcal{A}$ ? Et pour une augmentation de $t\%$ ?
\end{alphaquestions}
\end{exercize}
\end{document}