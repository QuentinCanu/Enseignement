\documentclass{exos}
\usepackage{main}

\title{Évolutions successives}
\date{8 Octobre 2025}
\author{Seconde 3}

\begin{document}
\maketitle


\section{Prix d'une télévision}

Un magasin d'électroménager vend des télévisions, dont le prix est fixé à \num{800}€ en 2022.

\begin{alphaquestions}
\item La gérante du magasin souhaite augmenter ce prix de \num{5}\% en 2023. Quel sera ce prix en 2023 ?
\item De nouveau, celle-ci souhaite augmenter ce prix de \num{4}\% en 2024. Quel sera ce prix en 2024 ?
\item Entre 2022 et 2024, le prix d'une télévision a-t-il augmenté de \num{9}\% ? Sinon, quel est le taux d'évolution du prix ?
\item Compléter le schéma suivant représentant l'évolution du prix d'une télévision entre 2022 et 2024 ?
\begin{center}
\begin{tikzpicture}
\node (A) at (0,0) {$800$};
\node (B) at (3,0) {$\dots$};
\node (C) at (6,0) {$\dots$};

\draw[->] (A) to[bend left] (B);
\draw[->] (B) to[bend left] (C);
\draw[->] (A) to[bend right] (C);
\draw (1.5,0.8) node {$\times \dots$};
\draw (4.5,0.8) node {$\times \dots$};
\draw (3,-1.3) node {$\times \dots$};
\end{tikzpicture}
\end{center}
\end{alphaquestions}

\section{Évolution des ventes}
Le magasin a vendu \num{3000} télévisions en 2022. Entre 2022 et 2023, ce nombre de ventes a diminué de \num{10}\%, puis il a augmenté de \num{10}\% entre 2023 et 2024.
\begin{alphaquestions}
\item Reproduire le schéma de la question précédente. En déduire le coefficient multiplicateur correspondant à l'évolution du nombre de ventes entre 2022 et 2024.
\item Retrouve-t-on le même nombre de ventes entre 2022 et 2024 ? Sinon, donne le taux d'évolution du nombre de ventes de télévisions entre 2022 et 2024.
\end{alphaquestions}

\section{Vente d'enceintes}

Le magasin vend aussi des enceintes. Il y a trois catégories d'enceintes, qui coûtaient toutes les trois le même prix en 2022.
\begin{enumerate}[label = \Alph*)]
\item Le prix de cette enceinte a été soldé de \num{20}\% entre 2022 et 2023, puis de \num{10}\% entre 2023 et 2024.
\item Le prix de cette enceinte a été soldé de \num{15}\% entre 2022 et 2023, puis de \num{15}\% entre 2023 et 2024.
\item Le prix de cette enceinte a été soldé de \num{30}\% entre 2022 et 2024.
\end{enumerate}
Quelle enceinte coûte le moins cher en 2024 ?

\end{document}