\documentclass{article}
\usepackage{main}

\title{Progression Seconde 3}
\date{2025-2026}
\author{Quentin Canu}

\begin{document}
\maketitle
\section{Premier Trimestre}
\subsection*{Chapitre 1 : Calcul littéral, développement, factorisation, identités remarquables, équations}
\paragraph{Début le 3 Septembre}
\begin{itemize}
\item Expressions algébriques (2h) 
\item Grève (1h)
\item Egalité et substitution : Si $x = y$, alors $P(x) = P(y)$. Exemple : conversion d'unités (1h)
\item Grève (2h)
\item Développement, double distributivité (1h, Evaluation de cours n°1)
\item Entrainement développement (1h)
\item Factorisation, Identités remarquables (2h)
\item Application (1h, Evaluation de cours n°2)
\end{itemize}
\paragraph{Fin le 19 Septembre}
\subsection*{Chapitre 2 : Information chiffrée : Proportions et Evolutions}
\paragraph{Début le 22 Septembre}
\begin{itemize}
\item Equations (demi-groupe, 1h)
\item Populations, sous-populations, proportions (2h)
\item Proportions de proportions (Evaluation de cours n°3) (1h)
\item Rappels sur les fractions (demi-groupe) (1h)
\item Variation absolue, variation relative, taux d'évolution (2h)
\item Contrôle (1h) 
\item Exercices bilan (demi-groupe)
\item Taux d'évolutions successifs, taux d'évolution réciproques (2h)
\end{itemize}
\paragraph{Fin le 8 Octobre}
\subsection*{Chapitre 3 : Géométrie repérée}
\paragraph{Début le 10 Octobre}
\begin{itemize}
\item Rappels abscisses/ordonnées (Evaluation de Cours) (1h)
\item Distance entre deux points (Demi-Groupes) (1h)
\item Distance entre deux-points/Milieu d'un segment (2h)
\item Milieu d'un segment (1h) (Evaluation de Cours) 
\end{itemize}
\paragraph{Fin le 17 Octobre}
\end{document}