\documentclass{poly}
\usepackage{main}

\title{Chapitre 3 : Géométrie Repérée}
\author{Seconde 3}
\date{}

\begin{document}
\maketitle
\section{Repères orthonormées}
\begin{definition}
Un textbf{repère} est donné par trois points $(O;I;J)$ non alignés :
\begin{itemize}
\item Le premier point, $O$, est appelé l'\textbf{origine} du repère.
\item La droite $(OI)$ est appelée \textbf{axe des abscisses}.  
\item La droite $(OJ)$ est appelée \textbf{axe des ordonnées}.  
\end{itemize}
L'unité des abscisses est donnée par la longueur $OI$. L'unité des ordonnées est donnée par la longueur $OJ$.
\end{definition}
\begin{example}
Ci-dessous sont représentés deux repères différents.
\begin{center}
\begin{tikzpicture}
\begin{scope}
\draw (0,0) node {$\bullet$} node[left] {$O$};
\draw (0,2) node {$\bullet$} node[left] {$I$};
\draw (2,1) node {$\bullet$} node[left] {$J$};
\end{scope}
\begin{scope}[xshift=8cm, yshift=1cm]
\draw (0,0) node {$\bullet$} node[left] {$O$};
\draw (-1,0) node {$\bullet$} node[left] {$I$};
\draw (0,-1) node {$\bullet$} node[left] {$J$};
\end{scope}
\end{tikzpicture}
\end{center}
\end{example}
\begin{exercize*}
\hfill
\begin{alphaquestions}
\item Tracer pour chacun des repères précédents son axe des abscisses et son axe des ordonnées.
\item À l'aide d'une règle graduée, donner approximativement la longueur en \unit{\centi\meter} de l'unité des abscisses et de l'unité des ordonnées de chaque repère.
\end{alphaquestions}
\end{exercize*}
\newpage
\begin{remark}
Pour calculer les coordonnées d'un point $M$ sur un repère $(O;I;J)$, il faut tracer un \textbf{parallélogramme} $OPMQ$, avec $P$ un point de $(OI)$ (axe des abscisses) et $Q$ un point de $(OJ)$ (axe des ordonnées).
\begin{itemize}
\item L'abscisse de $M$ est obtenu grâce à la longueur $OP$. Son signe est positif si $P$ appartient à la demi-droite $[OI)$. 
\item L'ordonnée de $M$ est obtenu grâce à la longueur $OQ$. Son signe est positif si $P$ appartient à la demi-droite $[OJ)$. 
\end{itemize}
\end{remark}
\begin{example}
Soit $(O;I;J)$ le repère représenté ci-contre.
\begin{center}
\begin{tikzpicture}[scale=1.5]
\draw (0,0) node {$\bullet$} node[below] {$O$};
\draw (2,0) node {$\bullet$} node[below] {$I$};
\draw (2,1) node {$\bullet$} node[above left] {$J$};
\draw[->, dashed] (-2,0) -- (4,0);
\draw[->, dashed] (-2,-1) -- (4,2);
\draw (2,0) ++ (1,0.5) node {$\times$} node[above right] {$M$};
\end{tikzpicture}
\end{center}
\end{example}
\begin{exercize*}
Tracer le parallélogramme $OPMQ$ tel que présenté plus tôt. Quelles sont les coordonnées de $M$ ?
\end{exercize*}
\begin{definition}
Soit $(O;I;J)$ un repère.
\begin{itemize}
\item On dit que le repère est \textbf{orthogonal} si l'axe des abscisses et l'axe des ordonnées sont perpendiculaires : $(OI) \perp (OJ)$.
\item On dit que le repère est \textbf{normé} si les longueurs $OI$ et $OJ$ sont les mêmes : $OI = OJ$.
\item On dit que le repère est \textbf{orthonormé} s'il est à la fois orthogonal et normé. 
\end{itemize}
\end{definition}
\begin{exercize*}
Tracer un repère orthonormé $(O;I;J)$, et y placer le point $M$ de coordonées $(-1;2)$.
\end{exercize*}
\vspace*{4.5cm}
\begin{tcolorbox}
À partir de maintenant, les repères $(O;I;J)$ rencontré dans ce cours sont orthonormés.
\end{tcolorbox}
\newpage
\section{Milieu d'un segment}
\begin{proposition}
Soit $(O;I;J)$ un repère du plan, et $A,B$ deux points du plan. On considère $I$ le milieu du segment $[AB]$. Alors $I$ a pour coordonnées
\begin{equation*}
\left(\dfrac{x_A + x_B}{2}; \dfrac{y_A + y_B}{2}\right)
\end{equation*}
\end{proposition}
\begin{remark}
Pour obtenir respectivement l'abscisse et l'ordonnée du milieu d'un segment $[AB]$, on fait respectivement la moyenne des abscisses et des ordonnées de $A$ et $B$.
\end{remark}
\begin{exercize*}
\hfill
\begin{alphaquestions}
\item Tracer un repère orthonormé $(O;I;J)$, puis placer les points $A(2;2)$; $B(-1;0)$; $C(0;-1)$ et $D(3;1)$.
\item Calculer les coordonnées du milieu du segment $[AC]$ et du segment $[BD]$.
\item Que remarquez-vous ? En déduire la nature du quadrilatère $ABCD$.
\end{alphaquestions}
\end{exercize*}
\newpage
\section{Distance entre deux points}
\begin{proposition}
Soit $(O;I;J)$ un repère \textbf{orthonormé} du plan, ainsi que $A(x_A;y_A)$ et $B(x_B;y_B)$ deux points du plan. Alors, la longueur du segment $[AB]$ est donnée par la formule suivante :
\begin{equation*}
AB = \sqrt{(x_A - x_B)^2 + (y_A - y_B)^2}
\end{equation*}
\end{proposition}
\begin{remark}
Cette formule est simplement une conséquence du théorème de Pythagore. On peut le voir à l'aide de la figure suivante :
\begin{center}
\begin{tikzpicture}
\tikzmath{\x = 4.25; \k = 0.1;}
\draw[help lines] (-\x,-\x) grid[step=0.5] (\x,\x);
\draw[->,thick] (-\x,0) -- (\x,0) node[right] {$x$};
\draw[->,thick] (0,-\x) -- (0,\x) node[above] {$y$};
\draw (0,0) node[below left] {$O$};
\draw (1,\k) -- (1,-\k) node[below] {$I$};
\draw (\k,1) -- (-\k,1) node[left] {$J$};

\coordinate (A) at (1.5,1.5);
\coordinate (B) at (3,3.5);
\draw (A) node {$\bullet$} node[below left] {$A$};
\draw (B) node {$\bullet$} node[above right] {$B$};
\draw (A) -- (B);
\draw[dashed] (A) -| (B);
\draw let \p1 = (A), \p2 = (B) in ($(\x2, \y1) + (-2*\k,0)$) |- ($(\x2,\y1) + (0,2*\k)$);
\draw let \p1 = (A), \p2 = (B) in (\x1,\k) -- (\x1,-\k) node[below] {$x_A$} (\x2,\k) -- (\x2,-\k) node[below] {$x_B$};
\draw let \p1 = (A), \p2 = (B) in (\k,\y1) -- (-\k,\y1) node[left] {$y_A$} (\k,\y2) -- (-\k,\y2) node[left] {$y_B$};
\draw[<->] let \p1 = (A), \p2 = (B) in ($(A) + (0,-\k)$) -- ($(\x2,\y1) + (0,-\k)$) node[midway,below] {$x_B - x_A$};
\draw[<->] let \p1 = (A), \p2 = (B) in ($(\x2,\y1) + (\k,0)$) -- ($(B) + (\k,0)$) node[midway,below,sloped] {$y_B - y_A$};
\end{tikzpicture}
\end{center}
\end{remark}
\begin{exercize*}
Soit $(O;I;J)$ un repère orthonormé. On considère les points $A(4;10)$ et $B(5;-3)$. Calculer la longueur du segment $[AB]$.

\begin{equation*}
AB = 
\end{equation*}
\end{exercize*}
\end{document}