\documentclass{article}
\usepackage{main}

\title{Loi Physiques}
\author{Première Spécialité Mathématiques}
\date{Jeudi 4 Septembre}

\begin{document}
\maketitle

\section{Température}

L'unité de mesure des températures en France est le \emph{degré Celsius} (noté \unit{\celsius}). Mais l'unité de température du Système International d'Unité (SIU) est le \emph{Kelvin} (noté \unit{\kelvin}). La formule de conversion d'une température $T$ en degré Celsius en une température en degré Kelvin est donnée par:
\begin{equation*}
K(T) = 273,15 + T
\end{equation*}
\begin{alphaquestions}
\item La température moyenne en bord de mer est de \qty{15}{\celsius}. Quelle est sa température en Kelvin ?
\item La température moyenne du corps humain est environ de \qty{308}{\kelvin}. Quelle est sa température en degré Celsius ? 
\end{alphaquestions}
\section{Gaz Parfait}
La loi de Gay-Lussac affirme que dans un gaz parfait, la pression exprimée en \emph{Pascal} (noté \unit{\pascal}) est proportionnelle à la température en Kelvin.

Dans une bouilloire, un gaz parfait est à une température de \qty{300}{\kelvin} et une pression de \qty{46620}{\pascal}. Après un temps de chauffe, cette température devient \qty{380}{\kelvin} et cette pression devient \qty{59052}{\pascal}. 
\begin{alphaquestions}
\item Calculer le coefficient de proportionnalité entre la pression et la température.
\item Quelle était la pression dans la bouilloire quand la température était de \qty{330}{\kelvin} ?
\item Quelle était la température dans la bouilloire quand la pression était de \qty{50000}{\pascal} ?
\item Si $t$ est la température en Kelvin, en déduire l'expression de la pression en fonction de $t$.
\item On suppose que $T$ est la température en \textbf{Celsius}. Montrer que la pression en fonction de $T$ s'exprime à l'aide de la formule
\begin{equation*}
155,4T + 42447,51  
\end{equation*}
\item La pression en Pascal est-elle proportionnelle à la température en Celsius ?
\end{alphaquestions}
\section{Augmentation de température}
\begin{alphaquestions}
\item La température passe de \qty{20}{\celsius} à \qty{30}{\celsius}. Montrer que la pression augmente de $10 \times 155,4$ dans ce cas.
\item Que dire pour une température passant de \qty{20}{\celsius} à \qty{15}{\celsius} ?
\item La variation de température est-elle proportionnelle à la variation de pression ? Si oui, en déduire le coefficient de proportionnalité.
\end{alphaquestions}

\end{document}