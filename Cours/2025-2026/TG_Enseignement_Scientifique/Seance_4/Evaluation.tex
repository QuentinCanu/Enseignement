\documentclass{exos}
\usepackage{main}

\title{Critères d'évaluation : Le jury non-spécialiste}
\author{Enseignement scientifique en mathématiques}
\date{6 Novembre 2025}

\begin{document}
\maketitle
Vous êtes membre du jury de non-spécialistes pour l'épreuve de grand oral de vos camarades. Cette fiche va vous permettre de vous donner les critères d'évaluation de votre point de vue de non-expert de la matière présentée par votre camarade. Ces critères sont issus d'une grille d'évaluation à disposition des jurys du baccalauréat.

\section{Les critères}
\subsection{Qualité orale}
Ce critère désigne la capacité du candidat ou de la candidate à maîtriser son discours, son vocabulaire, sa voix. On mesure ici la qualité du candidat ou de la candidate à se faire comprendre de son auditoire.
\subsection{Qualité de la prise de parole en continu}
Ce critère désigne la capacité du candidat ou de la candidate à maîtriser son temps durant sa présentation de \qty{10}{\minute}, et à approfondir ses idées.
\subsection{Qualité de l'interaction}
Ce critère désigne la capacité du candidat ou de la candidate à échanger avec le jury, que ce soit dans la réponse aux questions ou les relances envers le jury.
\subsection{Qualité et construction de l'argumentation}
Ce critère désigne la capacité du candidat ou de la candidate à maîtriser les enjeux de son sujet, et à construire une argumentation raisonnée.
\subsection{Qualité des connaissances}
Ce critère concerne surtout le membre du jury spécialiste du sujet, et évaluer la capacité du candidat ou de la candidate à mobiliser ses connaissances du sujet.
\section{Votre note}
Vous allez être noté en tant que jury non-spécialiste. Pendant chaque présentation de chaque groupe, vous allez prendre des notes et me remettrez cette copie à la fin des présentations.

Pour chaque présentation, votre copie fera figurer au moins cinq questions, que vous poseriez en tant que membre non-spécialiste du jury. Vous serez évalués sur la pertinence de vos questions.
\end{document}