\documentclass{article}
\usepackage{main}

\title{Cours : Accueil, Programmation}
\date{4 Septembre 2024}
\author{Seconde 3}

\begin{document}
\maketitle

\section{Accueil des élèves en SNT}
\subsection{Matériel}
\begin{itemize}
\item Cahier pour la SNT
\item Ordinateur région
\end{itemize}
\subsection{Notes}
\begin{itemize}
\item Une note à la fin de chaque chapitre sous forme de projets ou d'exposés.
\item Quelques évaluations de cours de temps en temps.
\end{itemize}

\section{Cours : programmation}

\textbf{Chapitre $0$ sur la programmation.}

\subsection{Notion d'instructions séquentielles.}

\begin{definition}
Un programme est une suite d'instructions. Lorsque l'on exécute un programme, les instructions sont exécutées les unes après les autres.
\end{definition}
\begin{example}
Le programme contrôlant un feu tricolore peut être comparé à un programme exécutant infiniment les instructions suivantes: 
\begin{enumerate}
\item allumer l'ampoule rouge pendant un temps donné
\item allumer l'ampoule verte pendant un temps donné
\item allumer l'ampoule orange pendant un temps donné
\end{enumerate}
\end{example}

\begin{exercize}
Proposer un programme similaire pour les objets suivants:
\begin{itemize}
\item Distributeur automatique de billets (au moment d'insérer la carte).
\item Distributeur de boissons.
\item 
\end{itemize}
\end{exercize}

Un ordinateur est très naïf, il exécute dans l'ordre toutes les instructions d'un programme sans discuter.

\begin{example}
On peut simuler à la main l'exécution d'un programme par une flèche qui lit une à une les instructions.
\begin{enumerate}
\item Penser à un nombre.
\item Le multiplier par $2$.
\item Ajouter $10$ au résultat.
\item Diviser par $2$ le résultat.
\item Enlever le nombre de départ.
\end{enumerate}
\end{example}

\subsection{Variables}

\begin{definition}
Une \emph{variable} est un emplacement mémoire dédié à stocker une valeur. Cette valeur peut ensuite être réutilisée plus tard par une autre instruction.
\end{definition}

\begin{example}
On peut imaginer chaque variable comme des boîtes contenant une seule valeur. Quand une variable n'a pas encore été définie, la boite n'existe pas encore.

Le programme suivant définit deux variables nommées $a$ et $b$ qui contiennent respectivement $2$ et $3$.
\begin{enumerate}
\item $a \leftarrow 2$ 
\item $b \leftarrow 3$ 
\end{enumerate}
Que font les programmes suivants ?
\begin{enumerate}
\item 
\begin{enumerate}
    \item $a \leftarrow 2$ 
    \item $b \leftarrow 3$ 
    \item $a \leftarrow 5$ 
\end{enumerate}
\item
\begin{enumerate}
    \item $a \leftarrow 2$ 
    \item $b \leftarrow 3$ 
    \item $a \leftarrow a+b$ 
\end{enumerate}
\item 
\begin{enumerate}
\item $a \leftarrow 2$
\item $b \leftarrow 3$
\item $c \leftarrow b$
\item $b \leftarrow a$
\item $a \leftarrow c$ 
\end{enumerate}
\end{enumerate}
\end{example}

\begin{exercize}
Trouver un programme qui initialise deux variables nommées $a$ et $b$, et qui échange la valeur de ces deux variables, mais sans utiliser de variables temporaires $c$.
\end{exercize}

\section{Devoirs}
J'interrogerais quelqu'un sur un programme très simple.

\end{document}