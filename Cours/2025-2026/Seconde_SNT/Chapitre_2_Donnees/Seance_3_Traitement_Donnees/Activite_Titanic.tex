\documentclass{exos}
\usepackage{main}

\title{Les naufragés du Titanic}
\author{Seconde 3}
\date{17 Novembre 2025}

\begin{document}
\maketitle

\section{Étude fichier csv}
\begin{alphaquestions}
\item Télécharger le fichier titanic.csv sur Pronote (voir sur le cahier de textes); puis ouvrir le fichier à l'aide du bloc-note (clic-droit et sélectionner \textbf{Ouvrir avec\dots $\to$ Bloc-note}). 
\item Quels sont les descripteurs de cette table de données ? Indiquez le format des données concernées (texte, nombre, lettre\dots)
\item À quoi correspond le descripteur \og Survie \fg ? Comment interpréter les nombres associés à ce descripteur ?
\item Quel est le séparateur de donnée ici ?
\end{alphaquestions}

\section{LibreOffice Calc}

À l'aide de \emph{LibreOffice Calc}, ouvrir le fichier \emph{titanic.csv} (Vous pouvez télécharger \emph{titanic.csv}, puis faire un clic-droit et sélectionner \textbf{Ouvrir avec\dots $\to$ LibreOffice Calc}). Une fenêtre de mise en page devrait s'ouvrir. À l'aide de l'aperçu en bas, cochez la ou les cases correspondant au séparateur de données adéquats. Une fois que l'aperçu donne un rendu correct, vous pouvez valider l'import.

Répondre aux questions suivantes. Pour chacune des réponses, on indiquera la méthode de résolution utilisée.
\begin{alphaquestions}
\item Combien de personnes sont agées de 10 ans ?
\item Quel était le prix du billet le plus cher ?
\item Combien de femmes en première classe ont voyagé sur le Titanic ? Qui était la plus jeune d'entre elle ? La plus vieille d'entre elles ?
\item Déterminer le nombre de femmes et d'enfants ayant survécu au naufrage. 
\item Quelle est la moyenne d'âge des personnes voyageant en 3\ieme{} classe ?
\item Parmi les trois classes de voyageurs présents sur le Titanic, laquelle compte le plus de survivants ?
\end{alphaquestions}


\end{document}