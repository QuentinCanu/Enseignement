\documentclass{exos}
\usepackage{main}

\title{TP Python : Prise en main}
\author{Seconde 3}
\date{6 Octobre 2025}


\begin{document}
\maketitle

\section{Présentation de l'interface d'EduPython}
\begin{center}
\includegraphics[width=0.9\textwidth]{Interface_Edupython.png}
\end{center}
\section{Python, comme une calculatrice}
\begin{exercize}
Entrez dans \textbf{la console} les instructions suivantes :
\begin{lstlisting}[language=Python]
>> 3+4
>> 3-9
>> 5*2
>> 5**2
>> 31/5
>> 31//5
>> 31%5
\end{lstlisting}
À quoi servent les opérateurs \lstinline[language=python]!**!; \lstinline[language=python]!//! et \lstinline[language=python]!%! ? Tester sur d'autres exemples.
\end{exercize}
\section{Début en programmation}
\begin{exercize}
Nous avons déjà travaillé sur des programmes écrits en pseudo-code. On a notamment utilisé les instructions suivantes:
\begin{lstlisting}[mathescape=true]
Affectation : variable $\leftarrow$ valeur
Conditionnelle : Si Condition Alors Instruction1 Sinon Instruction2
Boucle : TantQue Condition Faire Instruction
\end{lstlisting}
Recopier le programme suivant en python, et dire à quel type d'instruction correspond chaque ligne. On prendra garde aux nombre d'espaces à ajouter au début de chaque ligne.
\begin{lstlisting}[language=Python]
a = 2 + 5
while a > 6:
  if a == 5:
    a = a - 2
  else:
    a = a - 4
print(a)
\end{lstlisting}
Executer le programme. À quoi sert la ligne \lstinline[language=Python]!print(a)! ?
\end{exercize}
\begin{tcolorbox}
\begin{remark}
Pour faire une affectation de variables, on utilise le symbole $=$ en python. Il ne faut pas le confondre avec l'égalité en mathématiques.

Pour tester que deux valeurs sont égales en Python, on utilise l'opérateur \lstinline[language=Python]!==!.
\end{remark}
\end{tcolorbox}
\begin{exercize}
\begin{alphaquestions}
\item Écrire un programme qui initialise deux variables \lstinline[language=Python]!p! et \lstinline[language=Python]!q! aux valeurs de votre choix, et qui appelle l'instruction \lstinline[language=Python]!print("p est plus grand que q")! si \lstinline[language=Python]!p > q!.
\item Écrire un programme qui initialise une variable \lstinline[language=Python]!a! à la valeur de votre choix, puis qui double la valeur de \lstinline[language=Python]!a! jusqu'à que celle-ci dépasse le nombre $100$.
\item Écrire un programme qui initialise une variable \lstinline[language=Python]!S! à la valeur de votre choix, puis qui augmente cette valeur de $24\%$, et répète cette opération jusqu'à que la valeur de \lstinline[language=Python]!S! dépasse le triple de la valeur initiale.
\end{alphaquestions}
\end{exercize}
\begin{exercize}
Recopier le programme de Syracuse vu plus tôt en classe. Pour rappel, il consiste à initialiser une variable au nombre de choix, puis de répéter une boucle tant que cette valeur ne contient pas la valeur $1$. À chaque tour de boucle, si la variable est paire, alors la diviser par $2$, sinon la multiplier par $3$ et ajouter $1$ au résultat.

Exécuter votre programmer sur de grands nombres. Le programme termine-t-il ?
\end{exercize}
\end{document}