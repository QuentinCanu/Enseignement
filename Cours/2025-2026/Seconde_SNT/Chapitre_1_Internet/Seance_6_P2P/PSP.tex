\documentclass{exos}
\usepackage{main}

\title{Peer-to-peer}
\date{3 Novembre 2025}

\begin{document}
\maketitle
\section{Messages}
Lors d'un conflit, 4 soldats sont envoyés dans des postes avancés différents : Caroline, David, Evangeline et Fabien. Tous les postes avancés sont reliés deux à deux par une ligne télégraphique permettant de communiquer en morse.

\begin{alphaquestions}
\item Faire un schéma représentant la situation. On représentera les postes avancés par des points $C, D, E$ et $F$, et les lignes télégraphiques par des segments entre les différents points.
\item Se renseigner sur le morse, puis écrire le mot \og MESSAGE \fg en morse.
\item Avant d'envoyer les soldats en mission, l'état-major indique à Fabien qu'il trouvera dans son poste avancé $3$ messages à envoyer aux autres postes, et précise aux autres que Fabien est le seul autorisé à envoyer des messages. Combien de messages Fabien devra-t-il envoyer ? Sachant que chaque message fait \num{1200} mots à envoyer, et que chaque mot prend \qty{1}{\second} à écrire en morse, combien de temps mettra Fabien à envoyer tous les messages ?
\item Tous les soldats sont maintenant autorisés à envoyer des messages à n'importe qui. C'est toujours Fabien qui possède l'intégralité des $3$ messages au début de la mission. Comment peuvent s'organiser les soldats avant de partir en mission pour que tout le monde reçoive les $3$ messages le plus rapidement possible ? On organisera la réponse sous la forme d'un tableau similaire au tableau suivant \textbf{(Prévoir de la place)}:
\begin{center}
\begin{tabular}{|c|c|c|c}
\hline
&Instant 1&Instant 2&\dots\\  
\hline
Caroline&&&\\  
\hline
David&&&\\  
\hline
Evangeline&Message 1 à David&&\\  
\hline
Fabien&Message 1 à Evangeline&Message 2 à Evangeline&\\  
\hline
\end{tabular}
\end{center} 
\end{alphaquestions}
\section{Le Peer-to-peer}
\begin{alphaquestions}
\item Donner la définition d'une architecture client-serveur, et la définition d'une architecture peer-to-peer.
\item Quels sont les avantages d'une architecture peer-to-peer ?
\item Quels sont les usages du peer-to-peer en pratique ?
\item Que risque-t-on en utilisant le peer-to-peer pour des pratiques illégales ?
\end{alphaquestions}

\end{document}