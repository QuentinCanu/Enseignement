\documentclass{exos}
\usepackage{main}

\title{URL (Uniform Resource Locator) et HTTP (HyperText Transfert Protocol)}
\author{Seconde 3}
\date{8 Janvier 2026}

\begin{document}
\maketitle

\section{Definitions}
Le \textbf{web} est un des nombreux services sur Internet. Il permet d'accéder à des ressources situés sur les différents ordinateurs du réseau.

Pour cela se met en place une architecture client-serveur :
\begin{itemize}
\item Le \textbf{navigateur} fait office de client web. Il assure la communication avec le serveur web, et il sert aussi d'interpréteur HTML (HyperText Markup Language)
\item Le \textbf{serveur web} est un ordinateur distant du réseau, qui héberge différentes ressources comme des pages en HTML, en CSS, des images, des vidéos\dots  
\end{itemize}

Le client et le serveur communiquent grâce au protocole \textbf{HTTP} (HyperText Transfert Protocol). Celui-ci peut-être remplacé par le protocole HTTPS, où le S signifie \og Secured \fg : l'échange d'informations est chiffrée afin de ne pas pouvoir être intercepté et lu.

Le client web requiert de la part du serveur une ressource qui est identifiée à l'aide de son \textbf{URL} (Uniform Resource Locator). Cette chaîne de caractère identifie le protocole utilisé, l'adresse symbolique du serveur dans le réseau, et l'emplacement du fichier sur le serveur.

\section{URL}
On donne un exemple d'URL : \url{https://fr.wikipedia.org/wiki/Informatique}
\begin{itemize}
\item \url{https} : le protocole utilisé pour obtenir la resource;
\item \url{://} : le serveur est distant, c'est-à-dire qu'il est sur le réseau internet;
\item \url{fr.wikipedia.org} : l'adresse symbolique du serveur; il est associé à son adresse IP sur le réseau;
\item \url{wiki/Informatique} : l'emplacement de la ressource dans les dossiers et fichier du serveur.
\end{itemize}
\begin{alphaquestions}
\item Identifier les différentes parties de l'URL suivante : \url{https://fr.wikipedia.org/wiki/Sanglier}
\item Identifier les différentes parties de l'URL suivante : \url{https://www.1jour1actu.com/sport/les-elephants-vont-gagner}
\item Identifier les différentes parties de l'URL suivante : \url{file:///Users/utilisateur0/Desktop/Revisions.pdf}
\item Identifier les différentes parties de l'URL suivante : \url{http://192.168.1.1}
\end{alphaquestions}
\section{HTTP}

Le protocole HTTP (HyperText Transfert Protocol) est un ensemble de façons de formuler une demande à un serveur, et un ensemble de réponses possible.

Une fois la communication avec le serveur établie grâce à son adresse IP (ou symbolique), le client peut envoyer une demande dans le format requis par le protocole. Il y a différents types de demandes. En voici un exemple.

\begin{lstlisting}[language=html]
GET /mondossier/monFichier.html HTTP/1.1
User-Agent : Mozilla/5.0
Accept : text/html
\end{lstlisting}

\begin{alphaquestions}
\item Comment s'appelle le fichier requis par le navigateur dans cet exemple ? Où se trouve-t-il sur le serveur ?

La réponse du serveur est donnée sous le format suivant.

\begin{lstlisting}[language=html]
HTTP/1.1 200 OK
Date: Thu, 15 feb 2019 12:02:32 GMT
Server: Apache/2.0.54 (Debian GNU/Linux) DAV/2 SVN/1.1.4
Connection: close
Transfer-Encoding: chunked
Content-Type: text/html; charset=ISO-8859-1
<!doctype html>
<html lang="fr">
<head>
<meta charset="utf-8">
<title>Voici mon site</title>
</head>
<body>
<h1>Hello World! Ceci est un titre</h1>
<p>Ceci est un <strong>paragraphe</strong>. Avez-vous bien compris ?</p>
</body>
</html>
\end{lstlisting}

\item À quelle date la demande a été formulée ?
\item Quel est le titre de la page web obtenue ?
\item Que va-t-il s'afficher sur le navigateur ?
\end{alphaquestions}
\end{document}