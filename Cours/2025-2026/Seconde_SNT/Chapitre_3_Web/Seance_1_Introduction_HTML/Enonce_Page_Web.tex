\documentclass{exos}
\usepackage{main}

\title{Votre propre page Web}
\date{À rendre pour le 5 Janvier}
\author{Seconde 3}

\begin{document}
\maketitle

Pour le 3 Janvier 2025, votre travail consiste à faire votre propre page web. Le rendu sera individuel. 

\begin{tcolorbox}[title=Sujet]
\textbf{Le sujet de votre choix, à condition qu'il permette à monsieur Canu d'apprendre quelque chose qu'il ne savait pas.}
\end{tcolorbox}

\section*{Liste des critères}
\subsection*{Structure de la page}
\begin{enumerate}
\item Deux pages en tout: une page principale et une page secondaire, accessible via un lien hypertexte présent sur la page principale.
\item Chaque page commencera par un titre principal.
\item La page principale comportera plusieurs paragraphes, tous précédés d'un titre secondaire.
\end{enumerate}
\subsection*{Texte}
\begin{enumerate}
\item De manière générale, votre texte fera un total de mille mots au minimum.
\item Il devra y avoir dans votre texte cinq variations de police d'écriture: taille de police, souligné, gras, italique, couleur de texte\dots
\end{enumerate}
\subsection*{Couleur}
\begin{enumerate}
\item Le fond de votre page devra être d'une couleur différente que le blanc.
\item L'ensemble de votre page doit être lisible. 
\end{enumerate}
\subsection*{Images}
\begin{enumerate}
\item Votre page web devra comporter un minimum de deux images\dots
\end{enumerate}
\subsection*{Bonus}
Tout élément supplémentaire apportant de la plus-value à votre page sera récompensé.
  
\end{document}