\documentclass{controle}
\usepackage{main}

\title{Evaluation de cours n°1}
\author{Seconde 3}
\date{12 Septembre 2025}

\begin{document}
\maketitle
\begin{questions}
\titledquestion{Expressions}
Pour chacun des énoncés ci-dessous, donner l'expression mathématique demandée \textbf{(Une justification n'est pas nécessaire)}:
\begin{parts}
\part Quelle est l'expression de l'aire d'un rectangle de longueur $5$ et de largeur $3+l$ ?
\part Un boulanger se fournit en farine et levure. Il prend \qty{150}{\kilo\gram} de farine coûtant $f$ € le kilo, et \qty{50}{\kilo\gram} de levure coûtant $l$ € le kilo. Donner l'expression de la dépense totale du boulanger.
\part Une centrale électrique peut fournir une puissance de \qty{5000}{\watt} au total. Un hôpital nécessite \qty{300}{\watt} pour fonctionner, et on note $n$ le nombre d'hôpitaux de la région. Il y a aussi \num{3} écoles nécessitant chacune une puissance de $p$. Donner l'expression de la puissance restante à la centrale après avoir de l'électricité aux écoles et aux hopitaux. 
\end{parts}
\makeemptybox{5cm}
\titledquestion{Égalité et substitutions}
À l'aide des égalités données, procéder à une ou plusieurs substitutions dans les expressions correspondantes.
\begin{parts}
\part $p = 2q$ dans $3p + 4p^2$ 
\part $r = 8c - 5$ dans $\sqrt{8c - 5} + 2(8c - 5)^3$ 
\part $5d + 2 = a^2$ dans $\dfrac{2a + 2b}{5d + 2}$ 
\end{parts}
\makeemptybox{4cm}
\end{questions}
\newpage



\maketitle
\begin{questions}
\titledquestion{Expressions}
Pour chacun des énoncés ci-dessous, donner l'expression mathématique demandée \textbf{(Une justification n'est pas nécessaire)}:
\begin{parts}
\part Quelle est l'expression de l'aire d'un rectangle de longueur $8c$ et de largeur $4$ ?
\part Un magasin de bricolage reçoit une commande de vis et de clous. Le pack de clous vaut $c$ €, et le magasin a reçu $650$ packs, et le pack de vis vaut $v$ € et la magasin a reçu $300$ packs. Quelle est l'expression du montant total payé par le magasin ?
\part Une baignoire contient \qty{700}{\liter} d'eau. On utilise $c$ casseroles de \qty{5}{\liter} chacun ainsi que \num{50} marmites de $m$ \unit{\liter} chacun pour retirer de l'eau de cette baignoire. Donner l'expression du volume d'eau restant dans cette baignoire. 
\end{parts}
\makeemptybox{5cm}
\titledquestion{Égalité et substitutions}
À l'aide des égalités données, procéder à une ou plusieurs substitutions dans les expressions correspondantes.
\begin{parts}
\part $x = z + 1$ dans $5x - 4x^2$ 
\part $r = p \times q$ dans $\dfrac{12}{5(p \times q)^4}$ 
\part $l^2 = j + 7k$ dans $\sqrt{j + 7k}$ 
\end{parts}
\makeemptybox{4cm}
\end{questions}
\end{document}