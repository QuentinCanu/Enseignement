\documentclass{article}
\usepackage{main}

\author{}
\date{26 Février 2024}
\title{Exercices : Coefficient directeur}

\begin{document}
\maketitle
\begin{enumerate}[label=\textbf{ Exercice \arabic*}]
\item Soit $f$ une fonction affine. Pour chacune des situations suivantes, déterminer le coefficient directeur de $f$.
\begin{enumerate}[label=\emph{\alph*)}]
\multido{}{4}{
\item $f(\random{-5}{5}) = \random{-5}{5}$ et $f(\random{-5}{5}) = \random{-5}{5}$}
\end{enumerate}
\item Soit $f$ une fonction affine telle que $f(\random{1}{7}) = \random{-2}{2}$ et $f(\random{1}{7}) = \random{-2}{2}$.
\begin{enumerate}[label=\emph{\alph*)}]
\item Calculer le coefficient directeur de $f$.
\item Calculer son ordonnée à l'origine.
\end{enumerate}
\item Un restaurant propose un menu à $20$ \euro{}. Le responsable constate que baisser le prix du menu de $20$ centimes augmente le nombre moyen de clients à midi de $5$.
\begin{enumerate}[label=\emph{\alph*)}]
\item Combien y a-t-il de clients en moyenne quand le prix du menu est à $19,80$ \euro{}. Et $19,60$ \euro{} ?
\item On note $x$ le nombre moyen de clients et $f(x)$ le prix du menu en \euro{}. Déterminer $m$ et $n$ tel que $f(x) = mx + n$.
\item Quel sera le prix si le nombre de clients est de $120$ ?
\item Quel sera le nombre de clients moyen si le prix est de $16$ \euro{} ? 
\end{enumerate}
\end{enumerate}
\end{document}