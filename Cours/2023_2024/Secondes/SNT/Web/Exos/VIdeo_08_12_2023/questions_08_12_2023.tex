\documentclass{exam}
\usepackage{main}
\title{Web : Historique et fonctionnement}
\date{08 Décembre 2023}
\qformat{\textbf{Exercice \thequestion :}\hfill}
\begin{document}
\maketitle
\begin{questions}
\question Nous allons voir une video sur l'histoire du Web.
\begin{parts}
\part Qu'est-ce qu'un lien hypertexte ? Quelle peut être sa source ?
\part Résumer les liens existant entre l'hypertexte, les pages Web et les navigateurs.
\part Citer des différences entre pages Web statiques et dynamiques.
\end{parts}
\vspace{0.5cm}
\question Maintenant que nous avons des éléments historiques, passons au fonctionnement. Nous allons voir une vidéo sur l'architecture client-serveur, qui prend un autre sens que sur Internet.
\begin{parts}
\part Quel est le nom du programme faisant office de client Web ? Et de serveur Web ?
\part Quel est le nom du protocole régissant les échanges entre clients et serveur Web ?
\part Quel code d'erreur signifie qu'un serveur Web ne dispose pas de la ressource demandée par un client ? 
\end{parts}
\end{questions}
\end{document}