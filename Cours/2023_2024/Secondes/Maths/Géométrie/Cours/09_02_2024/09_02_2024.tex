\documentclass{article}
\usepackage{main}

\title{Cours : Géométrie repérée dans le plan}
\date{9 Février 2024}
\author{Quentin Canu}

\begin{document}
\maketitle
\section{Plan de Classe}
\section{Questions Flash}
Dessiner les points $(2;1)$, $(-3;5)$, $(1;-3)$ sur un repère.
\section{Activité}
Répondre à chacune des questions suivantes :
\begin{enumerate}
\item On dispose de $\qty{5}{\litre}$ d'eau à $\qty{28}{\degreeCelsius}$, et on y ajoute $\qty{5}{\litre}$ à $\qty{36}{\degreeCelsius}$. Quelle est la température des $\qty{10}{\litre}$ d'eau ?
\item Un ascenseur allant du 2\ieme sous-sol au 6\ieme étage est bloqué à mi-chemin. À quel étage est-il bloqué ?
\item On considère une droite graduée horizontale sur laquelle le point $A$ est placé sur la graduation $5$ et le point $B$ sur la graduation $13$. Sur quelle graduation se trouve le milieu du segment $[AB]$ ?
\item On considère une droite graduée verticale sur laquelle le point $A$ est placé sur la graduation $x$ et le point $B$ sur la graduation $y$. Quelle formule permet de calculer la graduation sur laquelle se trouve le milieu du segment ?
\item Soient deux points $A(x_A;y_A)$ et $B(x_B;y_B)$. Soit $I$ le milieu du segment $[AB]$. Comment calculer les coordonnées de $I$ ?
\end{enumerate}
\section{Cours}
Titre
\subsection*{Milieu d'un segment}
\begin{proposition}
Soient $A(x_A;y_A)$ et $B(x_B;y_B)$ deux points du plan. Soit $I$ le milieu du segment $[AB]$. Alors,
\begin{equation*}
I\left(\dfrac{x_A + x_B}{2};\dfrac{y_A + y_B}{2}\right)
\end{equation*}
\end{proposition}
\begin{remark}
Les coordonnées du milieu correspondent à la moyenne des coordonnées des extrémités du segment.
\end{remark}
\begin{example}
Soit $A(3;4)$ et $B(5;-2)$. Alors le milieu du segment $[AB]$ a pour coordonnées
\begin{equation*}
\left(\dfrac{3+5}{2};\dfrac{4-2}{2}\right) = \left(4;1\right)
\end{equation*}
\end{example}
\begin{remark}
\begin{itemize}
\item Un quadrilatère $ABCD$ est un parallélogramme si et seulement si ses diagonales $[AC]$ et $[BD]$ se coupent en leur milieu.
\item Un point $A$ est le symétrique d'un point $B$ par rapport à un point $C$ si et seulement si $C$ est le milieu du segment $[AB]$.
\end{itemize}
\end{remark}
\section{Exercices}
Exercices 33, 34, 35, 36 page 172.
Exercices 51, 54 page 174.
\end{document}