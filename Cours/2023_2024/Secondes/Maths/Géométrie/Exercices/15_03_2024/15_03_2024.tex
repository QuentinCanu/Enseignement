\documentclass{article}
\usepackage{main}

\title{Exercices}
\author{Seconde 9}
\date{15 Mars 2024}

\begin{document}
\maketitle
Ces exercices correspondent aux exercices 60 et 63 page 175 du manuel.
\begin{enumerate}[label=\textbf{Exercice \arabic* : }]
\item Dans un repère orthonormé, on considère $A(3;2)$, $B(9;4)$, $C(1;8)$ et $D(x;y)$ où $x$ et $y$ sont deux réels.
\begin{enumerate}
\item Montrer que $ABC$ est un triangle rectangle en $A$.
\item Calculer les coordonnées de $D$ pour que $ABDC$ soit un carré. 
\end{enumerate}
\item Dans un repère orthonormé, on considère les points $A(-2;3)$, $B(-3;1)$ et $C(4;0)$. On note $H$ le pied de la hauteur du triangle $ABC$ passant par $A$.
\begin{enumerate}
\item Faire une figure représentant la situation.
\item Montrer que le triangle $ABC$ est rectangle.
\item En déduire l'aire du triangle, puis la longueur $AH$.
\end{enumerate}
\end{enumerate}
\end{document}