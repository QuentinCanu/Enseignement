\documentclass{article}
\usepackage{main}

\title{Milieu d'un segment}
\date{9 Février 2024}
\author{}

\begin{document}
\maketitle
Répondre aux questions suivantes :
\begin{enumerate}
\item On dispose de $\qty{5}{\litre}$ d'eau à $\qty{28}{\degreeCelsius}$, et on y ajoute $\qty{5}{\litre}$ à $\qty{36}{\degreeCelsius}$. Quelle est la température des $\qty{10}{\litre}$ d'eau ?
\item Un ascenseur allant du 2\ieme sous-sol au 6\ieme étage est bloqué à mi-chemin. À quel étage est-il bloqué ?
\item On considère une droite graduée horizontale sur laquelle le point $A$ est placé sur la graduation $5$ et le point $B$ sur la graduation $13$. Sur quelle graduation se trouve le milieu du segment $[AB]$ ?
\item On considère une droite graduée verticale sur laquelle le point $A$ est placé sur la graduation $x$ et le point $B$ sur la graduation $y$. Quelle formule permet de calculer la graduation sur laquelle se trouve le milieu du segment ?
\item Soient deux points $A(x_A;y_A)$ et $B(x_B;y_B)$. Soit $I$ le milieu du segment $[AB]$. Comment calculer les coordonnées de $I$ ?
\end{enumerate}
\end{document}