\documentclass{exam}
\usepackage{main}
\title{Racines carrés, développements et factorisations}
\author{}
\date{15 Janvier 2024}
\qformatExos{}
\begin{document}
\maketitle
\begin{questions}
\question
\begin{parts}
\part \'Ecrire $A = 2\sqrt{90} - 4\sqrt{250} + 5\sqrt{40}$ sous la forme $a\sqrt{10}$ avec $a$ entier.    
\part \'Ecrire $B = 7\sqrt{72} - 2\sqrt{98} + 2\sqrt{128}$ sous la forme $b\sqrt{2}$ avec $b$ entier.    
\part \'Ecrire $C = 5\sqrt{242} - 7\sqrt{128} - 2\sqrt{50}$ sous la forme $c\sqrt{2}$ avec $c$ entier.    
\part \'Ecrire $D = 4\sqrt{96} - 2\sqrt{384} - 5\sqrt{150}$ sous la forme $d\sqrt{6}$ avec $d$ entier.    
\end{parts}
\vspace{0.5cm}
\question Développer les produits suivants :
\begin{parts}
\part $\left(-3\sqrt{6}-7\right)\left(8\sqrt{6}+5\right)$
\part $\left(9\sqrt{3}+5\right)\left(2 +5\sqrt{3}\right)$
\part $\left(4\sqrt{2}+5\right)\left(4 +4\sqrt{2}\right)$
\part $\left(6\sqrt{7} + 5\right)^2$
\part $\left(-3\sqrt{7} - 3\right)^2$
\part $\left(-2\sqrt{6} + 5\right)\left(-2\sqrt{6} - 5\right)$
\end{parts}
\vspace{0.5cm}
\question Supprimer la racine carrée du dénominateur dans les fractions suivantes :
\begin{parts}
\part $\dfrac{10}{\sqrt{11}}$
\part $\dfrac{-5}{\sqrt{17}}$
\part $\dfrac{9}{8 + 6\sqrt{10}}$
\part $\dfrac{2}{9 + 2\sqrt{4}}$
\part $\dfrac{7}{3 - 3\sqrt{2}}$
\part $\dfrac{8}{\left(12 - \sqrt{6}\right)^2}$
\part $\dfrac{20}{(8 + \sqrt{3})(7 + \sqrt{12})}$
\end{parts}
\end{questions}
\end{document}

