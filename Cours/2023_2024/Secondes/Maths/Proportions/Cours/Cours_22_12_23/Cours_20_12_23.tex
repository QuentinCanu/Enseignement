\documentclass{article}
\usepackage{main}
\title{Populations et sous-population}
\date{22 Décembre 2023}
\author{Quentin Canu}

\begin{document}
\section{Fin fonctions}
Exercice sur les tableaux de valeurs : 74 page 234
\section{Charte}
\section{DM}
\`A rendre pour le vendredi $12$ Janvier.
\section{Biais du survivant}
Projeter l'avion.
\section{Activité}
\section{Cours}
\begin{vocabulary}
Population, individus, sous-population, série statistique, effectifs\dots
\end{vocabulary}
\begin{definition}
Soit une population $P$ et une sous-population $S$. On note $n_P$ l'effectif total (le nombre d'individus dans la population) et $n_S$ l'effectif de $S$ (le nombre d'individus dans la sous-population $S$). Alors la proportion $p$ de $S$ dans $P$ est donnée par
\begin{equation*}
p = \dfrac{n_S}{n_P}
\end{equation*}
\end{definition}
Faire un diagramme "patate".
\begin{itemize}
\item Si on connait $p$ et $n_P$, alors $n_S = p \times n_P$.
\item Si on connait $p$ et $n_S$, alors $n_P = \dfrac{n_S}{p}$.
\item Si on souhaite la proportion en pourcentages, on fait $100 \times p$
\end{itemize}
\begin{example}
\begin{enumerate}
\item Un bouquet de fleurs est composé de $60$ fleurs. $40\%$ d'entre elles sont des roses. Combien y a-t-il de roses ?
\item Une collection de jeux videos comporte $12$ jeux PS5, ce qui correspond à $60\%$ de la collection totale.
\end{enumerate}
\end{example}
\end{document}