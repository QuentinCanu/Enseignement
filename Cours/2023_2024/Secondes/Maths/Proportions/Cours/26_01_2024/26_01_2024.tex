\documentclass{article}
\usepackage{main}

\title{Cours : Taux d'évoluation global}
\date{26 Janvier 2024}
\author{Quentin Canu}

\begin{document}
\maketitle
\section{Question Flash}
\begin{enumerate}
\item Résoudre $4x + 3 = 2x - 7$.
\item On augmente $V_d = 15$ de $60\%$, puis on diminue de $25\%$ le résultat, qu'obtient-on au final ?
\end{enumerate}
\section{Correction du contrôle}
\section{Vérification des cahiers}
\section{Correction de l'activité législatives}
\section{Correction des exercices 26 et 27}
Si nécessaire.

\section{Cours}
\subsection*{Taux d'évolution global, taux d'évolution réciproque}
\subsubsection*{Taux d'évolution global}
Faire un dessin.
\begin{definition}
Soient $CM_1$, $CM_2$ \dots plusieurs coefficients multiplicateurs. On appelle \emph{coefficient multiplicateur global} le produit $CM_g = CM_1 \times CM_2 \times ...$. Alors, le \emph{taux d'évolution global} est défini par $t_g = CM_g - 1$.
\end{definition}
\begin{example}
La température moyenne du mois de Janvier augmente de $5\%$ en Février, puis de $12\%$ en Mars. Quel est le taux d'évolution global de la température entre Janvier et Mars ?
\end{example}
\section{Devoirs}
Exercice 64 page 323.
\end{document}