\documentclass{article}
\usepackage{main}

\title{Cours : proportions de proportions}
\author{Quentin Canu}
\date{12 Janvier 2024}
\begin{document}
\maketitle
\section{Questions Flash (10 min)}
\begin{enumerate}
\item Une étude de $2005$ estime que le taux de gauchers en France est de $12,9\%$. La population française était de $63$ millions en 2005. Donnez la quantité de gauchers correspondante.
\item Un bouquet de fleurs comporte des tulipes rouges, ainsi que 15 tulipes bleues, ce qui constitue $30\%$ du bouquet. Combien y a-t-il de tulipes rouges dans le bouquet ?
\end{enumerate}
\section{Choses à dire (2min)}
\begin{itemize}
\item DM pour le lundi 15.
\item DS à rattraper pour les absents.
\end{itemize}
\section{Activité (30min)}
\url{Maternite.pdf}
\section{Cours}
\subsection*{Proportions de proportions}
\begin{proposition}
Etant donnée une population $E$, une sous-population $A$ dans $E$ de proportion $p$ ainsi qu'une sous-population $B$ dans $A$ de proportion $p'$, alors la proportion de $B$ dans $E$ est donnée par $pp'$.
\end{proposition}
Faire le dessin.
\begin{example}
Dans un lycée de $1000$ élèves, il y a $52\%$ de filles, et parmi elles, $47\%$ font de l'espagnol. Combien y a-t-il de filles faisant de l'espagnol dans ce lycée ? Environ $24,44\%$.    
\end{example}
\section{Cours : Taux d'Evolution (15 min)}

\section{Exercices}
Exercice 57 puis 56 page 321. 
\end{document}
