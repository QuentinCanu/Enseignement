\documentclass{exam}
\usepackage{mainExam}

\title{Correction : Exercices à savoir faire}
\date{24 Mai 2024}
\author{Seconde 9}

\begin{document}
\maketitle
\begin{questions}
\question
Résoudre une inéquation à une inconnue de degré $1$.

Exemple : Résoudre l'inéquation $3x - 4 \leq 2x + 2$. Il faudra donner l'ensemble des solutions sous la forme d'un intervalle.
\question
Etude d'une fonction à partir de sa courbe. Étant donné la courbe représentative d'une fonction, donner son ensemble de définition, son tableau de variation, son minimum, son maximum et les valeurs sur lesquels la fonction atteint son minimum et son maximum.
\question Étant donné deux points $A$ et $B$ dont les coordonnées peuvent être des fonction du temps $t$, donner et simplifier la formule de la distance $AB$ sur un exemple d'abord, et dans le cas général ensuite.

Exemple : $A(3;4)$ et $B(2-t;3+2t)$.
\begin{parts}
\part Quelle est la distance $AB$ quand $t = 1$ ?
\part Donner l'expression de la distance $AB$ en fonction de $t$. 
\end{parts}
\question Exprimer sous la forme d'une fonction une situation réelle. Etudier cette fonction à l'aide d'un tableau de variation et dessiner une courbe représentative possible.

Exemple : On étudie une route de montagne. Celle-ci monte en altitude durant les $50$ premiers kilomètres, puis redescend durant $60$ kilomètres pour atteindre la même altitude qu'au départ. On note $a(x)$ l'altitude au kilomètre $x$. Tracer un tableau de variation possible pour $a$, puis tracer une courbe représentative possible pour $x$ telle que $a(10) > a(100)$.
\question Soit $f$ une fonction à la fois paire et impaire. Donner la valeur de $f(1)$, $f(2)$ et $f(3)$.
\end{questions}
\end{document}