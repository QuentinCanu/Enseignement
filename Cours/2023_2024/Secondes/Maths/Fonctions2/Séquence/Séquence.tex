\documentclass{article}
\usepackage{main}

\title{Séquence sur les fonctions : sens de variation, extremum, parité}
\author{Quentin Canu}
\date{11 Mars 2024}

\begin{document}
\maketitle
\section{Contenus}
\begin{itemize}
\item Croissance, décroissance, monotonie d'une fonction définie sur un intervalle.
\item Tableau de variations.
\item Maximum, Minimum d'une fonction définie sur un intervalle.
\item Fonction paire, fonction impaire, traduction géométrique. 
\end{itemize}
\section{Attendus}
\begin{itemize}
\item Relier représentation graphique et tableau de variation.
\item Déterminer graphiquement les extremums d'une fonction.
\item Exploiter geogebra, la calculatrice ou Python pour décrire les variations d'une fonction donnée par une formule.
\end{itemize}
\section{Structure}
\subsection{Variation}
\subsubsection{Monotonie de fonction}
\begin{enumerate}
\item Définitions 
\end{enumerate}
\subsubsection{Tableaux de variation}
\begin{enumerate}
\item Construction d'un tableau de variation.
\item Lien avec représentation graphique.
\item Utilisation de logiciels et autres outils.
\end{enumerate}
\subsection{Extremums}
\begin{enumerate}
\item Définition.
\item Lecture graphique d'extremums. 
\end{enumerate}
\subsection{Parité}
\begin{enumerate}
\item Définition et exemples.
\item Représentation graphique.
\end{enumerate}
\section{Déroulement}
\subsection*{Lundi 18 Mars Matin : Inequations}
\begin{itemize}
\item Définition d'inéquation.
\item Vérifier une solution d'inéquations.
\end{itemize}
\subsection*{Lundi 18 Mars Après-Midi : Monotonie et tableau de variation}
\begin{itemize}
\item Cours : définition de la monotonie.
\item Exemples/Démonstration : fonction carrée, fonction racine, fonction inverse (début).
\end{itemize}
\subsection*{Vendredi 22 Mars 9h : Relier représentation graphique et tableau de variation}
\begin{itemize}
\item Fin démonstration.
\item Cours : tableau de variation.
\item Exercices : tableaux de variations de fonctions.
\end{itemize}
\subsection*{Vendredi 22 Mars 12h : Relier représentation graphique et tableau de variation}
\begin{itemize}
\item Utilisation d'un logiciel de géométrie dynamique : exemple de la fonction inverse.
\item Tableau de variation des fonctions de références : fonction inverse. Page 4 du poly.
\item Exercices ?
\end{itemize}
\subsection*{Lundi 25 Mars Matin : Inéquations}
\begin{itemize}
\item Exercices
\item Méthode de résolution d'une inéquation.
\item Forme de l'ensemble des solutions. 
\end{itemize}
\subsection*{Lundi 25 Mars Après-Midi : Extremums}
\begin{itemize}
\item Interrogation 
\item Activité
\item Définition
\item Exercices : déterminer graphiquement les extremums d'une fonction.
\end{itemize}
\subsection*{Vendredi 27 Mars : Exercices extremums}
\subsection*{Vendredi 5 Avril Matin : Parité / Utilisation de la calculatrice}
\subsection*{Vendredi 5 Avril Après-Midi : Récapitulatifs sur les tableaux : valeurs, signe, variation}
\end{document}