\documentclass{exam}
\usepackage[exos]{main}

\title{Évaluation de Cours}
\date{6 Mai 2024}
\author{Seconde 9}

\begin{document}
\maketitle
\thispagestyle{empty}
\paragraph{Version 1}
\begin{questions}
\question Résoudre dans $\R$ l'équation suivante :
\begin{equation*}
(32x + 17)(-23x-8)=0
\end{equation*}
On attachera de l'importance à la rédaction.
\question On place dans une urne $3$ boules rouges, $2$ boules vertes et $1$ boule noire. Compléter la loi de probabilité suivante.
\begin{center}
\begin{tabular}{|c|c|c|c|}
\hline
Univers $\Omega$ & Rouge & Verte & Noire\\
\hline
\rule[-0.4cm]{0pt}{1cm} Probabilité & & &\\
\hline
\end{tabular}
\end{center} 
\end{questions}
\newpage
\maketitle
\thispagestyle{empty}
\paragraph{Version 2}
\begin{questions}
\question Résoudre dans $\R$ l'équation suivante :
\begin{equation*}
(64x + 8)(-12x-40)=0
\end{equation*}
On attachera de l'importance à la rédaction.
\question On dessine un symbole \og Pierre \fg, \og Feuille \fg et \og Ciseaux \fg sur dix cartes que l'on place dans un chapeau. On a dessiné en tout $4$ \og Pierre \fg, $5$ \og Feuille \fg et $1$ \og Ciseaux \fg en tout. Compléter la loi de probabilité ci-dessous.
\begin{center}
\begin{tabular}{|c|c|c|c|}
\hline
Univers $\Omega$ & Pierre & Feuille & Ciseaux\\
\hline
\rule[-0.4cm]{0pt}{1cm} Probabilité & & &\\
\hline
\end{tabular}
\end{center}
\end{questions}
\end{document}