\documentclass{article}
\title{Phénomènes aléatoire : Séquence}
\date{}
\author{Quentin Canu}

\begin{document}
\maketitle
\section{Informations}
\subsection{Contenus}
\begin{itemize}
\item Fréquence Conditionnelle, Fréquence Marginale
\item Probabilités Conditionnelles
    \begin{itemize}
    \item Définition
    \item Notation
    \item Tableau croisé d'effectifs
    \item Arbre Pondéré 
    \end{itemize}
\item Indépendance d'événements, avec les probabilités conditionnelles.
\item Successions d'événements indépendants.
\end{itemize}
\subsection{Capacités attendues}
\begin{itemize}
\item Construire un tableau croisé d'effectifs.
\item Construire un arbre pondéré.
\item Interpréter un tableau croisé à l'aide des fréquences conditionnelles.
\item Calculer des fréquences conditionnelles. 
\item Calculer des fréquences marginales. 
\end{itemize}
\section{Structure du cours}
\begin{enumerate}
\item Fréquences et Probabilités
    \begin{enumerate}
    \item Rappels fréquences
    \item Loi des grands nombres
    \end{enumerate}
\item Probabilités conditionnelles
    \begin{enumerate}
    \item Définition et notation
    \item Calcul : arbres pondérés, tableaux croisés
    \end{enumerate}
\item Evenements indépendants
    \begin{enumerate}
    \item Définition et exemples.
    \item Succession d'événements indépendants.
    \end{enumerate}
\end{enumerate}
\section{Déroulement}
\begin{enumerate}
\item Paradoxes de probabilités : 11 Décembre 2023.
\item Rappels fréquences, applications aux probabilités : 18 Décembre 2023.
\item 8 Janvier 2024 : Probabilités conditionnelles : introduction, arbre pondéré.
\item 11 Janvier 2024 : Expression d'un gène.
\item 15 Janvier 2024 : Définition d'évenements indépendants.
\item 22 Janvier 2024 : exercices sur les événements indépendants.
\end{enumerate}
\end{document}